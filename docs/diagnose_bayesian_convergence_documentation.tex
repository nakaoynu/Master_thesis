% filepath: diagnose_bayesian_convergence_documentation.tex
\documentclass[a4paper,12pt,dvipdfmx]{jlreq}
\usepackage{amsmath,amssymb,amsthm}
\usepackage{bm}
\usepackage{physics}
\usepackage{tikz}
\usetikzlibrary{shapes.geometric, arrows, positioning, fit, backgrounds, calc}
\usepackage{algorithm2e}
\usepackage{booktabs}
\usepackage{graphicx}
\usepackage{hyperref}
\usepackage{xcolor}
\usepackage{tcolorbox}
\usepackage{listings}
\usepackage{siunitx}
\usepackage{multicol}

% カラーボックス設定
\newtcolorbox{keypoint}{colback=blue!5!white,colframe=blue!75!black,title=Key Point}
\newtcolorbox{warning}{colback=red!5!white,colframe=red!75!black,title=注意}
\newtcolorbox{definition}{colback=green!5!white,colframe=green!50!black,title=定義}
\newtcolorbox{diagnostic}{colback=orange!5!white,colframe=orange!75!black,title=診断指標}

\lstset{
    basicstyle=\footnotesize\ttfamily,
    keywordstyle=\color{blue},
    commentstyle=\color{gray},
    breaklines=true,
    frame=single
}

\title{\vspace{-2cm}
\textbf{diagnose\_bayesian\_convergence.py 理解ドキュメント}\\
\large MCMCサンプリングの収束診断ツール
}
\author{物理工学専攻 修士論文解析用教材}
\date{2026年1月版}

\begin{document}
\maketitle

\tableofcontents
\clearpage

%%%%%%%%%%%%%%%%%%%%%%%%%%%%%%%%%%%%%%%%%%%%%%%%%%%%%%%%%%%%%%%%%
\section{はじめに:このプログラムの目的}
%%%%%%%%%%%%%%%%%%%%%%%%%%%%%%%%%%%%%%%%%%%%%%%%%%%%%%%%%%%%%%%%%

\begin{keypoint}
このプログラムは、ベイズ推定(MCMC)の結果が\textbf{統計的に信頼できるか}を
診断するためのツールです。MCMCが正しく収束していなければ、
推定されたパラメータや不確かさは意味を持ちません。
\end{keypoint}

\subsection{なぜ収束診断が必要か?}

MCMCは「十分に長く実行すれば」事後分布に収束しますが、
実際には以下の問題が起こりえます:

\begin{enumerate}
    \item \textbf{未収束}:サンプル数が不足し、真の事後分布を探索できていない
    \item \textbf{局所解への固着}:パラメータ空間の一部にのみ滞在
    \item \textbf{高い自己相関}:連続サンプルが独立でなく、実質的なサンプル数が少ない
    \item \textbf{チェーン間の不一致}:異なるチェーンが異なる領域を探索
\end{enumerate}

\subsection{使用方法}

\begin{lstlisting}[language=bash]
python diagnose_bayesian_convergence.py <結果ディレクトリ>

# 例:
python diagnose_bayesian_convergence.py bayesian_results_v6_informed_HB_20260108_013540
\end{lstlisting}

%%%%%%%%%%%%%%%%%%%%%%%%%%%%%%%%%%%%%%%%%%%%%%%%%%%%%%%%%%%%%%%%%
\section{収束診断の指標}
%%%%%%%%%%%%%%%%%%%%%%%%%%%%%%%%%%%%%%%%%%%%%%%%%%%%%%%%%%%%%%%%%

\subsection{$\hat{R}$統計量(Gelman-Rubin診断)}

\begin{definition}
\textbf{$\hat{R}$(R-hat)}は、複数の独立したマルコフ連鎖(チェーン)を比較することで
収束を判定する指標です。
\end{definition}

\subsubsection{計算方法}

$M$個のチェーンがあり、各チェーンに$N$個のサンプルがあるとします。

\begin{enumerate}
    \item \textbf{チェーン内分散} $W$(Within-chain variance):
    \begin{equation}
    W = \frac{1}{M} \sum_{m=1}^{M} s_m^2, \quad s_m^2 = \frac{1}{N-1} \sum_{n=1}^{N} (\theta_{mn} - \bar{\theta}_m)^2
    \end{equation}
    
    \item \textbf{チェーン間分散} $B$(Between-chain variance):
    \begin{equation}
    B = \frac{N}{M-1} \sum_{m=1}^{M} (\bar{\theta}_m - \bar{\theta})^2
    \end{equation}
    
    \item \textbf{推定分散}:
    \begin{equation}
    \hat{\text{Var}}(\theta) = \frac{N-1}{N} W + \frac{1}{N} B
    \end{equation}
    
    \item \textbf{$\hat{R}$統計量}:
    \begin{equation}
    \boxed{\hat{R} = \sqrt{\frac{\hat{\text{Var}}(\theta)}{W}}}
    \end{equation}
\end{enumerate}

\subsubsection{解釈}

\begin{table}[h]
\centering
\caption{$\hat{R}$の解釈基準}
\begin{tabular}{cll}
\toprule
\textbf{$\hat{R}$の値} & \textbf{判定} & \textbf{アクション} \\
\midrule
$< 1.01$ & ✅ 優秀 & そのまま使用可能 \\
$1.01 - 1.05$ & ⚠️ 許容範囲 & 注意しながら使用 \\
$1.05 - 1.10$ & ⚠️ 要注意 & サンプル数増加を検討 \\
$> 1.10$ & ❌ 未収束 & 必ずサンプル数増加 \\
\bottomrule
\end{tabular}
\end{table}

\begin{figure}[h]
\centering
\begin{tikzpicture}[scale=0.8]
% 良好な収束
\begin{scope}
    \draw[->] (0,0) -- (6,0) node[right] {\small Iteration};
    \draw[->] (0,0) -- (0,3.5) node[above] {\small $\theta$};
    
    \foreach \c/\col in {1/red, 2/blue, 3/green!60!black, 4/orange} {
        \draw[\col, opacity=0.7] 
            plot[domain=0:5.5, samples=40, smooth] 
            (\x, {1.8 + 0.3*rand});
    }
    
    \node[below] at (3,-0.5) {\textbf{$\hat{R} \approx 1.0$}};
    \node[below] at (3,-1) {\small 全チェーンが同じ領域};
\end{scope}

% 未収束
\begin{scope}[xshift=9cm]
    \draw[->] (0,0) -- (6,0) node[right] {\small Iteration};
    \draw[->] (0,0) -- (0,3.5) node[above] {\small $\theta$};
    
    \draw[red, opacity=0.7] plot[domain=0:5.5, samples=30] 
        (\x, {0.8 + 0.15*\x + 0.15*rand});
    \draw[blue, opacity=0.7] plot[domain=0:5.5, samples=30] 
        (\x, {2.8 - 0.1*\x + 0.15*rand});
    \draw[green!60!black, opacity=0.7] plot[domain=0:5.5, samples=30] 
        (\x, {1.5 + 0.2*rand});
    \draw[orange, opacity=0.7] plot[domain=0:5.5, samples=30] 
        (\x, {2.0 + 0.08*\x + 0.2*rand});
    
    \node[below] at (3,-0.5) {\textbf{$\hat{R} > 1.1$}};
    \node[below] at (3,-1) {\small チェーンが分離};
\end{scope}
\end{tikzpicture}
\caption{$\hat{R}$による収束判定の可視化}
\label{fig:rhat}
\end{figure}

\subsection{有効サンプルサイズ(ESS)}

\begin{definition}
\textbf{ESS(Effective Sample Size)}は、自己相関を考慮した
「実質的に独立な」サンプル数です。
\end{definition}

MCMCサンプルは時系列データであり、連続したサンプルは相関を持ちます。
$N$個のサンプルがあっても、独立なサンプル数は$N$より少なくなります。

\subsubsection{計算方法}

自己相関関数$\rho_k$(ラグ$k$の相関)を用いて:

\begin{equation}
\boxed{\text{ESS} = \frac{N}{1 + 2\sum_{k=1}^{K} \rho_k}}
\end{equation}

ここで$K$は自己相関が無視できるようになるラグ。

\subsubsection{ESSの種類}

\begin{itemize}
    \item \textbf{ESS\_bulk}:分布の中心(平均、中央値)の推定精度
    \item \textbf{ESS\_tail}:分布の裾(信頼区間の端)の推定精度
\end{itemize}

\begin{table}[h]
\centering
\caption{ESSの解釈基準}
\begin{tabular}{cll}
\toprule
\textbf{ESS} & \textbf{判定} & \textbf{意味} \\
\midrule
$\geq 400$ & ✅ 十分 & 信頼できる推定 \\
$100 - 400$ & ⚠️ 不足気味 & 精度向上の余地あり \\
$< 100$ & ❌ 危険 & 推定が不安定 \\
\bottomrule
\end{tabular}
\end{table}

\begin{keypoint}
ESS $\geq 400$ の目安は、事後平均の標準誤差(MCSE)を事後標準偏差の5\%以下に
抑えるために必要なサンプル数に基づいています:
\[
\text{MCSE} = \frac{\text{SD}}{\sqrt{\text{ESS}}} \leq 0.05 \times \text{SD} \quad \Rightarrow \quad \text{ESS} \geq 400
\]
\end{keypoint}

\subsection{自己相関}

\begin{definition}
\textbf{自己相関}$\rho_k$は、$k$ステップ離れたサンプル間の相関係数:
\begin{equation}
\rho_k = \frac{\text{Cov}(\theta_t, \theta_{t+k})}{\text{Var}(\theta_t)}
\end{equation}
\end{definition}

理想的には、自己相関は小さいラグ$k$で急速に0に減衰すべきです。

\begin{figure}[h]
\centering
\begin{tikzpicture}[scale=0.8]
% 良好な自己相関
\begin{scope}
    \draw[->] (0,0) -- (5,0) node[right] {\small Lag $k$};
    \draw[->] (0,0) -- (0,3) node[above] {\small $\rho_k$};
    
    \draw[thick, blue] plot[domain=0:4.5, samples=20] 
        (\x, {2.5*exp(-\x)});
    
    \draw[dashed, gray] (0,0.2) -- (5,0.2);
    
    \node[below] at (2.5,-0.5) {\textbf{良好}};
    \node[below] at (2.5,-1) {\small 急速に減衰 → 高ESS};
\end{scope}

% 問題のある自己相関
\begin{scope}[xshift=8cm]
    \draw[->] (0,0) -- (5,0) node[right] {\small Lag $k$};
    \draw[->] (0,0) -- (0,3) node[above] {\small $\rho_k$};
    
    \draw[thick, red] plot[domain=0:4.5, samples=20] 
        (\x, {2.5*exp(-0.2*\x)});
    
    \draw[dashed, gray] (0,0.2) -- (5,0.2);
    
    \node[below] at (2.5,-0.5) {\textbf{問題あり}};
    \node[below] at (2.5,-1) {\small 緩やかな減衰 → 低ESS};
\end{scope}
\end{tikzpicture}
\caption{自己相関プロットの解釈}
\label{fig:autocorr}
\end{figure}

%%%%%%%%%%%%%%%%%%%%%%%%%%%%%%%%%%%%%%%%%%%%%%%%%%%%%%%%%%%%%%%%%
\section{プログラムの機能}
%%%%%%%%%%%%%%%%%%%%%%%%%%%%%%%%%%%%%%%%%%%%%%%%%%%%%%%%%%%%%%%%%

\subsection{生成される診断プロット}

\begin{table}[h]
\centering
\caption{出力ファイル一覧}
\begin{tabular}{ll}
\toprule
\textbf{ファイル名} & \textbf{内容} \\
\midrule
\texttt{detailed\_trace\_H.png} & H形式:事後分布ヒストグラム + トレースプロット \\
\texttt{detailed\_trace\_B.png} & B形式:事後分布ヒストグラム + トレースプロット \\
\texttt{ess\_comparison.png} & H/B形式のESS比較バープロット \\
\texttt{rhat\_comparison.png} & H/B形式の$\hat{R}$比較バープロット \\
\texttt{correlation\_H.png} & H形式:パラメータ間相関ペアプロット \\
\texttt{correlation\_B.png} & B形式:パラメータ間相関ペアプロット \\
\texttt{rank\_plot\_H.png} & H形式:ランクプロット \\
\texttt{rank\_plot\_B.png} & B形式:ランクプロット \\
\texttt{autocorr\_H.png} & H形式:自己相関プロット \\
\texttt{autocorr\_B.png} & B形式:自己相関プロット \\
\texttt{convergence\_report.md} & 収束診断レポート(Markdown) \\
\bottomrule
\end{tabular}
\end{table}

\subsection{詳細トレースプロット}

各パラメータについて2つのパネルを生成:

\begin{enumerate}
    \item \textbf{左パネル}:事後分布ヒストグラム
    \begin{itemize}
        \item 青ヒストグラム:サンプル分布
        \item 赤破線:事後平均
        \item オレンジ点線:2.5\%, 97.5\%パーセンタイル(95\%信頼区間)
    \end{itemize}
    
    \item \textbf{右パネル}:トレースプロット
    \begin{itemize}
        \item 8本のチェーンを色分け表示
        \item ESS と $\hat{R}$ をテキスト表示
        \item ✅/⚠️ マークで収束状態を示す
    \end{itemize}
\end{enumerate}

\subsection{パラメータ間相関プロット}

\begin{definition}
\textbf{ペアプロット}は、パラメータの全ペアについて2次元事後分布を可視化します。
強い相関があると、MCMCの探索効率が低下します。
\end{definition}

\begin{figure}[h]
\centering
\begin{tikzpicture}[scale=0.7]
% 低相関
\begin{scope}
    \draw[->] (0,0) -- (4,0) node[right] {$\theta_1$};
    \draw[->] (0,0) -- (0,4) node[above] {$\theta_2$};
    
    \filldraw[blue!30, draw=blue] (2,2) circle (1cm and 0.9cm);
    
    \node[below] at (2,-0.5) {\textbf{低相関}};
    \node[below] at (2,-1) {\small 円形 → 探索容易};
\end{scope}

% 高相関
\begin{scope}[xshift=7cm]
    \draw[->] (0,0) -- (4,0) node[right] {$\theta_1$};
    \draw[->] (0,0) -- (0,4) node[above] {$\theta_2$};
    
    \filldraw[red!30, draw=red, rotate around={45:(2,2)}] (2,2) ellipse (1.5cm and 0.3cm);
    
    \node[below] at (2,-0.5) {\textbf{高相関}};
    \node[below] at (2,-1) {\small 細長い楕円 → 探索困難};
\end{scope}
\end{tikzpicture}
\caption{パラメータ間相関と事後分布の形状}
\label{fig:correlation}
\end{figure}

\subsection{ランクプロット}

\begin{definition}
\textbf{ランクプロット}は、各チェーンのサンプルをプールし、
全体でのランク(順位)を計算して、チェーンごとにヒストグラム化します。
\end{definition}

収束している場合、全チェーンのランクヒストグラムは一様分布に近くなります。
特定のチェーンだけが高い/低いランクに偏っていれば、未収束の兆候です。

%%%%%%%%%%%%%%%%%%%%%%%%%%%%%%%%%%%%%%%%%%%%%%%%%%%%%%%%%%%%%%%%%
\section{プログラムフロー}
%%%%%%%%%%%%%%%%%%%%%%%%%%%%%%%%%%%%%%%%%%%%%%%%%%%%%%%%%%%%%%%%%

\begin{figure}[h]
\centering
\begin{tikzpicture}[
    node distance=0.8cm,
    startstop/.style={rectangle, rounded corners, minimum width=3cm, minimum height=0.7cm, 
                      text centered, draw=black, fill=red!30},
    process/.style={rectangle, minimum width=3.5cm, minimum height=0.7cm, 
                    text centered, draw=black, fill=orange!30},
    io/.style={trapezium, trapezium left angle=70, trapezium right angle=110, 
               minimum width=2.5cm, minimum height=0.7cm, text centered, draw=black, fill=blue!30},
    decision/.style={diamond, minimum width=2cm, minimum height=1cm, 
                     text centered, draw=black, fill=green!30},
    arrow/.style={thick,->,>=stealth}
]

\node (start) [startstop] {開始};
\node (load) [io, below=of start] {トレース読み込み (H/B)};
\node (summary) [process, below=of load] {統計サマリー計算};
\node (check) [decision, below=of summary, aspect=2] {$\hat{R} > 1.01$?};
\node (warn) [process, right=2cm of check, fill=red!20] {警告出力};
\node (ess) [decision, below=of check, aspect=2] {ESS $< 400$?};
\node (warn2) [process, right=2cm of ess, fill=red!20] {警告出力};
\node (plot1) [process, below=of ess] {詳細トレースプロット};
\node (plot2) [process, below=of plot1] {相関プロット};
\node (plot3) [process, below=of plot2] {ESS/$\hat{R}$比較プロット};
\node (report) [io, below=of plot3] {レポート生成};
\node (end) [startstop, below=of report] {終了};

\draw [arrow] (start) -- (load);
\draw [arrow] (load) -- (summary);
\draw [arrow] (summary) -- (check);
\draw [arrow] (check) -- node[above] {Yes} (warn);
\draw [arrow] (check) -- node[right] {No} (ess);
\draw [arrow] (warn) |- (ess);
\draw [arrow] (ess) -- node[above] {Yes} (warn2);
\draw [arrow] (ess) -- node[right] {No} (plot1);
\draw [arrow] (warn2) |- (plot1);
\draw [arrow] (plot1) -- (plot2);
\draw [arrow] (plot2) -- (plot3);
\draw [arrow] (plot3) -- (report);
\draw [arrow] (report) -- (end);

\end{tikzpicture}
\caption{プログラムフローチャート}
\label{fig:flow}
\end{figure}

%%%%%%%%%%%%%%%%%%%%%%%%%%%%%%%%%%%%%%%%%%%%%%%%%%%%%%%%%%%%%%%%%
\section{収束診断レポートの読み方}
%%%%%%%%%%%%%%%%%%%%%%%%%%%%%%%%%%%%%%%%%%%%%%%%%%%%%%%%%%%%%%%%%

\subsection{レポートの構成}

\texttt{convergence\_report.md}には以下が含まれます:

\begin{enumerate}
    \item \textbf{H形式/B形式ごとの診断結果}
    \begin{itemize}
        \item $\hat{R} > 1.01$ のパラメータ一覧
        \item ESS $< 400$ のパラメータ一覧
        \item 全パラメータの統計サマリー
    \end{itemize}
    
    \item \textbf{推奨事項}
    \begin{itemize}
        \item 問題がある場合の対処法
        \item サンプリング設定の調整案
    \end{itemize}
\end{enumerate}

\subsection{問題がある場合の対処法}

\begin{table}[h]
\centering
\caption{収束問題と対処法}
\begin{tabular}{lp{8cm}}
\toprule
\textbf{問題} & \textbf{対処法} \\
\midrule
$\hat{R} > 1.1$ & 
\begin{itemize}
    \item サンプル数(draws)を2〜3倍に増加
    \item ウォームアップ(tune)を増加
    \item チェーン数を増やして確認
\end{itemize} \\
\midrule
ESS $< 100$ & 
\begin{itemize}
    \item サンプル数を大幅に増加
    \item 事前分布の$\sigma$を拡大
    \item 再パラメータ化を検討
\end{itemize} \\
\midrule
高い相関 & 
\begin{itemize}
    \item 相関するパラメータ間で再パラメータ化
    \item 事前分布を緩和
    \item より効率的なサンプラーを検討
\end{itemize} \\
\bottomrule
\end{tabular}
\end{table}

%%%%%%%%%%%%%%%%%%%%%%%%%%%%%%%%%%%%%%%%%%%%%%%%%%%%%%%%%%%%%%%%%
\section{主要関数の説明}
%%%%%%%%%%%%%%%%%%%%%%%%%%%%%%%%%%%%%%%%%%%%%%%%%%%%%%%%%%%%%%%%%

\subsection{diagnose\_convergence}

\begin{lstlisting}[language=Python]
def diagnose_convergence(trace, model_name, output_dir):
    """
    収束診断の詳細分析
    
    Parameters
    ----------
    trace : az.InferenceData
        ArviZトレースオブジェクト
    model_name : str
        'H' または 'B'
    output_dir : pathlib.Path
        出力ディレクトリ
    
    Returns
    -------
    summary : pd.DataFrame
        パラメータ統計サマリー
    
    出力
    ----
    - R-hat > 1.01 のパラメータを警告
    - ESS < 400 のパラメータを警告
    - ESS < 100 のパラメータを危険として報告
    """
\end{lstlisting}

\subsection{plot\_detailed\_trace}

\begin{lstlisting}[language=Python]
def plot_detailed_trace(trace, model_name, output_dir):
    """
    詳細なトレースプロット生成
    
    各パラメータについて:
    - 左: 事後分布ヒストグラム(平均、95%CI付き)
    - 右: 8チェーンのトレースプロット(ESS/R-hat表示)
    
    出力: detailed_trace_{model_name}.png
    """
\end{lstlisting}

\subsection{generate\_convergence\_report}

\begin{lstlisting}[language=Python]
def generate_convergence_report(traces, output_dir):
    """
    収束診断レポート生成(Markdown形式)
    
    内容:
    - H/B形式ごとの問題パラメータ
    - 全パラメータのサマリー統計
    - 問題がある場合の推奨事項
    
    出力: convergence_report.md
    """
\end{lstlisting}

%%%%%%%%%%%%%%%%%%%%%%%%%%%%%%%%%%%%%%%%%%%%%%%%%%%%%%%%%%%%%%%%%
\section{実践的な使用例}
%%%%%%%%%%%%%%%%%%%%%%%%%%%%%%%%%%%%%%%%%%%%%%%%%%%%%%%%%%%%%%%%%

\subsection{典型的なワークフロー}

\begin{enumerate}
    \item ベイズ推定を実行
    \begin{lstlisting}[language=bash]
python bayesian_v6_informed_priors_HB.py
    \end{lstlisting}
    
    \item 収束診断を実行
    \begin{lstlisting}[language=bash]
python diagnose_bayesian_convergence.py bayesian_results_v6_...
    \end{lstlisting}
    
    \item レポートを確認
    \begin{lstlisting}[language=bash]
cat bayesian_results_v6_.../diagnostics/convergence_report.md
    \end{lstlisting}
    
    \item 問題があれば設定を調整して再実行
\end{enumerate}

\subsection{出力例の解釈}

\begin{lstlisting}
======================================================================
📊 H形式 収束診断
======================================================================

【R-hat診断】
✅ 全パラメータでR-hat < 1.01

【ESS診断】
⚠️ ESS_bulk < 400 のパラメータ:
   - B4: ESS_bulk = 287, ESS_tail = 312
   - B6: ESS_bulk = 203, ESS_tail = 245
\end{lstlisting}

解釈:
\begin{itemize}
    \item $\hat{R}$は問題なし(全チェーン収束済み)
    \item $B_4$, $B_6$のESSが低い $\to$ サンプル数増加推奨
\end{itemize}

%%%%%%%%%%%%%%%%%%%%%%%%%%%%%%%%%%%%%%%%%%%%%%%%%%%%%%%%%%%%%%%%%
\section{まとめ}
%%%%%%%%%%%%%%%%%%%%%%%%%%%%%%%%%%%%%%%%%%%%%%%%%%%%%%%%%%%%%%%%%

\begin{enumerate}
    \item \textbf{目的}:MCMCサンプリングの信頼性を定量的に評価
    
    \item \textbf{主要指標}:
    \begin{itemize}
        \item $\hat{R}$:チェーン間の一致度($< 1.01$が理想)
        \item ESS:実質的な独立サンプル数($\geq 400$が推奨)
        \item 自己相関:連続サンプル間の依存性
    \end{itemize}
    
    \item \textbf{出力}:
    \begin{itemize}
        \item 詳細トレースプロット
        \item パラメータ間相関プロット
        \item ESS/$\hat{R}$比較バープロット
        \item Markdownレポート
    \end{itemize}
    
    \item \textbf{活用}:
    \begin{itemize}
        \item 問題パラメータの特定
        \item サンプリング設定の改善方針決定
        \item 推定結果の信頼性評価
    \end{itemize}
\end{enumerate}

\begin{keypoint}
MCMCの結果を報告する際は、必ず収束診断の結果($\hat{R}$、ESS)を添えましょう。
収束していないMCMCの結果は、統計的に無意味です。
\end{keypoint}

\end{document}
