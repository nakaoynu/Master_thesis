% filepath: test_hdi_plot_documentation.tex
\documentclass[a4paper,12pt,dvipdfmx]{jlreq}
\usepackage{amsmath,amssymb,amsthm}
\usepackage{bm}
\usepackage{physics}
\usepackage{tikz}
\usetikzlibrary{shapes.geometric, arrows, positioning, fit, backgrounds, calc, patterns}
\usepackage{algorithm2e}
\usepackage{booktabs}
\usepackage{graphicx}
\usepackage{hyperref}
\usepackage{xcolor}
\usepackage{tcolorbox}
\usepackage{listings}
\usepackage{siunitx}
\usepackage{multicol}

% カラーボックス設定
\newtcolorbox{keypoint}{colback=blue!5!white,colframe=blue!75!black,title=Key Point}
\newtcolorbox{warning}{colback=red!5!white,colframe=red!75!black,title=注意}
\newtcolorbox{definition}{colback=green!5!white,colframe=green!50!black,title=定義}
\newtcolorbox{diagnostic}{colback=orange!5!white,colframe=orange!75!black,title=診断項目}

\lstset{
    basicstyle=\footnotesize\ttfamily,
    keywordstyle=\color{blue},
    commentstyle=\color{gray},
    breaklines=true,
    frame=single
}

\title{\vspace{-2cm}
\textbf{test\_hdi\_plot.py 理解ドキュメント}\\
\large HDI信頼区間プロットの診断ツール
}
\author{物理工学専攻 修士論文解析用教材}
\date{2026年1月版}

\begin{document}
\maketitle

\tableofcontents
\clearpage

%%%%%%%%%%%%%%%%%%%%%%%%%%%%%%%%%%%%%%%%%%%%%%%%%%%%%%%%%%%%%%%%%
\section{はじめに:このプログラムの目的}
%%%%%%%%%%%%%%%%%%%%%%%%%%%%%%%%%%%%%%%%%%%%%%%%%%%%%%%%%%%%%%%%%

\begin{keypoint}
このプログラムは、ベイズ推定結果から計算される\textbf{HDI(Highest Density Interval)信頼区間}が
正しく計算・描画されているかを診断するためのツールです。
透過スペクトルの不確かさを可視化する際の問題を特定します。
\end{keypoint}

\subsection{背景:なぜこのツールが必要か?}

ベイズ推定の重要な利点は、パラメータだけでなく\textbf{予測の不確かさ}も定量化できることです。
しかし、以下の問題が発生することがあります:

\begin{enumerate}
    \item HDI区間の幅が極端に狭く、可視化できない
    \item 事後分布の分散が小さすぎて、信頼区間が意味をなさない
    \item プロットの線が重なって境界線が見えない
\end{enumerate}

\subsection{使用方法}

\begin{lstlisting}[language=bash]
python test_hdi_plot.py <結果ディレクトリ>

# 例:
python test_hdi_plot.py bayesian_results_v6_informed_HB_20260108_013540
\end{lstlisting}

%%%%%%%%%%%%%%%%%%%%%%%%%%%%%%%%%%%%%%%%%%%%%%%%%%%%%%%%%%%%%%%%%
\section{HDI(Highest Density Interval)とは}
%%%%%%%%%%%%%%%%%%%%%%%%%%%%%%%%%%%%%%%%%%%%%%%%%%%%%%%%%%%%%%%%%

\subsection{定義}

\begin{definition}
\textbf{HDI(Highest Density Interval)}は、事後分布において確率密度が最も高い領域を
囲む信頼区間です。95\% HDIは、事後分布の95\%の確率質量を含む最小幅の区間です。
\end{definition}

HDIは従来の\textbf{パーセンタイル区間}(2.5\%〜97.5\%)と類似していますが、
非対称な分布では異なる結果を与えます。

\begin{figure}[h]
\centering
\begin{tikzpicture}[scale=0.85]
% 対称分布
\begin{scope}
    \draw[->] (0,0) -- (6,0) node[right] {$\theta$};
    \draw[->] (0,0) -- (0,3.5) node[above] {$P(\theta)$};
    
    % 正規分布
    \draw[thick, blue, domain=0.5:5.5, samples=50] 
        plot (\x, {3*exp(-(\x-3)*(\x-3)/0.8)});
    
    % HDI区間
    \fill[blue!20, domain=1.3:4.7, samples=50] 
        (1.3,0) -- plot (\x, {3*exp(-(\x-3)*(\x-3)/0.8)}) -- (4.7,0) -- cycle;
    
    \draw[<->, thick, red] (1.3,0.2) -- (4.7,0.2);
    \node[below, red] at (3,0) {\small 95\% HDI};
    
    \node[below] at (3,-0.7) {\textbf{対称分布}};
    \node[below] at (3,-1.2) {\small HDI = パーセンタイル区間};
\end{scope}

% 非対称分布
\begin{scope}[xshift=8cm]
    \draw[->] (0,0) -- (6,0) node[right] {$\theta$};
    \draw[->] (0,0) -- (0,3.5) node[above] {$P(\theta)$};
    
    % 歪んだ分布
    \draw[thick, blue, domain=0.5:5.5, samples=50] 
        plot (\x, {2.5*(\x-0.3)*exp(-0.7*(\x-0.3))});
    
    % HDI区間
    \fill[blue!20, domain=0.8:3.8, samples=50] 
        (0.8,0) -- plot (\x, {2.5*(\x-0.3)*exp(-0.7*(\x-0.3))}) -- (3.8,0) -- cycle;
    
    \draw[<->, thick, red] (0.8,0.2) -- (3.8,0.2);
    \node[below, red] at (2.3,0) {\small 95\% HDI};
    
    \node[below] at (3,-0.7) {\textbf{非対称分布}};
    \node[below] at (3,-1.2) {\small HDI $\neq$ パーセンタイル区間};
\end{scope}
\end{tikzpicture}
\caption{対称分布と非対称分布におけるHDI}
\label{fig:hdi}
\end{figure}

\subsection{透過スペクトルへの適用}

各周波数点$\omega_i$において、事後サンプルから透過率$T(\omega_i)$を計算し、
その分布の95\% HDIを求めます:

\begin{equation}
\boxed{
\text{95\% HDI at } \omega_i = \left[ T_{2.5\%}(\omega_i), \; T_{97.5\%}(\omega_i) \right]
}
\end{equation}

ここで$T_{p\%}$は$p$パーセンタイル値。

\subsection{HDI帯の可視化}

\begin{figure}[h]
\centering
\begin{tikzpicture}[scale=0.9]
    \draw[->] (0,0) -- (10,0) node[right] {Frequency [THz]};
    \draw[->] (0,0) -- (0,5) node[above] {Transmittance};
    
    % HDI帯
    \fill[red!20] 
        plot[domain=0.5:9.5, samples=50] (\x, {3.5 + 0.8*sin(60*\x) + 0.3})
        -- plot[domain=9.5:0.5, samples=50] (\x, {3.5 + 0.8*sin(60*\x) - 0.3})
        -- cycle;
    
    % 下限境界線
    \draw[dashed, red, thick] 
        plot[domain=0.5:9.5, samples=50] (\x, {3.5 + 0.8*sin(60*\x) - 0.3});
    
    % 上限境界線
    \draw[dashed, red, thick] 
        plot[domain=0.5:9.5, samples=50] (\x, {3.5 + 0.8*sin(60*\x) + 0.3});
    
    % 平均線
    \draw[thick, darkgray] 
        plot[domain=0.5:9.5, samples=50] (\x, {3.5 + 0.8*sin(60*\x)});
    
    % データ点
    \foreach \x in {1,2,...,9} {
        \fill[blue] (\x, {3.5 + 0.8*sin(60*\x) + 0.15*rand}) circle (2pt);
    }
    
    % 凡例
    \node[right] at (6,4.8) {\small \textcolor{blue}{●} データ};
    \node[right] at (6,4.4) {\small ─ 事後平均};
    \node[right] at (6,4.0) {\small \textcolor{red}{- -} 95\% HDI境界};
    \node[right] at (6,3.6) {\small \colorbox{red!20}{ } HDI帯};
\end{tikzpicture}
\caption{透過スペクトルのHDI可視化}
\label{fig:hdi_spectrum}
\end{figure}

%%%%%%%%%%%%%%%%%%%%%%%%%%%%%%%%%%%%%%%%%%%%%%%%%%%%%%%%%%%%%%%%%
\section{診断項目}
%%%%%%%%%%%%%%%%%%%%%%%%%%%%%%%%%%%%%%%%%%%%%%%%%%%%%%%%%%%%%%%%%

\subsection{診断される項目}

\begin{table}[h]
\centering
\caption{HDI診断項目}
\begin{tabular}{lll}
\toprule
\textbf{項目} & \textbf{健全な値} & \textbf{問題となる値} \\
\midrule
HDI幅の平均 & $> 0.01$ & $< 0.001$ \\
HDI幅の最大 & $> 0.05$ & $< 0.01$ \\
標準偏差の平均 & $> 0.005$ & $< 0.001$ \\
\bottomrule
\end{tabular}
\end{table}

\subsection{問題の原因と対処}

\begin{diagnostic}
\textbf{HDI幅が極端に狭い場合}(平均 $< 0.001$)

原因:
\begin{itemize}
    \item 事前分布が狭すぎる($\sigma$が小さい)
    \item データの情報量が強すぎる(過度に制約される)
    \item サンプル数が少なく、分散が過小評価されている
\end{itemize}

対処法:
\begin{itemize}
    \item 事前分布の$\sigma$を3〜5倍に拡大
    \item より弱い情報的事前分布を使用
    \item サンプル数を増やす(100 $\to$ 500〜1000)
\end{itemize}
\end{diagnostic}

\begin{diagnostic}
\textbf{HDI境界線が見えない場合}

原因:
\begin{itemize}
    \item HDI幅が小さく、平均線と重なっている
    \item 線の太さ(linewidth)が細すぎる
    \item 色のコントラストが不足
\end{itemize}

対処法:
\begin{itemize}
    \item linewidthを2.0 $\to$ 3.0に増加
    \item 破線スタイルを明確化
    \item 色を濃くする
\end{itemize}
\end{diagnostic}

%%%%%%%%%%%%%%%%%%%%%%%%%%%%%%%%%%%%%%%%%%%%%%%%%%%%%%%%%%%%%%%%%
\section{プログラムの処理フロー}
%%%%%%%%%%%%%%%%%%%%%%%%%%%%%%%%%%%%%%%%%%%%%%%%%%%%%%%%%%%%%%%%%

\begin{figure}[h]
\centering
\begin{tikzpicture}[
    node distance=0.9cm,
    startstop/.style={rectangle, rounded corners, minimum width=3cm, minimum height=0.7cm, 
                      text centered, draw=black, fill=red!30},
    process/.style={rectangle, minimum width=4cm, minimum height=0.7cm, 
                    text centered, draw=black, fill=orange!30},
    io/.style={trapezium, trapezium left angle=70, trapezium right angle=110, 
               minimum width=2.5cm, minimum height=0.7cm, text centered, draw=black, fill=blue!30},
    arrow/.style={thick,->,>=stealth}
]

\node (start) [startstop] {開始};
\node (load) [io, below=of start] {トレース・データ読み込み};
\node (sample) [process, below=of load] {事後サンプル取得 (n=100)};
\node (param) [process, below=of sample] {パラメータ統計表示};
\node (calc) [process, below=of param] {各サンプルで透過スペクトル計算};
\node (stat) [process, below=of calc] {統計量計算 (平均, SD, HDI)};
\node (diag) [process, below=of stat] {診断判定};
\node (plot) [io, below=of diag] {テストプロット生成};
\node (end) [startstop, below=of plot] {終了};

\draw [arrow] (start) -- (load);
\draw [arrow] (load) -- (sample);
\draw [arrow] (sample) -- (param);
\draw [arrow] (param) -- (calc);
\draw [arrow] (calc) -- (stat);
\draw [arrow] (stat) -- (diag);
\draw [arrow] (diag) -- (plot);
\draw [arrow] (plot) -- (end);

% 注釈
\node[right=0.5cm of calc, align=left] {\small ループ処理\\(サンプル×周波数点)};

\end{tikzpicture}
\caption{test\_hdi\_plot.pyの処理フロー}
\label{fig:flow}
\end{figure}

%%%%%%%%%%%%%%%%%%%%%%%%%%%%%%%%%%%%%%%%%%%%%%%%%%%%%%%%%%%%%%%%%
\section{透過スペクトル計算}
%%%%%%%%%%%%%%%%%%%%%%%%%%%%%%%%%%%%%%%%%%%%%%%%%%%%%%%%%%%%%%%%%

\subsection{事後サンプルからの計算}

各事後サンプル$s = 1, 2, \ldots, N_{\text{samples}}$について:

\begin{algorithm}[H]
\SetAlgoLined
\KwIn{事後サンプル $\{\bm{\theta}^{(s)}\}$、周波数配列 $\{\omega_i\}$}
\KwOut{透過スペクトルサンプル $\{T^{(s)}(\omega_i)\}$}

\For{$s = 1$ \KwTo $N_{\text{samples}}$}{
    パラメータ抽出: $g^{(s)}, a^{(s)}, B_4^{(s)}, B_6^{(s)}, \varepsilon^{(s)}, \gamma^{(s)}$\;
    
    ハミルトニアン構築: $\hat{H}^{(s)} = \hat{H}_{\text{CF}}(B_4^{(s)}, B_6^{(s)}) + \hat{H}_{\text{Zee}}(g^{(s)}, B)$\;
    
    感受率計算: $\chi^{(s)}(\omega) = \text{calculate\_susceptibility}(\omega, \hat{H}^{(s)}, T, \gamma^{(s)})$\;
    
    透過率計算: $T^{(s)}(\omega) = \text{calculate\_transmission}(\omega, \chi^{(s)}, \varepsilon^{(s)})$\;
}

\Return{$\{T^{(s)}(\omega_i)\}_{s,i}$}
\caption{透過スペクトルの事後サンプル計算}
\end{algorithm}

\subsection{統計量の計算}

各周波数点$\omega_i$において:

\begin{align}
\text{事後平均}: \quad & \bar{T}(\omega_i) = \frac{1}{N} \sum_{s=1}^{N} T^{(s)}(\omega_i) \\
\text{標準偏差}: \quad & \text{SD}(\omega_i) = \sqrt{\frac{1}{N-1} \sum_{s=1}^{N} (T^{(s)}(\omega_i) - \bar{T}(\omega_i))^2} \\
\text{HDI下限}: \quad & T_{2.5\%}(\omega_i) = \text{percentile}(\{T^{(s)}(\omega_i)\}, 2.5) \\
\text{HDI上限}: \quad & T_{97.5\%}(\omega_i) = \text{percentile}(\{T^{(s)}(\omega_i)\}, 97.5) \\
\text{HDI幅}: \quad & \Delta T(\omega_i) = T_{97.5\%}(\omega_i) - T_{2.5\%}(\omega_i)
\end{align}

%%%%%%%%%%%%%%%%%%%%%%%%%%%%%%%%%%%%%%%%%%%%%%%%%%%%%%%%%%%%%%%%%
\section{生成されるテストプロット}
%%%%%%%%%%%%%%%%%%%%%%%%%%%%%%%%%%%%%%%%%%%%%%%%%%%%%%%%%%%%%%%%%

出力ファイル: \texttt{plots/hdi\_test\_diagnostic.png}

\subsection{4パネル構成}

\begin{enumerate}
    \item \textbf{全体図}(左上)
    \begin{itemize}
        \item データ点(灰色○)
        \item 95\% HDI帯(薄赤の塗りつぶし)
        \item HDI下限・上限(赤破線)
        \item 事後平均(濃赤実線)
    \end{itemize}
    
    \item \textbf{HDI幅の周波数依存性}(右上)
    \begin{itemize}
        \item $\Delta T(\omega)$ vs $\omega$
        \item 平均HDI幅を赤破線で表示
    \end{itemize}
    
    \item \textbf{標準偏差の周波数依存性}(左下)
    \begin{itemize}
        \item SD$(\omega)$ vs $\omega$
        \item 平均SDを赤破線で表示
    \end{itemize}
    
    \item \textbf{拡大図}(右下)
    \begin{itemize}
        \item 0.25〜0.35 THz領域を拡大
        \item HDI境界線の視認性確認
    \end{itemize}
\end{enumerate}

\begin{figure}[h]
\centering
\begin{tikzpicture}[scale=0.6]
% 4パネルレイアウト
\draw[thick] (0,0) rectangle (6,4);
\draw[thick] (7,0) rectangle (13,4);
\draw[thick] (0,-5) rectangle (6,-1);
\draw[thick] (7,-5) rectangle (13,-1);

% ラベル
\node at (3,3.5) {\small \textbf{全体図}};
\node at (10,3.5) {\small \textbf{HDI幅}};
\node at (3,-1.5) {\small \textbf{標準偏差}};
\node at (10,-1.5) {\small \textbf{拡大図}};

% 簡易的な図示
% 全体図
\draw[gray!50] (0.5,1) -- (5.5,1);
\fill[red!20] (0.5,1.5) -- (5.5,1.8) -- (5.5,2.2) -- (0.5,2.5) -- cycle;
\draw[red, dashed] (0.5,1.5) -- (5.5,1.8);
\draw[red, dashed] (0.5,2.5) -- (5.5,2.2);
\draw[darkgray, thick] (0.5,2) -- (5.5,2);

% HDI幅
\draw[blue] (7.5,2) sin (8.5,2.5) cos (9.5,2) sin (10.5,2.5) cos (11.5,2) sin (12.5,2.5);
\draw[red, dashed] (7.5,2.2) -- (12.5,2.2);

% 標準偏差
\draw[green!60!black] (0.5,-3) sin (1.5,-2.5) cos (2.5,-3) sin (3.5,-2.5) cos (4.5,-3) sin (5.5,-2.5);
\draw[red, dashed] (0.5,-2.8) -- (5.5,-2.8);

% 拡大図
\fill[red!20] (7.5,-2.5) rectangle (12.5,-3.5);
\draw[red, dashed, thick] (7.5,-2.5) -- (12.5,-2.5);
\draw[red, dashed, thick] (7.5,-3.5) -- (12.5,-3.5);
\draw[darkgray, thick] (7.5,-3) -- (12.5,-3);

\end{tikzpicture}
\caption{テストプロットの4パネル構成}
\label{fig:panels}
\end{figure}

%%%%%%%%%%%%%%%%%%%%%%%%%%%%%%%%%%%%%%%%%%%%%%%%%%%%%%%%%%%%%%%%%
\section{診断結果の解釈}
%%%%%%%%%%%%%%%%%%%%%%%%%%%%%%%%%%%%%%%%%%%%%%%%%%%%%%%%%%%%%%%%%

\subsection{正常な場合}

\begin{lstlisting}
🔍 診断結果
======================================================================
✅ HDI区間は正しく計算されています
✅ 破線は描画されていますが、HDI幅が非常に狭いため
   平均値の線と重なって見えにくい可能性があります
\end{lstlisting}

この場合、計算は正しく行われており、HDI幅が狭いのは
事後分布の分散が小さいためです。物理的に意味がある結果かもしれません。

\subsection{問題がある場合}

\begin{lstlisting}
🔍 診断結果
======================================================================
⚠️ HDI幅が極端に狭い(平均 < 0.001)
   → 事後分布の分散が小さすぎる可能性があります
   → 事前分布が狭すぎるか、データの情報量が強すぎる可能性
⚠️ 標準偏差が極端に小さい(平均 < 0.001)
   → サンプル間の揺らぎがほとんどない

問題が検出されました:
  ⚠️ HDI幅が極端に狭い(平均 < 0.001)
  ⚠️ 標準偏差が極端に小さい(平均 < 0.001)
\end{lstlisting}

\subsection{推奨事項}

問題がある場合の対処法:

\begin{enumerate}
    \item \textbf{事前分布の$\sigma$を拡大}(3〜5倍)
    \begin{lstlisting}[language=Python]
# 変更前
sigma=0.05
# 変更後
sigma=0.15  # 3倍に拡大
    \end{lstlisting}
    
    \item \textbf{サンプル数を増加}
    \begin{lstlisting}[language=Python]
# 変更前
n_samples=100
# 変更後
n_samples=500  # 5倍に増加
    \end{lstlisting}
    
    \item \textbf{プロットの線幅を増加}
    \begin{lstlisting}[language=Python]
# 変更前
linewidth=2.0
# 変更後
linewidth=3.0
    \end{lstlisting}
\end{enumerate}

%%%%%%%%%%%%%%%%%%%%%%%%%%%%%%%%%%%%%%%%%%%%%%%%%%%%%%%%%%%%%%%%%
\section{主要関数の説明}
%%%%%%%%%%%%%%%%%%%%%%%%%%%%%%%%%%%%%%%%%%%%%%%%%%%%%%%%%%%%%%%%%

\subsection{calculate\_transmission\_for\_params}

\begin{lstlisting}[language=Python]
def calculate_transmission_for_params(freq, B, T, g, a, B4, B6, 
                                       eps, gamma_array, model_form='H'):
    """
    指定パラメータで透過スペクトルを計算
    
    Parameters
    ----------
    freq : array
        周波数配列 [THz]
    B : float
        磁場 [T]
    T : float
        温度 [K]
    g, a, B4, B6, eps : float
        物理パラメータ
    gamma_array : array
        7個の緩和率 [THz]
    model_form : str
        'H' または 'B'
    
    Returns
    -------
    trans : array
        透過率配列
    """
\end{lstlisting}

\subsection{test\_hdi\_calculation}

\begin{lstlisting}[language=Python]
def test_hdi_calculation(results_dir, n_samples=100):
    """
    HDI区間の計算と統計を確認
    
    処理内容:
    1. トレース読み込み
    2. 事後サンプル取得
    3. パラメータ統計表示
    4. 透過スペクトル計算(各サンプル)
    5. 統計量計算(平均、SD、HDI)
    6. 診断判定
    7. テストプロット生成
    
    出力:
    - コンソールへの診断結果
    - plots/hdi_test_diagnostic.png
    """
\end{lstlisting}

%%%%%%%%%%%%%%%%%%%%%%%%%%%%%%%%%%%%%%%%%%%%%%%%%%%%%%%%%%%%%%%%%
\section{まとめ}
%%%%%%%%%%%%%%%%%%%%%%%%%%%%%%%%%%%%%%%%%%%%%%%%%%%%%%%%%%%%%%%%%

\begin{enumerate}
    \item \textbf{目的}:HDI信頼区間の計算・描画が正しいかを診断
    
    \item \textbf{診断項目}:
    \begin{itemize}
        \item HDI幅の統計(最小、最大、平均)
        \item 標準偏差の統計
        \item 特定周波数点でのサンプル分布
    \end{itemize}
    
    \item \textbf{出力}:
    \begin{itemize}
        \item 4パネル診断プロット
        \item コンソール診断レポート
    \end{itemize}
    
    \item \textbf{典型的な問題と対処}:
    \begin{itemize}
        \item HDI幅が狭すぎる → 事前分布$\sigma$を拡大
        \item 境界線が見えない → 線幅を増加
    \end{itemize}
\end{enumerate}

\begin{keypoint}
HDI信頼区間は、ベイズ推定の重要な成果物です。
可視化に問題がある場合は、このツールで原因を特定し、
適切な対処を行ってください。
\end{keypoint}

\end{document}
