% filepath: bayesian_v6_informed_priors_HB_documentation.tex
\documentclass[a4paper,12pt,dvipdfmx]{jlreq}
\usepackage{amsmath,amssymb,amsthm}
\usepackage{bm}
\usepackage{physics}
\usepackage{tikz}
\usetikzlibrary{shapes.geometric, arrows, positioning, fit, backgrounds, calc}
\usepackage{algorithm2e}
\usepackage{booktabs}
\usepackage{graphicx}
\usepackage{hyperref}
\usepackage{xcolor}
\usepackage{tcolorbox}
\usepackage{listings}
\usepackage{siunitx}
\usepackage{multicol}

% カラーボックス設定
\newtcolorbox{keypoint}{colback=blue!5!white,colframe=blue!75!black,title=Key Point}
\newtcolorbox{warning}{colback=red!5!white,colframe=red!75!black,title=注意}
\newtcolorbox{definition}{colback=green!5!white,colframe=green!50!black,title=定義}
\newtcolorbox{bayesian}{colback=purple!5!white,colframe=purple!75!black,title=ベイズ推定}

\lstset{
    basicstyle=\footnotesize\ttfamily,
    keywordstyle=\color{blue},
    commentstyle=\color{gray},
    breaklines=true,
    frame=single
}

\title{\vspace{-2cm}
\textbf{bayesian\_v6\_informed\_priors\_HB.py 理解ドキュメント}\\
\large 情報的事前分布によるベイズ推定(H形式・B形式同時処理)
}
\author{物理工学専攻 修士論文解析用教材}
\date{2026年1月版}

\begin{document}
\maketitle

\tableofcontents
\clearpage

%%%%%%%%%%%%%%%%%%%%%%%%%%%%%%%%%%%%%%%%%%%%%%%%%%%%%%%%%%%%%%%%%
\section{はじめに:このプログラムの位置づけ}
%%%%%%%%%%%%%%%%%%%%%%%%%%%%%%%%%%%%%%%%%%%%%%%%%%%%%%%%%%%%%%%%%

\begin{keypoint}
このプログラムは、\textbf{pre\_test\_v6\_shared\_gamma.py}で得られた最適化結果を
\textbf{情報的事前分布(Informative Prior)}として活用し、
マルコフ連鎖モンテカルロ法(MCMC)によるベイズ推定を行います。
H形式とB形式の両方を同時に処理し、不確かさを定量化します。
\end{keypoint}

\subsection{解析パイプラインにおける役割}

\begin{figure}[h]
\centering
\begin{tikzpicture}[
    node distance=0.8cm,
    box/.style={rectangle, rounded corners, minimum width=4cm, minimum height=1cm, 
                text centered, draw=black, fill=white, align=center},
    arrow/.style={thick,->,>=stealth}
]
\node (v6) [box, fill=orange!30] {pre\_test\_v6\_shared\_gamma.py\\(最小二乗法フィット)};
\node (output1) [box, below=of v6, fill=yellow!30] {v6最適化結果\\(12パラメータ点推定値)};
\node (bayes) [box, below=of output1, fill=blue!30] {bayesian\_v6\_informed\_priors\_HB.py\\(ベイズ推定)};
\node (output2) [box, below=of bayes, fill=green!30] {事後分布\\(パラメータ不確かさ付き)};

\draw [arrow] (v6) -- (output1);
\draw [arrow] (output1) -- node[right] {事前分布の中心値} (bayes);
\draw [arrow] (bayes) -- (output2);

% 実験データ
\node (data) [box, left=2cm of bayes, fill=gray!20] {実験データ\\(10条件)};
\draw [arrow] (data) -- (bayes);
\end{tikzpicture}
\caption{解析パイプラインにおける本プログラムの位置づけ}
\label{fig:pipeline}
\end{figure}

\subsection{なぜベイズ推定が必要か?}

最小二乗法フィット(v6)で得られるのは\textbf{点推定値}のみです。
ベイズ推定により以下が可能になります:

\begin{enumerate}
    \item \textbf{パラメータの不確かさ}:各パラメータの信頼区間を定量化
    \item \textbf{パラメータ間相関}:$g$と$B_4$の相関など、物理的制約を反映
    \item \textbf{モデル比較}:H形式とB形式の確率的優劣を評価
    \item \textbf{予測分布}:将来の測定に対する予測と不確かさ
\end{enumerate}

%%%%%%%%%%%%%%%%%%%%%%%%%%%%%%%%%%%%%%%%%%%%%%%%%%%%%%%%%%%%%%%%%
\section{ベイズ推定の基礎理論}
%%%%%%%%%%%%%%%%%%%%%%%%%%%%%%%%%%%%%%%%%%%%%%%%%%%%%%%%%%%%%%%%%

\subsection{ベイズの定理}

\begin{bayesian}
パラメータ$\bm{\theta}$の事後分布$P(\bm{\theta}|D)$は、ベイズの定理により:
\begin{equation}
\boxed{
P(\bm{\theta}|D) = \frac{P(D|\bm{\theta}) \cdot P(\bm{\theta})}{P(D)}
}
\label{eq:bayes}
\end{equation}
ここで:
\begin{itemize}
    \item $P(\bm{\theta}|D)$: \textbf{事後分布}(データを見た後のパラメータの確率分布)
    \item $P(D|\bm{\theta})$: \textbf{尤度}(パラメータが与えられたときのデータの確率)
    \item $P(\bm{\theta})$: \textbf{事前分布}(データを見る前のパラメータの確率分布)
    \item $P(D)$: \textbf{周辺尤度}(正規化定数)
\end{itemize}
\end{bayesian}

\subsection{情報的事前分布}

\begin{definition}
\textbf{情報的事前分布(Informative Prior)}は、既存の知識(先行研究、物理的制約、
または予備解析結果)を反映した事前分布です。本プログラムでは、v6最適化結果を
事前分布の中心値として使用します。
\end{definition}

対照的に、\textbf{無情報事前分布(Non-informative Prior)}は広い範囲で一様な分布で、
事前知識を入れない場合に使用されます。

\begin{figure}[h]
\centering
\begin{tikzpicture}[scale=0.9]
    % 無情報事前分布
    \begin{scope}
        \draw[->] (0,0) -- (5,0) node[right] {$\theta$};
        \draw[->] (0,0) -- (0,2.5) node[above] {$P(\theta)$};
        \draw[thick, blue] (0.3,1.5) -- (4.7,1.5);
        \node[below] at (2.5,-0.3) {無情報事前分布};
        \draw[dashed] (0.3,0) -- (0.3,1.5);
        \draw[dashed] (4.7,0) -- (4.7,1.5);
    \end{scope}
    
    % 情報的事前分布
    \begin{scope}[xshift=7cm]
        \draw[->] (0,0) -- (5,0) node[right] {$\theta$};
        \draw[->] (0,0) -- (0,2.5) node[above] {$P(\theta)$};
        \draw[thick, red, domain=0.3:4.7, samples=50] 
            plot (\x, {2*exp(-(\x-2.5)*(\x-2.5)/0.5)});
        \node[below] at (2.5,-0.3) {情報的事前分布};
        \draw[dashed] (2.5,0) -- (2.5,2);
        \node[above] at (2.5,2) {$\theta_{v6}$};
    \end{scope}
\end{tikzpicture}
\caption{無情報事前分布(左)と情報的事前分布(右)。本プログラムでは右の形式を採用。}
\label{fig:priors}
\end{figure}

\subsection{尤度関数}

観測データ$\{T_i^{\text{obs}}\}$とモデル予測$\{T_i^{\text{model}}(\bm{\theta})\}$に対して、
ガウス尤度を仮定:

\begin{equation}
P(D|\bm{\theta}) = \prod_{i=1}^{N} \frac{1}{\sqrt{2\pi}\sigma_i^{\text{eff}}} 
\exp\left[ -\frac{(T_i^{\text{obs}} - T_i^{\text{model}}(\bm{\theta}))^2}{2(\sigma_i^{\text{eff}})^2} \right]
\label{eq:likelihood}
\end{equation}

対数尤度は:
\begin{equation}
\log P(D|\bm{\theta}) = -\frac{1}{2} \sum_{i=1}^{N} 
\left[ \frac{(T_i^{\text{obs}} - T_i^{\text{model}})^2}{(\sigma_i^{\text{eff}})^2} + \log(2\pi (\sigma_i^{\text{eff}})^2) \right]
\end{equation}

%%%%%%%%%%%%%%%%%%%%%%%%%%%%%%%%%%%%%%%%%%%%%%%%%%%%%%%%%%%%%%%%%
\section{事前分布の設計}
%%%%%%%%%%%%%%%%%%%%%%%%%%%%%%%%%%%%%%%%%%%%%%%%%%%%%%%%%%%%%%%%%

\subsection{切断正規分布}

本プログラムでは、全パラメータに\textbf{切断正規分布(Truncated Normal)}を使用します:

\begin{equation}
P(\theta) = \begin{cases}
\displaystyle \frac{1}{Z} \cdot \frac{1}{\sqrt{2\pi}\sigma} 
\exp\left[-\frac{(\theta - \mu)^2}{2\sigma^2}\right] & \text{if } L \leq \theta \leq U \\
0 & \text{otherwise}
\end{cases}
\end{equation}

ここで$Z$は正規化定数、$\mu$はv6最適化結果、$\sigma$は不確かさ、$[L, U]$は物理的境界。

\begin{figure}[h]
\centering
\begin{tikzpicture}[scale=0.9]
    \draw[->] (0,0) -- (8,0) node[right] {$\theta$};
    \draw[->] (0,0) -- (0,3) node[above] {$P(\theta)$};
    
    % 通常の正規分布(点線)
    \draw[dashed, gray, domain=0:8, samples=100] 
        plot (\x, {2.5*exp(-(\x-4)*(\x-4)/1.5)});
    
    % 切断正規分布(実線)
    \draw[thick, blue, domain=1.5:6.5, samples=100] 
        plot (\x, {2.5*exp(-(\x-4)*(\x-4)/1.5)});
    \draw[thick, blue] (1.5,0) -- (1.5,{2.5*exp(-(1.5-4)*(1.5-4)/1.5)});
    \draw[thick, blue] (6.5,0) -- (6.5,{2.5*exp(-(6.5-4)*(6.5-4)/1.5)});
    
    % アノテーション
    \draw[<->, thick] (1.5,-0.3) -- (6.5,-0.3);
    \node[below] at (4,-0.5) {物理的許容範囲 $[L, U]$};
    
    \draw[dashed] (4,0) -- (4,2.7);
    \node[above] at (4,2.7) {$\mu = \theta_{v6}$};
    
    % 凡例
    \node[gray] at (7,2) {通常正規分布};
    \node[blue] at (7,1.5) {切断正規分布};
\end{tikzpicture}
\caption{切断正規分布:物理的制約を満たす範囲のみで定義}
\label{fig:truncnorm}
\end{figure}

\subsection{各パラメータの事前分布設計}

\begin{table}[h]
\centering
\caption{事前分布の設計(H形式の例)}
\begin{tabular}{lcccc}
\toprule
\textbf{パラメータ} & \textbf{中心値 $\mu$} & \textbf{標準偏差 $\sigma$} & \textbf{下限 $L$} & \textbf{上限 $U$} \\
\midrule
$g$ (g因子) & $g_{v6}$ & 0.05 & $\max(1.7, g_{v6}-0.15)$ & $\min(2.1, g_{v6}+0.15)$ \\
$a$ (スケール) & $a_{v6}$ & 1.0 & $\max(8.0, a_{v6}-3.0)$ & $\min(15.0, a_{v6}+3.0)$ \\
$B_4$ (結晶場) & $B_{4,v6}$ & 0.0005 & $\max(0.005, B_{4,v6}-0.002)$ & $\min(0.012, B_{4,v6}+0.002)$ \\
$B_6$ (結晶場) & $B_{6,v6}$ & 0.00003 & $\max(-0.0004, B_{6,v6}-0.0001)$ & $\min(-0.0001, B_{6,v6}+0.0001)$ \\
$\varepsilon_{\text{bg}}$ & $\varepsilon_{v6}$ & 0.3 & $\max(13.0, \varepsilon_{v6}-1.0)$ & $\min(16.0, \varepsilon_{v6}+1.0)$ \\
$\gamma_k$ (緩和率) & $\gamma_{k,v6}$ & $0.15 \gamma_{k,v6}$ & $\max(0.005, 0.7\gamma_{k,v6})$ & $\min(0.4, 1.5\gamma_{k,v6})$ \\
\bottomrule
\end{tabular}
\label{tab:priors}
\end{table}

\begin{keypoint}
事前分布の幅($\sigma$)の設定は重要です:
\begin{itemize}
    \item \textbf{狭すぎる}$\to$ v6結果に過度に拘束され、データの情報を無視
    \item \textbf{広すぎる}$\to$ 収束が遅く、v6の利点を活かせない
    \item \textbf{適切な幅}$\to$ v6結果を中心に、データに応じて微調整
\end{itemize}
本プログラムでは、$\gamma_k$に対して15\%の相対幅を採用しています。
\end{keypoint}

%%%%%%%%%%%%%%%%%%%%%%%%%%%%%%%%%%%%%%%%%%%%%%%%%%%%%%%%%%%%%%%%%
\section{MCMCサンプリング}
%%%%%%%%%%%%%%%%%%%%%%%%%%%%%%%%%%%%%%%%%%%%%%%%%%%%%%%%%%%%%%%%%

\subsection{マルコフ連鎖モンテカルロ法の原理}

\begin{definition}
\textbf{MCMC(Markov Chain Monte Carlo)}は、複雑な確率分布からサンプルを生成する
アルゴリズムです。事後分布$P(\bm{\theta}|D)$から直接サンプリングできない場合に、
その分布に従うサンプル列を生成します。
\end{definition}

基本的なアイデア:
\begin{enumerate}
    \item 初期パラメータ$\bm{\theta}^{(0)}$から開始
    \item 提案分布から候補$\bm{\theta}'$を生成
    \item 受理確率に基づいて候補を受理または棄却
    \item 十分な回数繰り返すと、サンプルは事後分布に収束
\end{enumerate}

\subsection{Sliceサンプリング}

本プログラムでは\textbf{Sliceサンプラー}を使用します。

\begin{figure}[h]
\centering
\begin{tikzpicture}[scale=0.85]
    % 確率密度関数
    \draw[->] (0,0) -- (8,0) node[right] {$\theta$};
    \draw[->] (0,0) -- (0,4) node[above] {$P(\theta|D)$};
    
    \draw[thick, blue, domain=0.5:7.5, samples=100] 
        plot (\x, {3*exp(-(\x-3)^2/2) + 1.5*exp(-(\x-6)^2/0.8)});
    
    % 現在点
    \fill[red] (3,2.8) circle (3pt);
    \node[above] at (3,3) {$\theta^{(t)}$};
    
    % スライス(水平線)
    \draw[dashed, orange, thick] (0.8,1.5) -- (6.8,1.5);
    \node[left] at (0.8,1.5) {$u$};
    
    % スライス内の範囲
    \draw[<->, green!50!black, thick] (1.5,1.3) -- (4.2,1.3);
    \draw[<->, green!50!black, thick] (5.4,1.3) -- (6.8,1.3);
    
    % 新しい点
    \fill[green!50!black] (5.8,1.5) circle (3pt);
    \node[below] at (5.8,1.1) {$\theta^{(t+1)}$};
    
    % 説明
    \node[align=left] at (10,2.5) {
        1. 現在点$\theta^{(t)}$\\
        2. 高さ$u$をランダムに選択\\
        3. スライス内で$\theta'$をサンプル\\
        4. $\theta^{(t+1)} = \theta'$
    };
\end{tikzpicture}
\caption{Sliceサンプリングの概念図。確率密度の「スライス」内で一様にサンプリング。}
\label{fig:slice}
\end{figure}

\begin{keypoint}
Sliceサンプラーの利点:
\begin{itemize}
    \item \textbf{勾配不要}:微分できない関数でも適用可能
    \item \textbf{チューニング不要}:メトロポリス法のようなステップサイズ調整が不要
    \item \textbf{CPU環境対応}:GPUなしでも効率的に動作
\end{itemize}
\end{keypoint}

\subsection{サンプリング設定}

\begin{table}[h]
\centering
\caption{MCMCサンプリング設定}
\begin{tabular}{lcc}
\toprule
\textbf{設定項目} & \textbf{値} & \textbf{説明} \\
\midrule
Chains(チェーン数) & 8 & 独立した並列マルコフ連鎖 \\
Draws(サンプル数) & 4,000 & 各チェーンの事後サンプル数 \\
Tune(調整期間) & 2,000 & ウォームアップ(破棄される) \\
Cores(並列コア) & 8 & CPU並列処理コア数 \\
合計サンプル数 & $8 \times 4,000 = 32,000$ & 事後分布の推定に使用 \\
\bottomrule
\end{tabular}
\end{table}

%%%%%%%%%%%%%%%%%%%%%%%%%%%%%%%%%%%%%%%%%%%%%%%%%%%%%%%%%%%%%%%%%
\section{重み付け戦略}
%%%%%%%%%%%%%%%%%%%%%%%%%%%%%%%%%%%%%%%%%%%%%%%%%%%%%%%%%%%%%%%%%

\subsection{3段階重み付け}

尤度計算における実効的ノイズ$\sigma_i^{\text{eff}}$は、周波数領域に応じて調整:

\begin{equation}
\sigma_i^{\text{eff}} = \frac{\sigma_0}{\sqrt{w_i}}
\end{equation}

ここで$\sigma_0 = 0.01$(基準ノイズ)、$w_i$は領域重み:

\begin{table}[h]
\centering
\caption{周波数領域別の重み設定}
\begin{tabular}{lccc}
\toprule
\textbf{領域} & \textbf{周波数範囲} & \textbf{重み $w$} & \textbf{実効$\sigma$} \\
\midrule
ポラリトン & $f \leq 0.3615$ THz & 1.5 & $0.01/\sqrt{1.5} \approx 0.0082$ \\
共振器モード & $f \geq 0.45$ THz & 1.0 & $0.01$ \\
背景(その他) & それ以外 & 0.01 & $0.01/\sqrt{0.01} = 0.1$ \\
\bottomrule
\end{tabular}
\end{table}

\begin{figure}[h]
\centering
\begin{tikzpicture}[scale=0.9]
    \draw[->] (0,0) -- (10,0) node[right] {Frequency [THz]};
    \draw[->] (0,0) -- (0,4) node[above] {$\sigma^{\text{eff}}$};
    
    % 領域の塗りつぶし
    \fill[red!20] (0,0) rectangle (2.5,0.8);
    \fill[cyan!20] (4.5,0) rectangle (9,1.0);
    \fill[gray!10] (2.5,0) rectangle (4.5,3.5);
    
    % 実効σの値
    \draw[thick, red] (0,0.8) -- (2.5,0.8);
    \draw[thick, blue] (4.5,1.0) -- (9,1.0);
    \draw[thick, gray] (2.5,3.5) -- (4.5,3.5);
    
    % 縦線
    \draw[dashed] (2.5,0) -- (2.5,4);
    \draw[dashed] (4.5,0) -- (4.5,4);
    
    % ラベル
    \node at (1.25,0.5) {\small ポラリトン};
    \node at (1.25,0.2) {\small (1.5$\times$)};
    \node at (3.5,2) {\small 背景};
    \node at (3.5,1.7) {\small (0.01$\times$)};
    \node at (6.75,0.5) {\small 共振器};
    \node at (6.75,0.2) {\small (1.0$\times$)};
    
    % 数値
    \node[left] at (0,0.8) {\small 0.0082};
    \node[left] at (0,1.0) {\small 0.01};
    \node[left] at (0,3.5) {\small 0.1};
    
    \node[below] at (2.5,0) {\small 0.36};
    \node[below] at (4.5,0) {\small 0.45};
\end{tikzpicture}
\caption{周波数領域別の実効ノイズ$\sigma^{\text{eff}}$。低い値ほどフィットへの寄与が大きい。}
\label{fig:weighting}
\end{figure}

\begin{keypoint}
この重み付けの意味:
\begin{itemize}
    \item \textbf{ポラリトン領域}:物理的に重要、高精度でフィット
    \item \textbf{共振器モード}:分光学的情報を持つ、適度な精度
    \item \textbf{背景領域}:情報が少ない、フィットへの影響を抑制
\end{itemize}
\end{keypoint}

%%%%%%%%%%%%%%%%%%%%%%%%%%%%%%%%%%%%%%%%%%%%%%%%%%%%%%%%%%%%%%%%%
\section{PyTensor Opによる物理モデル実装}
%%%%%%%%%%%%%%%%%%%%%%%%%%%%%%%%%%%%%%%%%%%%%%%%%%%%%%%%%%%%%%%%%

\subsection{カスタムOpの必要性}

PyMC(ベイズ推定ライブラリ)では、モデル関数を計算グラフとして定義する必要があります。
しかし、本プログラムの物理モデル(ハミルトニアン対角化、感受率計算など)は
標準のPyTensor演算で直接表現できません。

そこで、\textbf{カスタムOp}を定義してNumPy計算をラップします。

\subsection{InformedPriorModelOp クラス}

\begin{lstlisting}[language=Python, caption=カスタムOpの構造]
class InformedPriorModelOp(Op):
    """情報的事前分布を持つモデルOp(H/B形式選択可能)"""
    
    def __init__(self, datasets, model_form='H'):
        self.datasets = datasets      # 全10データセット
        self.model_form = model_form  # 'H' or 'B'
    
    def make_node(self, a_scale, gamma_vec, g_factor, B4, B6, eps_bg):
        # 入力テンソルの定義
        # 出力: 全周波数点の予測透過率(連結)
        
    def perform(self, node, inputs, output_storage):
        # 実際の計算(NumPy)
        # 各データセットに対して:
        #   1. ハミルトニアン構築
        #   2. 感受率計算
        #   3. 比透磁率計算(H形式 or B形式)
        #   4. 透過率計算
\end{lstlisting}

\begin{figure}[h]
\centering
\begin{tikzpicture}[
    node distance=0.6cm,
    box/.style={rectangle, minimum width=2.5cm, minimum height=0.7cm, 
                text centered, draw=black},
    arrow/.style={thick,->,>=stealth}
]
% 入力パラメータ
\node (g) [box, fill=orange!30] {$g$};
\node (a) [box, right=0.5cm of g, fill=orange!30] {$a$};
\node (B4) [box, right=0.5cm of a, fill=orange!30] {$B_4$};
\node (B6) [box, right=0.5cm of B4, fill=orange!30] {$B_6$};
\node (eps) [box, right=0.5cm of B6, fill=orange!30] {$\varepsilon$};
\node (gamma) [box, right=0.5cm of eps, fill=orange!30] {$\gamma_{1..7}$};

% カスタムOp
\node (op) [box, below=1cm of B6, minimum width=10cm, fill=blue!30] 
    {InformedPriorModelOp (NumPy計算)};

% 内部処理
\node (h1) [box, below=0.8cm of op, fill=yellow!20, minimum width=2cm] {$\hat{H}$};
\node (chi) [box, right=0.5cm of h1, fill=yellow!20, minimum width=2cm] {$\chi(\omega)$};
\node (mur) [box, right=0.5cm of chi, fill=yellow!20, minimum width=2cm] {$\mu_r$};
\node (trans) [box, right=0.5cm of mur, fill=yellow!20, minimum width=2cm] {$T(\omega)$};

% 出力
\node (output) [box, below=0.8cm of chi, fill=green!30, minimum width=6cm] 
    {予測透過率 $\{T_i^{\text{model}}\}$};

% 矢印
\draw [arrow] (g) |- (op);
\draw [arrow] (a) |- (op);
\draw [arrow] (B4) -- (op);
\draw [arrow] (B6) -- (op);
\draw [arrow] (eps) |- (op);
\draw [arrow] (gamma) |- (op);

\draw [arrow] (op) -- (h1);
\draw [arrow] (h1) -- (chi);
\draw [arrow] (chi) -- (mur);
\draw [arrow] (mur) -- (trans);
\draw [arrow] (trans) |- (output);

\end{tikzpicture}
\caption{InformedPriorModelOpの計算フロー}
\label{fig:op}
\end{figure}

%%%%%%%%%%%%%%%%%%%%%%%%%%%%%%%%%%%%%%%%%%%%%%%%%%%%%%%%%%%%%%%%%
\section{H形式とB形式の同時処理}
%%%%%%%%%%%%%%%%%%%%%%%%%%%%%%%%%%%%%%%%%%%%%%%%%%%%%%%%%%%%%%%%%

\subsection{処理フロー}

本プログラムの最大の特徴は、H形式とB形式を\textbf{独立に}、しかし\textbf{同時に}
処理することです。

\begin{figure}[h]
\centering
\begin{tikzpicture}[
    node distance=0.8cm,
    box/.style={rectangle, rounded corners, minimum width=3.5cm, minimum height=0.8cm, 
                text centered, draw=black},
    arrow/.style={thick,->,>=stealth}
]

% 共通部分
\node (v6H) [box, fill=orange!30] {v6\_H最適化結果};
\node (v6B) [box, right=4cm of v6H, fill=orange!30] {v6\_B最適化結果};
\node (data) [box, below=1cm of $(v6H)!0.5!(v6B)$, fill=gray!20] {実験データ (10条件)};

% H形式パイプライン
\node (priorH) [box, below left=1.5cm and 0.5cm of data, fill=red!20] {H形式事前分布};
\node (modelH) [box, below=0.6cm of priorH, fill=red!30] {H形式モデル};
\node (mcmcH) [box, below=0.6cm of modelH, fill=red!40] {MCMC (H形式)};
\node (postH) [box, below=0.6cm of mcmcH, fill=red!50] {H形式事後分布};

% B形式パイプライン
\node (priorB) [box, below right=1.5cm and 0.5cm of data, fill=blue!20] {B形式事前分布};
\node (modelB) [box, below=0.6cm of priorB, fill=blue!30] {B形式モデル};
\node (mcmcB) [box, below=0.6cm of modelB, fill=blue!40] {MCMC (B形式)};
\node (postB) [box, below=0.6cm of mcmcB, fill=blue!50] {B形式事後分布};

% 比較
\node (compare) [box, below=1.5cm of $(postH)!0.5!(postB)$, fill=green!30, minimum width=5cm] 
    {H形式 vs B形式 比較};

% 矢印
\draw [arrow] (v6H) |- (priorH);
\draw [arrow] (v6B) |- (priorB);
\draw [arrow] (data) -| (priorH);
\draw [arrow] (data) -| (priorB);

\draw [arrow] (priorH) -- (modelH);
\draw [arrow] (modelH) -- (mcmcH);
\draw [arrow] (mcmcH) -- (postH);

\draw [arrow] (priorB) -- (modelB);
\draw [arrow] (modelB) -- (mcmcB);
\draw [arrow] (mcmcB) -- (postB);

\draw [arrow] (postH) |- (compare);
\draw [arrow] (postB) |- (compare);

\end{tikzpicture}
\caption{H形式とB形式の並列処理フロー}
\label{fig:parallel}
\end{figure}

\subsection{なぜ別々の事前分布を使うか?}

v6最適化の結果、H形式とB形式では異なるパラメータ値が得られます:

\begin{itemize}
    \item H形式: $\mu_r = 1 + \chi$ $\to$ 弱磁化近似、パラメータ$\bm{\theta}^{(H)}$
    \item B形式: $\mu_r = 1/(1-\chi)$ $\to$ 強磁化補正、パラメータ$\bm{\theta}^{(B)}$
\end{itemize}

これらは異なる物理モデルなので、それぞれに適した事前分布を使用します。

%%%%%%%%%%%%%%%%%%%%%%%%%%%%%%%%%%%%%%%%%%%%%%%%%%%%%%%%%%%%%%%%%
\section{収束診断}
%%%%%%%%%%%%%%%%%%%%%%%%%%%%%%%%%%%%%%%%%%%%%%%%%%%%%%%%%%%%%%%%%

\subsection{MCMCが収束したかどうかの判定}

MCMCサンプルが事後分布からの正しいサンプルであるためには、
マルコフ連鎖が定常分布(事後分布)に\textbf{収束}している必要があります。

\subsection{$\hat{R}$統計量(Gelman-Rubin診断)}

\begin{definition}
\textbf{$\hat{R}$(R-hat)}は、複数のチェーン間のばらつきとチェーン内のばらつきを比較する指標です:
\begin{equation}
\hat{R} = \sqrt{\frac{\text{Var}_{\text{between}} + \text{Var}_{\text{within}}}{\text{Var}_{\text{within}}}}
\end{equation}
\end{definition}

\begin{table}[h]
\centering
\caption{$\hat{R}$の解釈}
\begin{tabular}{cl}
\toprule
\textbf{$\hat{R}$の値} & \textbf{解釈} \\
\midrule
$< 1.01$ & 優れた収束 ✓ \\
$1.01 - 1.05$ & 許容範囲 \\
$1.05 - 1.1$ & 要注意 \\
$> 1.1$ & 未収束、サンプル数増加が必要 \\
\bottomrule
\end{tabular}
\end{table}

\subsection{有効サンプルサイズ(ESS)}

MCMCサンプルは自己相関を持つため、独立なサンプル数は総サンプル数より少なくなります。

\begin{definition}
\textbf{ESS(Effective Sample Size)}は、自己相関を考慮した実質的な独立サンプル数:
\begin{equation}
\text{ESS} = \frac{N}{1 + 2\sum_{k=1}^{\infty} \rho_k}
\end{equation}
ここで$\rho_k$はラグ$k$の自己相関係数。
\end{definition}

目安として、$\text{ESS} > 400$が推奨されます。

\subsection{トレースプロット}

\begin{figure}[h]
\centering
\begin{tikzpicture}[scale=0.7]
% 良好な収束
\begin{scope}
    \draw[->] (0,0) -- (6,0) node[right] {\small Iteration};
    \draw[->] (0,0) -- (0,3) node[above] {\small $\theta$};
    
    % 複数チェーン
    \foreach \seed in {1,2,3,4} {
        \draw[blue!\seed0!red, opacity=0.7] 
            plot[domain=0:5.5, samples=50, smooth] 
            (\x, {1.5 + 0.3*rand});
    }
    \node[below] at (3,-0.5) {\textbf{良好な収束}};
\end{scope}

% 未収束
\begin{scope}[xshift=9cm]
    \draw[->] (0,0) -- (6,0) node[right] {\small Iteration};
    \draw[->] (0,0) -- (0,3) node[above] {\small $\theta$};
    
    % 発散するチェーン
    \draw[blue, opacity=0.7] plot[domain=0:5.5, samples=30, smooth] 
        (\x, {0.5 + 0.3*\x + 0.2*rand});
    \draw[red, opacity=0.7] plot[domain=0:5.5, samples=30, smooth] 
        (\x, {2.5 - 0.2*\x + 0.2*rand});
    \draw[green!60!black, opacity=0.7] plot[domain=0:5.5, samples=30, smooth] 
        (\x, {1.5 + 0.15*\x + 0.2*rand});
    
    \node[below] at (3,-0.5) {\textbf{未収束}};
\end{scope}
\end{tikzpicture}
\caption{トレースプロットによる収束診断。左:全チェーンが同じ領域を探索(良好)。右:チェーンが異なる領域にいる(問題あり)。}
\label{fig:trace}
\end{figure}

%%%%%%%%%%%%%%%%%%%%%%%%%%%%%%%%%%%%%%%%%%%%%%%%%%%%%%%%%%%%%%%%%
\section{出力ファイルと結果の解釈}
%%%%%%%%%%%%%%%%%%%%%%%%%%%%%%%%%%%%%%%%%%%%%%%%%%%%%%%%%%%%%%%%%

\subsection{生成されるファイル一覧}

\begin{table}[h]
\centering
\caption{出力ファイル}
\begin{tabular}{ll}
\toprule
\textbf{ファイル名} & \textbf{内容} \\
\midrule
\texttt{trace\_H.nc} & H形式MCMCトレース(NetCDF形式) \\
\texttt{trace\_B.nc} & B形式MCMCトレース \\
\texttt{summary\_H.csv} & H形式パラメータ統計サマリー \\
\texttt{summary\_B.csv} & B形式パラメータ統計サマリー \\
\texttt{parameters\_H.csv} & H形式事後平均値 \\
\texttt{parameters\_B.csv} & B形式事後平均値 \\
\texttt{v6\_comparison.json} & v6最適化 vs ベイズ推定の比較 \\
\texttt{plots/trace\_H.png} & H形式トレースプロット \\
\texttt{plots/trace\_B.png} & B形式トレースプロット \\
\texttt{plots/energy\_levels\_comparison.png} & エネルギー準位比較 \\
\texttt{plots/populations\_comparison.png} & 占有確率比較 \\
\texttt{plots/susceptibility\_comparison.png} & 感受率比較 \\
\texttt{plots/transmission\_spectra\_comparison.png} & 透過スペクトル比較 \\
\bottomrule
\end{tabular}
\end{table}

\subsection{サマリーCSVの読み方}

\texttt{summary\_H.csv}には以下の列が含まれます:

\begin{table}[h]
\centering
\caption{サマリー統計量の意味}
\begin{tabular}{ll}
\toprule
\textbf{列名} & \textbf{意味} \\
\midrule
\texttt{mean} & 事後平均(点推定値として使用可能) \\
\texttt{sd} & 事後標準偏差(不確かさの指標) \\
\texttt{hdi\_3\%} & 94\%信頼区間の下限 \\
\texttt{hdi\_97\%} & 94\%信頼区間の上限 \\
\texttt{mcse\_mean} & モンテカルロ標準誤差(推定精度) \\
\texttt{ess\_bulk} & バルクESS \\
\texttt{ess\_tail} & テールESS \\
\texttt{r\_hat} & $\hat{R}$統計量 \\
\bottomrule
\end{tabular}
\end{table}

\subsection{結果の解釈例}

\begin{lstlisting}[caption=サマリー出力例]
           mean      sd  hdi_3%  hdi_97%  r_hat  ess_bulk
g_factor  1.952   0.012   1.930    1.974   1.00     3200
a_scale  11.45    0.82   10.02    12.91   1.01     2800
B4        0.0037  0.0003  0.0032   0.0043  1.00     3100
...
\end{lstlisting}

解釈:
\begin{itemize}
    \item $g = 1.952 \pm 0.012$(94\%信頼区間:[1.930, 1.974])
    \item 全パラメータで$\hat{R} \approx 1.0$、$\text{ESS} > 2000$ $\to$ 良好な収束
\end{itemize}

%%%%%%%%%%%%%%%%%%%%%%%%%%%%%%%%%%%%%%%%%%%%%%%%%%%%%%%%%%%%%%%%%
\section{プログラム全体フロー}
%%%%%%%%%%%%%%%%%%%%%%%%%%%%%%%%%%%%%%%%%%%%%%%%%%%%%%%%%%%%%%%%%

\begin{figure}[h]
\centering
\begin{tikzpicture}[
    node distance=1cm,
    startstop/.style={rectangle, rounded corners, minimum width=3cm, minimum height=0.7cm, 
                      text centered, draw=black, fill=red!30},
    process/.style={rectangle, minimum width=3.5cm, minimum height=0.7cm, 
                    text centered, draw=black, fill=orange!30},
    io/.style={trapezium, trapezium left angle=70, trapezium right angle=110, 
               minimum width=3cm, minimum height=0.7cm, text centered, draw=black, fill=blue!30},
    decision/.style={diamond, minimum width=2cm, minimum height=0.8cm, 
                     text centered, draw=black, fill=green!30},
    arrow/.style={thick,->,>=stealth}
]

\node (start) [startstop] {開始};
\node (loadv6) [io, below=of start] {v6最適化結果読み込み (H/B)};
\node (loaddata) [io, below=of loadv6] {実験データ読み込み (10セット)};
\node (weight) [process, below=of loaddata] {重み配列生成};

\node (modelH) [process, below left=1cm and 0.5cm of weight] {H形式モデル構築};
\node (modelB) [process, below right=1cm and 0.5cm of weight] {B形式モデル構築};

\node (mcmcH) [process, below=of modelH] {MCMCサンプリング (H)};
\node (mcmcB) [process, below=of modelB] {MCMCサンプリング (B)};

\node (save) [io, below=1.5cm of $(mcmcH)!0.5!(mcmcB)$] {結果保存 (trace, summary)};
\node (plot) [process, below=of save] {比較プロット生成};
\node (end) [startstop, below=of plot] {終了};

\draw [arrow] (start) -- (loadv6);
\draw [arrow] (loadv6) -- (loaddata);
\draw [arrow] (loaddata) -- (weight);
\draw [arrow] (weight) -| (modelH);
\draw [arrow] (weight) -| (modelB);
\draw [arrow] (modelH) -- (mcmcH);
\draw [arrow] (modelB) -- (mcmcB);
\draw [arrow] (mcmcH) |- (save);
\draw [arrow] (mcmcB) |- (save);
\draw [arrow] (save) -- (plot);
\draw [arrow] (plot) -- (end);

\end{tikzpicture}
\caption{プログラム全体のフローチャート}
\label{fig:flowchart}
\end{figure}

%%%%%%%%%%%%%%%%%%%%%%%%%%%%%%%%%%%%%%%%%%%%%%%%%%%%%%%%%%%%%%%%%
\section{物理量の可視化}
%%%%%%%%%%%%%%%%%%%%%%%%%%%%%%%%%%%%%%%%%%%%%%%%%%%%%%%%%%%%%%%%%

\subsection{生成されるプロット}

\begin{enumerate}
    \item \textbf{トレースプロット}(trace\_H.png, trace\_B.png)
    \begin{itemize}
        \item 各パラメータのMCMC軌跡
        \item 収束診断に使用
    \end{itemize}
    
    \item \textbf{エネルギー準位比較}(energy\_levels\_comparison.png)
    \begin{itemize}
        \item 各条件でのエネルギー固有値
        \item H形式 vs B形式(赤 vs 青の棒グラフ)
    \end{itemize}
    
    \item \textbf{占有確率比較}(populations\_comparison.png)
    \begin{itemize}
        \item ボルツマン分布による各準位の占有確率
        \item 温度・磁場依存性の可視化
    \end{itemize}
    
    \item \textbf{感受率比較}(susceptibility\_comparison.png)
    \begin{itemize}
        \item $|\chi(\omega)|$の周波数依存性
        \item 共鳴ピーク位置の比較
    \end{itemize}
    
    \item \textbf{透過スペクトル比較}(transmission\_spectra\_comparison.png)
    \begin{itemize}
        \item 実験データとフィット曲線
        \item RMSE表示
        \item 重み領域の色分け
    \end{itemize}
\end{enumerate}

%%%%%%%%%%%%%%%%%%%%%%%%%%%%%%%%%%%%%%%%%%%%%%%%%%%%%%%%%%%%%%%%%
\section{まとめ}
%%%%%%%%%%%%%%%%%%%%%%%%%%%%%%%%%%%%%%%%%%%%%%%%%%%%%%%%%%%%%%%%%

\begin{enumerate}
    \item \textbf{目的}:v6最適化の点推定値に確率的不確かさを付与
    
    \item \textbf{方法}:
    \begin{itemize}
        \item 情報的事前分布(v6結果を中心とした切断正規分布)
        \item Sliceサンプラーによる勾配不要MCMC
        \item 8チェーン×4,000サンプルの大規模サンプリング
    \end{itemize}
    
    \item \textbf{特徴}:
    \begin{itemize}
        \item H形式とB形式を同時処理
        \item 3段階重み付け(ポラリトン=1.5, 共振器=1.0, 背景=0.01)
        \item 豊富な診断・可視化機能
    \end{itemize}
    
    \item \textbf{出力}:
    \begin{itemize}
        \item 事後分布(トレースファイル)
        \item 統計サマリー(平均、標準偏差、信頼区間、$\hat{R}$、ESS)
        \item 比較プロット(エネルギー、占有確率、感受率、透過率)
    \end{itemize}
\end{enumerate}

\begin{keypoint}
ベイズ推定の利点:
\begin{itemize}
    \item パラメータ不確かさの定量化 $\to$ 誤差伝播の正確な評価
    \item 事前知識(v6結果)と実験データの統合
    \item モデル選択(H vs B)への拡張可能性
\end{itemize}
\end{keypoint}

%%%%%%%%%%%%%%%%%%%%%%%%%%%%%%%%%%%%%%%%%%%%%%%%%%%%%%%%%%%%%%%%%
\section{付録:用語集}
%%%%%%%%%%%%%%%%%%%%%%%%%%%%%%%%%%%%%%%%%%%%%%%%%%%%%%%%%%%%%%%%%

\begin{description}
    \item[事後分布(Posterior)] データを観測した後のパラメータの確率分布
    \item[事前分布(Prior)] データを観測する前のパラメータの確率分布
    \item[尤度(Likelihood)] パラメータが与えられたときのデータの確率
    \item[MCMC] マルコフ連鎖モンテカルロ法。複雑な分布からサンプルを生成
    \item[トレース] MCMCで生成されたパラメータサンプルの時系列
    \item[$\hat{R}$] 収束診断指標。1に近いほど良好
    \item[ESS] 有効サンプルサイズ。自己相関を考慮した独立サンプル数
    \item[切断正規分布] 範囲制限付きの正規分布
    \item[PyMC] Pythonベイズ推定ライブラリ
    \item[ArviZ] ベイズ推定結果の診断・可視化ライブラリ
    \item[PyTensor] PyMCの計算バックエンド(旧Theano)
\end{description}

\end{document}