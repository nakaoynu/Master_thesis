\chapter{結果と考察}
\label{chap:results}

本章では, 本研究で得られた主要な結果について述べる. まず, 重み付き非線形最小二乗法によるパラメータ推定結果を示し(\ref{sec:wnlls_results}節), 次にベイズ推定に基づくパラメータの不確実性評価とモデル比較結果を提示する(\ref{sec:bayes_results}節). 最後に, これらの結果に基づく物理的考察を行う(\ref{sec:discussion}節).
\section{実験データセット}
\begin{table}[h]
\centering
\caption{使用データセット(10条件)}
\begin{tabular}{lccc}
\toprule
\textbf{データセット} & \textbf{磁場 $B$ [T]} & \textbf{温度 $T$ [K]} & \textbf{パターン} \\
\midrule
4K & 9.0 & 4.0 & 温度変化 \\
10K & 9.0 & 10.0 & 温度変化 \\
20K & 9.0 & 20.0 & 温度変化 \\
30K & 9.0 & 30.0 & 温度変化 \\
4.2T & 4.2 & 1.5 & 磁場変化 \\
5T & 5.0 & 1.5 & 磁場変化 \\
6T & 6.0 & 1.5 & 磁場変化 \\
7T & 7.0 & 1.5 & 磁場変化 \\
8T & 8.0 & 1.5 & 磁場変化 \\
9T & 9.0 & 1.5 & 磁場変化 \\
\bottomrule
\end{tabular}
\end{table}

\section{重み付き非線形最小二乗法によるパラメータ推定}
\label{sec:wnlls_results}
まず, Kritzellらの実験データ\(^{\cite{Kritzell2024}}\)に対して重み付き非線形最小二乗法を適用し, HモデルとBモデル, それぞれのモデルのパラメータをフィッティングした. このフィッティングパラメータに基づき, 各準位のエネルギー固有値と占有確率, 磁気感受率, 透過スペクトルを算出した結果を以下に示す.

\subsection*{最適化パラメータと評価指標の比較}
\label{subsec:wnlls_parameters}
表\ref{tab:parameters_results_wnlls}に, 重み付き非線形最小二乗法に基づくHモデルおよびBモデルの最適化パラメータと評価指標を示す.
\begin{table}[htbp]
    \centering
    \caption{共有\(\gamma\)モデルにおける最適化パラメータおよび評価指標の比較}
    \label{tab:parameters_results_wnlls}
    \renewcommand{\arraystretch}{1.2} % 行間の調整
    \begin{tabular}{@{}lcc@{}}
        \toprule
        \textbf{パラメータ} & \textbf{Hモデル} & \textbf{Bモデル} \\
        \midrule
        \multicolumn{3}{@{}l@{}}{\textit{Global}} \\
        \quad \(g\)& 1.925 & 2.086 \\
        \quad \(a\) & 5.000 & 5.000 \\
        \quad  \(B_4\) & \(3.000 \times 10^{-2}\) & \(1.592 \times 10^{-4}\) \\
        \quad \(B_6\) & \(-1.000 \times 10^{-3}\) & \(-9.989 \times 10^{-4}\) \\
        \quad \(\varepsilon_{\rm bg}\) & 14.001 & 14.131 \\
        \midrule
        \multicolumn{3}{@{}l@{}}{\textit{Shared (\(\gamma_k\))}} \\
        \quad \(\gamma_0\)(基底状態) & 0.0245 & 0.0237 \\
        \quad \(\gamma_1\) & 0.0130 & 0.1445 \\
        \quad \(\gamma_2\) & 0.1159 & 0.1148 \\
        \quad \(\gamma_3\) & 0.0950 & 0.0100 \\
        \quad \(\gamma_4\) & 0.0100 & 0.0100 \\
        \quad \(\gamma_5\) & 0.0100 & 0.0283 \\
        \quad \(\gamma_6\) & 0.0100 & 0.0100 \\
        \midrule
        \multicolumn{3}{@{}l@{}}{\text{評価指標}} \\
        \quad Final Cost (\(S_{\text{min}}\)) & \(1.475 \times 10^{5}\) & \(1.395 \times 10^{5}\) \\
        \quad Condition Number (\(\kappa\)) & \(3.85 \times 10^{5}\) & \(1.15 \times 10^{6}\) \\
        \bottomrule
    \end{tabular}
\end{table}
\clearpage
\subsubsection*{フィッティング結果に関する考察}
両形式ともに\(g\)因子はGd\(^{3+}\)の理論値(\(g \approx 2.0\))に近い値を示した. 
結合定数スケール\(a\)は探索範囲上限の5.0に収束しており,
実験パラメータ(GGG試料の膜厚,スピン密度など)の不確定性を補償していると考えられる. 

結晶場パラメータ\(B_4\)については,HモデルとBモデルで大きく異なる値が得られた:
\begin{itemize}
    \item Hモデル:\(B_4 = 30.0\)~mK(探索範囲上限)
    \item Bモデル:\(B_4 = 0.159\)~mK(先行研究値\(\approx 2\)~mK\(^{\cite{Yamada2024}}\)より1桁小さい)
\end{itemize}
この差異は,両形式で異なる最適解が存在することを示唆している. \\

基底状態の緩和率\(\gamma_0\)は両形式で約0.024~THzと一致しており,
これは時間領域に換算すると\(\tau_0 \approx 7\)~psに相当する. 
高励起状態(\(k \geq 4\))では緩和率が下限値0.01~THzに収束する傾向が見られ,
高温領域ではBoltzmann分布により高励起状態の占有が抑制されるため,
これらのパラメータ\(\gamma_{k \geq 4}\)の感度が平均値として低いことを示している. 

条件数は共有ガンマモデルにより\(10^5\)--\(10^6\)に改善され,
56遷移すべてを考慮したモデルの\(10^{16}\)から約10桁改善された. 

\subsection*{フィッティング品質の統計解析}
\label{subsec:wnlls_statistical_analysis}
\subsubsection*{データセット別の統計量}

表\ref{tab:fit_stats_H}および表\ref{tab:fit_stats_B}に
各データセットに対するフィット品質の統計量を示す. 

\begin{table}[htbp]
\centering
\caption{Hモデルによるフィット品質統計量}
\label{tab:fit_stats_H}
\begin{tabular}{lccccc}
\hline
\textbf{ラベル} & \textbf{RMSE} & \textbf{Max Error} & \(R^2\) & \( |\chi|_{\rm max} \)\\
\hline
\multicolumn{3}{@{}l@{}}{\text{温度変化(9.0T)}} \\
4K & 0.117 & 0.224 & 0.721 & 1.10\\
10K & 0.112 & 0.246 & 0.747 & 1.04 \\
20K & 0.108 & 0.225 & 0.717 & 0.78 \\
30K & 0.118 & 0.290 & 0.652 & 0.62 \\
\hline
\multicolumn{3}{@{}l@{}}{\text{磁場変化(1.5K)}} \\
4.2T & 0.157 & 0.364 & 0.486 & 0.48\\
5.0T & 0.146 & 0.330 & 0.540 & 0.81 \\
6.0T & 0.129 & 0.248 & 0.676 & 1.05 \\
7.0T & 0.112 & 0.221 & 0.749 & 1.05 \\
8.0T & 0.117 & 0.229 & 0.759 & 1.05 \\
9.0T & 0.125 & 0.361 & 0.705 & 1.04 \\
\hline
\end{tabular}
\end{table}

\begin{table}[htbp]
\centering
\caption{Bモデルによるフィット品質統計量}
\label{tab:fit_stats_B}
\begin{tabular}{lccccc}
\hline
\textbf{ラベル} & \textbf{RMSE} & \textbf{Max Error} & \(R^2\) & \( |\chi|_{\rm max} \) & 不安定\% \\
\hline
4K & 0.112 & 0.257 & 0.744 & 1.17 & 4.6 \\
10K & 0.105 & 0.246 & 0.776 & 0.82 & 0.0 \\
20K & 0.101 & 0.231 & 0.755 & 0.72 & 0.0 \\
30K & 0.110 & 0.254 & 0.695 & 0.65 & 0.0 \\
\hline
4.2T & 0.156 & 0.340 & 0.493 & 0.70 & 0.0 \\
5.0T & 0.144 & 0.320 & 0.548 & 1.19 & 1.7 \\
6.0T & 0.125 & 0.244 & 0.694 & 1.26 & 5.7 \\
7.0T & 0.108 & 0.212 & 0.764 & 1.26 & 5.7 \\
8.0T & 0.116 & 0.331 & 0.764 & 1.27 & 5.7 \\
9.0T & 0.120 & 0.337 & 0.727 & 1.27 & 5.7 \\
\hline
\end{tabular}
\end{table}
\clearpage

\textbf{HモデルとBモデルの比較}
\begin{itemize}
    \item \textbf{全体的な傾向}:Bモデルの方がやや低いRMSE\footnote{\(\text{RMSE} = \sqrt{\frac{1}{N} \sum_{i=1}^{N} (y_i - \hat{y}_i)^2}\)で定義される. 予測値と実測値の「平均的な誤差」を示す指標である.}を示す(平均RMSE: Hモデル0.124, Bモデル0.120)
    \item \textbf{最終コスト}:Bモデル(139,482)がHモデル(147,482)より5.4\%低い
    \item \textbf{決定係数\(R^2\)}\footnote{\(R^2 = 1 - (\text{残差平方和}) / (\text{全平方和})\)で定義される.観測データの分散がモデルによってどれだけ説明されるかを示す指標である.\(R^2=1\)はモデルによるデータの完全再現を意味する.}:温度依存データでは両モデルとも\(R^2 > 0.65\)を達成
    \item \textbf{低磁場領域}:4.2T--5.0Tでは両モデルとも\(R^2 \approx 0.5\)と低く,
          改善の余地がある
\end{itemize}

\textbf{数値安定性の評価}

Bモデルでは\( |\chi| > 1 \)となる周波数点が一部存在し,
これは\(\mu_r^{(B)} = 1/(1-\chi)\)の発散を示唆する. 
この不安定点は表\ref{tab:fit_stats_B}の「不安定\%」列に示すように,
高磁場(\(B \geq 6\)~T)で約5.7\%のデータ点が不安定領域に入っている. 

\subsection*{物理的解釈}
\subsubsection*{エネルギー固有値}
\label{subsec:E_P_wnlls}
磁気感受率\(\chi (\omega)\)の計算で用した各準位 \(k\)のエネルギー固有値 \(E_0^{k}\)を図\ref{fig:wnls_E_P_HB}に示す. \\
各準位間の分裂幅が等間隔であることは,式\eqref{eq:H_Zeeman}で定義されるZeeman分裂の効果が支配的であり,式\eqref{eq:H_CF}による結晶場ハミルトニアンの寄与が相対的に小さいことを示唆する.実際に,フィッティング結果の図\ref{fig:wnls_E_P_HB}を見ると,Hモデル・Bモデルともに各準位のエネルギー固有値は磁場に対してほぼ線形に増加しており,Zeeman効果の支配を反映していることが分かる.本解析では,立方晶系GGGで本来支配的となる2次の結晶場項\(B_2\)を考慮していないため\footnote{\(B_2\)項は基底状態のゼロ磁場分裂を決定する主要項であり,通常\(B_4, B_6\)より1--2桁大きい.},\(B_4, B_6\)が見かけ上過小評価されている可能性がある.ただし,後述するスペクトルフィッティング結果(図\ref{fig:wnls_T_HB})において,全10条件でRMSE \(< 0.16\)という高い記述精度が得られており,Kritzellらが報告したGGGのTHz磁気光学応答の再現には,本近似モデルで十分であると考えられる.\\
\begin{figure}
    \centering
    \includegraphics[width=1.0\textwidth]{figures/wnlls_global_fitting_results_comparison_v6/energy_levels_HB_comparison.png}
    \caption{重み付き非線形最小二乗法によるHモデルとBモデルの各準位のエネルギー固有値. 赤はHモデル, 青はBモデルの結果を示す. }
    \label{fig:wnls_E_P_HB}
\end{figure}

\textbf{エネルギーギャップ}
表\ref{tab:energy_gaps}に各磁場条件での基底状態--第一励起状態間のエネルギーギャップを示す. 

\begin{table}[htbp]
\centering
\caption{基底状態--第一励起状態間のエネルギーギャップ}
\label{tab:energy_gaps}
\begin{tabular}{lccc}
\hline
\textbf{磁場 [T]} & \textbf{\(\Delta E\) [meV] (H形式)} & \textbf{\(\Delta E\) [meV] (B形式)} \\
\hline
4.2 & 22.4 & 24.3 \\
5.0 & 26.7 & 29.0 \\
6.0 & 32.0 & 34.8 \\
7.0 & 37.4 & 40.6 \\
8.0 & 42.7 & 46.4 \\
9.0 & 48.1 & 52.2 \\
\hline
\end{tabular}
\end{table}

エネルギーギャップは磁場に対してほぼ線形に増加しており,
これはZeeman効果の支配を反映している. 
B形式の方がやや大きいギャップを与えており,
結晶場パラメータ\(B_4\)の差異に起因すると考えられる. 
\subsubsection*{Boltzmann分布と基底状態占有率}
\label{subsec:P_wnlls}
各準位 \(k\)の占有確率 \(p_k\)を図\ref{fig:wnls_P_HB}に示す. また, 表\ref{tab:P0_results_wnlls}に各条件での基底状態占有確率をまとめる. \\

\begin{figure}
    \centering
    \includegraphics[width=1.0\textwidth]{figures/wnlls_global_fitting_results_comparison_v6/populations_HB_comparison.png}
    \caption{重み付き非線形最小二乗法によるHモデルとBモデルの各準位の占有確率. 赤はHモデル, 青はBモデルの結果を示す. }
    \label{fig:wnls_P_HB}
\end{figure}

\textbf{温度依存性による多準位系の重要性}

図\ref{fig:wnls_P_HB}から, 温度上昇に伴う基底状態占有確率の顕著な減少が観察される:
\begin{itemize}
    \item \textbf{低温域}(1.5K--4K): 基底状態占有率 \(p_0 > 94\%\)
    \begin{itemize}
        \item 第一励起状態の占有確率は数\%程度であり, 2準位近似が妥当
        \item 磁気感受率は主に基底状態--第一励起状態間の遷移で決定される
    \end{itemize}
    \item \textbf{中温域}(10K): 基底状態占有率 \(p_0 \approx 69\%\)(Hモデル), \(72\%\)(Bモデル)
    \begin{itemize}
        \item 第一励起状態の占有確率が約20\%に達し, 第二励起状態も約7\%占有される
        \item 複数の準位間遷移が磁気応答に寄与するため, 多準位モデルが必要
    \end{itemize}
    \item \textbf{高温域}(20K--30K): 基底状態占有率 \(p_0 < 45\%\)
    \begin{itemize}
        \item 30K条件では基底状態占有率が約34\%まで低下し, 高励起状態(\(k \geq 3\))の総占有率が30\%を超える
        \item Boltzmann分布により準位間の占有確率が平坦化し, 多数の準位間遷移が磁気スペクトルに寄与
        \item この条件下では, 本研究の8準位モデル(\(S=7/2\))が不可欠である
    \end{itemize}
\end{itemize}

\textbf{HモデルとBモデルの占有確率の差異}

表\ref{tab:P0_results_wnlls}より, Bモデルの方がHモデルよりも系統的に高い基底状態占有率を示している. この差異は, 両モデルのエネルギーギャップの違い(表\ref{tab:energy_gaps})に起因する:
\begin{align*}
p_0 &= \frac{1}{Z} = \frac{1}{\sum_{k=0}^{7} \exp(-E_k/k_B T)} \\
&\propto \exp(\Delta E / k_B T) \quad (\Delta E = E_1 - E_0)
\end{align*}
Bモデルは全磁場条件でHモデルより約2~meV大きなエネルギーギャップを持つため, 基底状態占有率が1--3\%高くなる. ただし, この差異は図\ref{fig:wnls_T_HB}の透過スペクトルにおけるモデル間の差よりも小さく, 占有確率の違いだけでは透過スペクトルのずれを説明できない. 

\textbf{磁場依存性とZeeman効果}

磁場変化系列(1.5K)では, 磁場増加に伴いエネルギーギャップが線形に増加するため(表\ref{tab:energy_gaps}), 基底状態占有率が単調増加する:
\[
\Delta E = g \mu_B B \approx 5.36 \text{ meV/T} \times B
\]
9Tでは \(\Delta E \approx 48\)~meV \(\approx 557\)~Kとなり, 1.5K条件(\(k_B T \approx 0.13\)~meV)では \(\exp(-\Delta E/k_B T) \approx 10^{-160}\) と極めて小さくなるため, 基底状態占有率が99.9\%以上に達する. このような極低温・高磁場条件では, 実質的に基底状態のみが占有される状態が実現している. \\
\begin{table}[htbp]
\centering
\caption{H/Bモデルの基底状態占有確率}
\label{tab:P0_results_wnlls}
\begin{tabular}{lcc}
\hline
\textbf{ラベル} & \textbf{Hモデル} & \textbf{Bモデル}\\
\hline
\multicolumn{3}{@{}l@{}}{\text{温度変化(9.0T)}} \\
4K & 0.945 & 0.957\\
10K & 0.688 & 0.717 \\
20K & 0.445 & 0.471 \\
30K & 0.337 & 0.356 \\
\hline
\multicolumn{3}{@{}l@{}}{\text{磁場変化(1.5K)}} \\
4.2T & 0.973 & 0.980 \\
5.0T & 0.986 & 0.991 \\
6.0T & 0.994 & 0.996 \\
7.0T & 0.998 & 0.999 \\
8.0T & 0.999 & 0.999 \\
9.0T & 1.000 & 1.000 \\
\hline
\end{tabular}
\end{table}
\clearpage

\subsubsection*{磁気感受率}
\label{subsec:chi_wnlls}
重み付き非線形最小二乗法によるHモデルとBモデルの磁気感受率フィッティング結果をそれぞれ図\ref{fig:wnls_chi_H}および図\ref{fig:wnls_chi_B}に示す. \\

\textbf{全体的な再現性}

両モデルともに, 実部\(\mathrm{Re}[\chi]\)・虚部\(\mathrm{Im}[\chi]\)共に図\ref{fig:wnls_T_HB}の実験データ(灰色点)を良好に再現していることが確認できる. 図\ref{fig:wnls_chi_H}と図\ref{fig:wnls_chi_B}を比較すると, 磁気感受率\(\chi(\omega)\)の形状そのものは両モデルで大きな差が見られない. 特に, 磁気共鳴周波数\(250\)GHz付近における以下の特徴的な挙動が両モデルで高精度で捉えられている:
\begin{itemize}
    \item 実部\(\mathrm{Re}[\chi]\)の共鳴周波数を中心とした分散型の急峻な変化
    \item 虚部\(\mathrm{Im}[\chi]\)の共鳴周波数でのピーク構造
    \item 共鳴周波数の磁場・温度依存性
\end{itemize}

\textbf{温度・磁場依存性}

\begin{itemize}
    \item \textbf{温度依存性}(9T, 温度変化系列):温度上昇に伴い, \(\mathrm{Im}[\chi]\)のピーク高さが減少し, 線幅が広がる傾向が両モデルで再現されている. これはBoltzmann分布による高励起状態の熱的占有増加と, 緩和率\(\gamma_k\)の準位依存性を反映している.
    
    \item \textbf{磁場依存性}(1.5K, 磁場変化系列):磁場増加に伴い, Zeemanポラリトンが形成されるまでの過程が両モデルで捉えられている. これはUSC領域に到達するために必要なZeeman分裂の大きさを反映している.
\end{itemize}

\textbf{HモデルとBモデルの微小な差異とその物理的帰結}

磁気感受率のプロット上では両モデルの差異は視覚的に小さいが, 表\ref{tab:fit_stats_H}と表\ref{tab:fit_stats_B}の\(|\chi|_{\rm max}\)を比較すると, わずかながら重要な違いが存在する:

\begin{table}[h]
\centering
\begin{tabular}{lcc}
\hline
\textbf{条件} & \textbf{Hモデル \(|\chi|_{\rm max}\)} & \textbf{Bモデル \(|\chi|_{\rm max}\)} \\
\hline
4K (9T) & 1.10 & 1.17 \\
6T (1.5K) & 1.05 & 1.26 \\
7T--9T (1.5K) & 1.04--1.05 & 1.26--1.27 \\
\hline
\end{tabular}
\end{table}

この\(\chi\)の最大値の差(約0.1--0.2)は小さく見えるが, 透過率の計算において決定的な影響を及ぼす:

\textbf{透磁率を通した非線形増幅効果}

磁気感受率\(\chi\)と比透磁率\(\mu_r\)の関係は, 式\eqref{eq:mu_rH}および式\eqref{mu_rB}で示すように採用する形式によって異なる:
\begin{align*}
\text{Hモデル:} \quad &\mu_r^{(H)} = 1 + \chi \\
\text{Bモデル:} \quad &\mu_r^{(B)} = \frac{1}{1-\chi}
\end{align*}

Bモデルの形式では, \(|\chi| \to 1\)の極限で\(|\mu_r^{(B)}| \to \infty\)となる\textbf{発散的振る舞い}を示す. 実際に数値例を示すと:

\begin{itemize}
    \item \(\chi = 1.05\)の場合:
    \begin{itemize}
        \item Hモデル: \(\mu_r^{(H)} = 2.05\)
        \item Bモデル: \(\mu_r^{(B)} = 1/(1-1.05) = -20.0\)(絶対値:20.0)
    \end{itemize}
    \item \(\chi = 1.27\)の場合(Bモデルの9T条件):
    \begin{itemize}
        \item Hモデル相当値: \(\mu_r = 2.27\)
        \item Bモデル: \(\mu_r^{(B)} = 1/(1-1.27) = -3.7\)(絶対値:3.7)
    \end{itemize}
\end{itemize}

このように, 磁気感受率\(\chi\)のわずか0.2程度の差が, 実効透磁率\(\mu_r\)では約9倍の差(2.27 vs 20.0)として増幅される. この非線形増幅が, 図\ref{fig:wnls_T_HB}で観察される透過スペクトルのモデル間での差異を引き起こす.

表\ref{tab:fit_stats_B}の「不安定\%」列は, \(|\chi| > 1\)となり\(\mu_r^{(B)}\)が負値となる周波数点の割合を示しており, 高磁場条件(6T--9T)で約5.7\%のデータ点が数値的に不安定な領域に入っていることを示す. 一方, Hモデルでは\(\mu_r^{(H)} = 1 + \chi\)という線形関係のため, このような発散は生じない.

\textbf{結論}

磁気感受率\(\chi(\omega)\)そのものは両モデルで視覚的に大きな差がないものの, その最大値のわずかな違い(約10\%)が, 実効透磁率\(\mu_r\)への変換を通して非線形に増幅され, 最終的な透過スペクトル\(T(\omega)\)において顕著な差として現れる. これは, Bモデル形式\(\mu_r = 1/(1-\chi)\)が持つ\(|\chi| \approx 1\)での発散的性質に起因しており, 高磁場・低温条件での不安定性を示唆している.

\begin{figure}
    \centering
    \includegraphics[width=1.0\textwidth]{global_fitting_results_H_v6/chi_distribution_H.png}
    \caption{重み付き非線形最小二乗法によるHモデルの磁気感受率フィッティング結果. 実線は実部, 破線は虚部を示す. }
    \label{fig:wnls_chi_H}
\end{figure}

\begin{figure}
    \centering
    \includegraphics[width=1.0\textwidth]{global_fitting_results_B_v6/chi_distribution_B.png}
    \caption{重み付き非線形最小二乗法によるBモデルの磁気感受率フィッティング結果. 実線は実部, 破線は虚部を示す. }
    \label{fig:wnls_chi_B}
\end{figure}
\clearpage

\subsubsection*{透過スペクトル}
\label{subsec:T_wnlls}
重み付き非線形最小二乗法によるHモデルとBモデルの透過スペクトルフィッティング結果を図\ref{fig:wnls_T_HB}に示す. Hモデルは赤線, Bモデルは青線, 実験データは灰色点で示している. \\

\textbf{全体的なフィッティング品質}

Hモデルは全10条件において実験データのピーク構造を良好に再現している. 一方, Bモデルは高磁場・低温条件で顕著なずれを示す. 各条件での特徴を以下に述べる:

\textbf{温度依存性}(9.0T, 温度変化系列)

\begin{enumerate}
    \item \textbf{4K条件}: 
    \begin{itemize}
        \item Hモデル: 3つの主要ピーク(約0.25, 0.6, 0.85~THz)の位置・高さ・線幅が実験と良好に一致(RMSE = 0.117)
        \item Bモデル: 最初の2つのピークで高周波側へのシフトが観察され, ピーク間の強度バランスも実験と異なる(RMSE = 0.112だが, ピーク位置のずれが大きい)
    \end{itemize}
    \item \textbf{10K--30K条件}:
    \begin{itemize}
        \item 温度上昇に伴い, ピーク高さが減少し線幅が広がる傾向が両モデルで再現されている
        \item これはBoltzmann分布による準位占有の分散化と, 緩和率\(\gamma_k\)による均質広がりの増大を反映
        \item 30K条件では, 多準位系の効果により複数のピークが重なり合い, スペクトルが滑らかになる
        \item 両モデルの差異は高温になるほど小さくなる傾向(表\ref{tab:fit_stats_H}, \ref{tab:fit_stats_B}のRMSEが収束)
    \end{itemize}
\end{enumerate}

\textbf{磁場依存性}(1.5K, 磁場変化系列)

\begin{enumerate}
    \item \textbf{低磁場領域}(4.2T--5.0T):
    \begin{itemize}
        \item 両モデルとも \(R^2 \approx 0.5\) と低く, ピーク構造の再現が不十分
        \item 実験データのS/N比が相対的に低い可能性, または低磁場特有の物理(例:結晶場効果の相対的増大)がモデルに不足している可能性
    \end{itemize}
    \item \textbf{中磁場領域}(6.0T--7.0T):
    \begin{itemize}
        \item Hモデル: 実験データとの一致が向上(\(R^2 \approx 0.7\)--0.75)
        \item Bモデル: Hモデルと同等のフィッティング品質を示す
    \end{itemize}
    \item \textbf{高磁場領域}(8.0T--9.0T):
    \begin{itemize}
        \item Hモデル: ピーク位置が高周波側にシフトする様子を精密に捉え, 3つのピーク構造が明瞭に再現されている(\(R^2 \approx 0.73\)--0.76)
        \item Bモデル: \textbf{致命的な不一致}が観察される:
        \begin{itemize}
            \item 第一ピーク(最低周波数ピーク)が実験より高周波側に約0.02--0.03~THz(20--30~GHz)シフト
            \item ピーク線幅が実験の約1.5倍に広がり, ピーク高さが過小評価される
            \item これは磁気感受率\(\chi(\omega)\)の共鳴周波数ずれ(図\ref{fig:wnls_chi_B})と対応
        \end{itemize}
    \end{itemize}
\end{enumerate}

\textbf{モデル間の差異の物理的解釈}

Bモデルの高磁場での不一致は, 以下の要因が複合的に作用していると考えられる:

\begin{enumerate}
    \item \textbf{結晶場パラメータの過小評価}: Bモデルの\(B_4 = 0.159\)~mKはHモデルの30.0~mKに比べて約200倍小さく, 先行研究値(約2~mK)とも1桁異なる. この過小評価により, Zeeman分裂が支配的な高磁場条件で準位構造が不正確になる.
    
    \item \textbf{磁気感受率の過大評価}: 表\ref{tab:fit_stats_B}に示すように, Bモデルでは\(|\chi|_{\rm max} \gtrsim 1.2\)となり, 実効透磁率\(\mu_r^{(B)} = 1/(1-\chi)\)が不安定領域に入る. これが透過率\(T = |t|^2\)の計算精度を低下させる.
    
    \item \textbf{緩和率パラメータの不整合}: Bモデルでは\(\gamma_1 = 0.145\)~THzとHモデルの0.013~THzに比べて約11倍大きい(表\ref{tab:parameters_results_wnlls}). この大きな緩和率が, 第一励起状態に関与する遷移の線幅を不自然に広げている可能性がある.
\end{enumerate}

以上の解析から, 重み付き非線形最小二乗法に基づくフィッティングでは, \textbf{Hモデルが全条件で実験データを一貫して良好に再現しており, 物理的にも妥当なパラメータセットを与える}ことが示された. Bモデルは低磁場・高温条件では競合的な性能を示すものの, 高磁場・低温条件での系統的なずれにより, Kritzellらの実験データを記述するモデルとしては不適切であると結論付けられる.

\begin{figure}[htbp]
    \centering
    \includegraphics[width=1.0\textwidth]{figures/wnlls_global_fitting_results_comparison_v6/fit_all_spectra_HB_comparison.png}
    \caption{重み付き非線形最小二乗法によるHモデルの透過スペクトルフィッティング結果. 赤線はHモデルのフィッティング結果を示し, 青線はBモデルのフィッティング結果を示し, 点線は実験データを示す. ただし, 実験データとフィッティング結果はフィッティングの便宜上, 各磁場・温度条件でそれぞれ正規化されている.}
    \label{fig:wnls_T_HB}  
\end{figure}

\section{ベイズ推定に基づくパラメータ不確実性評価とモデル比較}
\label{sec:bayes_results}
本節では, 前節\ref{sec:wnls_results}のフィッティング結果を事前情報として利用し, 重み付き尤度に基づくベイズ推定を用いて, 各モデルのパラメータの事後分布を求めた結果を示す. また, 事後分布からサンプリングしたパラメータセットを用いて, 各準位のエネルギー固有値と占有確率, 磁気感受率, 透過スペクトルを算出した結果を提示する. 最後に, PSIS-LOOCVに基づくモデル比較結果も提示する.
\subsection*{パラメータの事前分布}
\label{subsec:prior_distributions}

\subsection*{パラメータの事後分布}
\label{subsec:posterior_distributions}
\subsection*{パラメータの相関}
\label{subsec:parameter_correlation}
\subsection*{エネルギー固有値と占有確率}
\label{subsec:E_P_bayes}
\subsection*{磁気感受率}
\label{subsec:chi_bayes}
\subsection*{透過スペクトル}
\label{subsec:T_bayes}
\subsection*{PSIS-LOOCVに基づくモデル比較結果}
\label{subsec:model_comparison}
\section{結果の物理的考察}
\label{sec:discussion}
