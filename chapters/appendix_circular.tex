\chapter{Faraday配置における磁気感受率テンソルの円偏光基底への変換}
\label{chap:appendix_circular}
Kritzellらの実験では等方性物質であるGGGを用いて, Faraday配置(磁場 \(\vb*{B} \parallel \vb*{k} \parallel z\))で透過スペクトルを測定している.\(^{\cite{Kritzell2024}}\) この配置では, 円偏光基底における磁気感受率テンソル\(\chi_{\pm}\)が重要となる. そこで, 直線偏光基底での磁気感受率テンソル\(\chi_{ij}\)を, 円偏光基底に変換する方法を以下に示す.

\section{問題設定}
Faraday配置(磁場 \(\vb*{B} \parallel \vb*{k} \parallel z\))における, 等方性媒質(または立方晶系)の線形応答を考える. 
直線偏光基底(デカルト座標系 \(\{x, y, z\}\))における磁気感受率テンソル \(\vb*{\chi}_{\text{lin}}\) は, 磁気光学効果により以下の反対称成分を持つ. 

\begin{equation}
    \vb*{\chi}_{\text{lin}} = 
    \begin{pmatrix}
    \chi_{xx} & \chi_{xy} & 0 \\
    -\chi_{xy} & \chi_{xx} & 0 \\
    0 & 0 & \chi_{zz}
    \end{pmatrix}
\end{equation}
ここで, \(\chi_{xx}\) は対角成分, \(\chi_{xy}\) はジャイロトロピックな非対角成分を表す. 

\section{基底変換行列 (Unitary Transformation)}
直線偏光基底から円偏光基底への変換を考える. 
ここでは, 右円偏光 (\(\sigma_+\)) および左円偏光 (\(\sigma_-\)) の基底ベクトルを, 標準的なジョーンズベクトルの定義(時間依存性 \(e^{-i\omega t}\))に従い以下のように定義する. 
\begin{align}
    \hat{\vb*{e}}_+ &= \frac{1}{\sqrt{2}}(\hat{\vb*{x}} - i\hat{\vb*{y}}) \\
    \hat{\vb*{e}}_- &= \frac{1}{\sqrt{2}}(\hat{\vb*{x}} + i\hat{\vb*{y}})
\end{align}
このとき, 基底変換行列(ユニタリ行列)\(U\) は以下のように与えられる. 
\begin{equation}
    U = \frac{1}{\sqrt{2}}
    \begin{pmatrix}
    1 & i & 0 \\
    1 & -i & 0 \\
    0 & 0 & \sqrt{2}
    \end{pmatrix}
\end{equation}
この行列 \(U\) は, 直線偏光基底のベクトル \(\vb*{v}_{\text{lin}}\) を円偏光基底 \(\vb*{v}_{\text{circ}}\) へ変換する(\(\vb*{v}_{\text{circ}} = U \vb*{v}_{\text{lin}}\)). 

\section{円偏光基底でのテンソル表現}
テンソルの変換則 \(\vb*{\chi}_{\text{circ}} = U \vb*{\chi}_{\text{lin}} U^{-1}\) (ユニタリ性より \(U^{-1} = U^{\dagger}\))に従い計算を行う. 

\begin{align}
    \vb*{\chi}_{\text{circ}} &= U \vb*{\chi}_{\text{lin}} U^{\dagger} \notag \\
    &= \frac{1}{2}
    \begin{pmatrix}
    1 & i & 0 \\
    1 & -i & 0 \\
    0 & 0 & \sqrt{2}
    \end{pmatrix}
    \begin{pmatrix}
    \chi_{xx} & \chi_{xy} & 0 \\
    -\chi_{xy} & \chi_{xx} & 0 \\
    0 & 0 & \chi_{zz}
    \end{pmatrix}
    \begin{pmatrix}
    1 & 1 & 0 \\
    -i & i & 0 \\
    0 & 0 & \sqrt{2}
    \end{pmatrix} \notag \\
    &= \frac{1}{2}
    \begin{pmatrix}
    \chi_{xx}-i\chi_{xy} & \chi_{xy}+i\chi_{xx} & 0 \\
    \chi_{xx}+i\chi_{xy} & -\chi_{xy}+i\chi_{xx} & 0 \\
    0 & 0 & 2\chi_{zz}
    \end{pmatrix}
    \begin{pmatrix}
    1 & 1 & 0 \\
    -i & i & 0 \\
    0 & 0 & \sqrt{2}
    \end{pmatrix} \notag \\
    &= 
    \begin{pmatrix}
    \chi_{xx} - i\chi_{xy} & 0 & 0 \\
    0 & \chi_{xx} + i\chi_{xy} & 0 \\
    0 & 0 & \chi_{zz}
    \end{pmatrix}
\end{align}

\section{結論と物理的解釈}
計算の結果, 円偏光基底において磁気感受率テンソルは\textbf{対角化}されることが示された. 
\begin{equation}
    \vb*{\chi}_{\text{circ}} = 
    \begin{pmatrix}
    \chi_+ & 0 & 0 \\
    0 & \chi_- & 0 \\
    0 & 0 & \chi_{zz}
    \end{pmatrix}
\end{equation}
ここで, 対角成分(固有値)は以下のように定義される. 
\begin{equation}
    \chi_{\pm} = \chi_{xx} \mp i\chi_{xy}
\end{equation}
これは, Faraday配置において円偏光がこの系の\textbf{固有モード (Eigenmodes)} であることを意味する. 