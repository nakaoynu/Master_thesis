\chapter{Zeemanポラリトンの形成}
\label{chap:Zeeman_polariton}
\section{固体量子電気力学におけるハイブリッド量子系の展開}

光と物質の相互作用は, 量子力学の黎明期から現代に至るまで物理学の中心的なテーマであり続けている. 近年, 量子情報処理技術の飛躍的な発展に伴い, 異なる物理系をコヒーレントに結合させる「ハイブリッド量子系」の研究が急速に進展している. その中でも, マイクロ波空洞共振器中の電磁場モードと, 固体中のスピン集団を強結合させる試みは, 量子メモリや量子変換器の実現に向けた有力なアプローチとして注目を集めている.\(^{\cite{Zeeman}}\) 

付録\ref{chap:Zeeman_polariton}では, 常磁性絶縁体であるGGG中のガドリニウムイオン\(\text{Gd}^{3+}\)スピン集団と, キャビティ光子との相互作用によって形成される「Zeemanポラリトン」の微視的理論について, その基礎となるハミルトニアンの導出から, 量子光学におけるDicke模型への写像に至るまでを網羅的に詳述する. 

GGGは常磁性体であり, 個々の\(\text{Gd}^{3+}\)イオンは互いに独立した量子エミッターとして振る舞う. この系における光子との結合は, 単純なボゾン間の結合ではなく, 有限のエネルギー準位構造を持つスピン集団とボゾン場との相互作用となる. 
ここでの核心は, GGGにおけるZeemanポラリトン形成が, 量子光学の基本的なモデルである「Dicke模型」の理想的な具現化であることを, 微視的な結晶場理論とハミルトニアン解析に基づいて実証することにある. 特に, 結晶場の低対称性に起因する準位の非調和性や, 温度依存性が示す「スピン的」な飽和挙動に焦点を当て, 微視的パラメータがいかにして巨視的な量子現象を支配するかを解明する. 

\section{エネルギー準位構造と非調和性}

\subsection{ゼロ磁場分裂(ZFS)}
外部磁場がない場合 (\(B=0\)), \(S=7/2\)(半整数スピン)であるため, クラマースの定理により, 8つの基底状態は4つのクラマース二重項に分裂する. \(^{\cite{White2007}}\)

\subsection{強磁場下での準位構造と非調和ラダー}
強磁場(例えば \(B \parallel z\))を印加すると, ゼーマン項が支配的となり, 準位は再配列される. しかし, 結晶場項 \(B_2^0 O_2^0\) の存在により, エネルギー固有値の一次摂動近似は以下のようになる:

\begin{equation}
E_m \approx g \mu_B B_0 m + B_2^0 
\end{equation}

隣接する準位間の遷移エネルギー \(\Delta E_{m \to m+1}\) は:

\begin{equation}
\Delta E_{m \to m+1} = E_{m+1} - E_m \approx g \mu_B B_0 + 3 B_2^0 (2m+1)
\end{equation}

この式は, 遷移エネルギーが量子数 \(m\) に依存する(\(3 B_2^0 (2m+1)\) の項)という\textbf{非調和性}を示している. この「非等間隔なエネルギー梯子」構造こそが, この系をボゾン系と区別し, Dicke模型における「スピン」としての性質を際立たせる要因である. 

\section{Zeemanポラリトンの形成理論}

\subsection{スピン-光子相互作用ハミルトニアン}
キャビティ内の量子化された磁場 \(\hat{\vb*{h}}_{cav}\) とスピン集団との相互作用は, 磁気双極子結合によって以下の形に記述される:
\begin{equation}
H_{int} = - \sum_{j=1}^N \hat{\vb*{\mu}}_j \cdot \hat{\vb*{h}}_{cav}(\vb*{r}_j)
\end{equation}
ここで, 磁気モーメント演算子 \(\hat{\vb*{\mu}}_j = - g \mu_B \hat{\vb*{S}}_j\) であり, \(N\) はスピン数を表す. キャビティ光子の生成・消滅演算子をそれぞれ \(\hat{a}^\dagger, \hat{a}\) とし, 単一モード近似を採用すると, 磁場演算子は以下のように表される: 
\begin{equation}
\hat{\vb*{h}}_{cav}(\vb*{r}) =  \vb*{x_i}(\vb*{r}) \sqrt{\frac{\hbar \omega_c}{2 \mu_0 V_{eff}}}  (\hat{a} + \hat{a}^\dagger)
\end{equation}
\(\vb*{x_i}(\vb*{r})\) はモード関数, \(\omega_c\) はキャビティ共振周波数, \(V_{eff}\) は有効モード体積である. マイクロ波磁場を\(x\)軸方向とすると, 相互作用項は\(S_{x}=(S_{+}+S_{-})/2\)を含み, 以下のように書ける:
\begin{equation}
H_{int} = \sum_{j=1}^N \hbar g_0 (\vb*{r}_j) (S_{j,+} + S_{j,-}) (\hat{a} + \hat{a}^\dagger)
\end{equation}

ここで, \(g_0\) は単一スピンと単一光子の結合定数である. 

\subsection{集団結合増強と超強結合}
\(N\) 個のエミッターがコヒーレントに相互作用する場合, 実効的な結合定数 \(g_{coll}\) は \(\sqrt{N}\) 倍に増強される:

\begin{equation}
g_{coll} = g_0 \sqrt{N}
\end{equation}

GGG結晶中のスピン密度は非常に高く, \(N\) は \(10^{18} \sim 10^{21}\) のオーダーに達する. これにより結合強度は数GHzに達し, 比率 \(\eta = g_{coll} / \omega_c\) が \(0.1\) を超える\textbf{USC}に到達する. 

\subsection{ポラリトン固有状態と真空ラビ分裂}
ハミルトニアンの対角化により得られる「Zeemanポラリトン」は, 共鳴条件において大きな分裂(真空ラビ分裂)を示す:

\begin{equation}
2\Omega_{Rabi} \approx 2 g_{coll} = 2 g_0 \sqrt{N}
\end{equation}

\section{Dicke模型への写像と温度依存性}

\subsection{一般化Dickeハミルトニアン}
\(N\) 個の \(S=7/2\) スピンと単一光子モードの結合系に対するハミルトニアンは以下のように書ける:

\begin{equation}
H_{Dicke} = \hbar \omega_c \hat{a}^\dagger \hat{a} + \hbar \omega_Z \hat{S}_z^{tot} + \frac{\hbar \lambda}{\sqrt{2S N}} (\hat{S}_+^{tot} + \hat{S}_-^{tot}) (\hat{a} + \hat{a}^\dagger)
\end{equation}

\subsection{スピン性の発現と熱脱分極}
常磁性体であるGGGでは, 温度 \(T\) において各スピンがボルツマン分布に従って準位を占有するため, 実効的な結合強度は準位間の\textbf{占有数差}に依存する\(^{\cite{Zeeman}}\). 

\begin{equation}
g_{eff}(T) \propto \sqrt{N \left( P_m(T) - P_{m+1}(T) \right)}
\end{equation}

温度上昇に伴って真空ラビ分裂幅が減少する「熱脱分極」の観測事実は, 系がボゾン(無限準位)ではなく, スピン(有限準位)であることを証明する決定的な証拠である. これは, GGG系がDicke物理を探求するプラットフォームたり得ることを示している. 

また, 磁気双極子結合系であるGGGでは, ゲージ不変性に由来する \(A^2\) 項(反磁性項)が存在しないため, SRPTを阻害するNo-go定理を回避できる純粋なDicke模型を実現できる点も重要である. 