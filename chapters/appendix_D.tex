\chapter{超強結合領域におけるZeemanポラリトンの定量的記述}
\label{chap:zeeman_polariton}

\section{電磁場の量子化}
\section{ベクトルポテンシャルの導入}
クーロンゲージ(\(\nabla \cdot \vb*{A} = 0\))を採用し, 体積 \(V\) のキャビティ内における単一モードの電磁場を考える. 
ベクトルポテンシャル \(\hat{\vb*{A}}(\vb*{r}, t)\) を, キャビティの固有モード関数 \(\vb*{u}(\vb*{r})\) を用いて以下のように量子化する. 
\begin{equation}
    \hat{\vb*{A}}(\vb*{r}) = \sqrt{\frac{\hbar}{2\epsilon_0 \omega_c V}} \left( \hat{a} \vb*{u}(\vb*{r}) + \hat{a}^\dagger \vb*{u}^*(\vb*{r}) \right)
\end{equation}
ここで, \(\hat{a}, \hat{a}^\dagger\) はそれぞれ光子の消滅・生成演算子であり, 交換関係 \([\hat{a}, \hat{a}^\dagger] = 1\) を満たす. \(\omega_c\) はキャビティの共鳴角周波数である. 

\section*{磁場演算子の導出}
磁束密度演算子 \(\hat{\vb*{B}}(\vb*{r})\) は, ベクトルポテンシャルの回転として定義される. 
\begin{equation}
    \hat{\vb*{B}}(\vb*{r}) = \nabla \times \hat{\vb*{A}}(\vb*{r}) = \sqrt{\frac{\hbar}{2\epsilon_0 \omega_c V}} \left( \hat{a} (\nabla \times \vb*{u}(\vb*{r})) + \hat{a}^\dagger (\nabla \times \vb*{u}^*(\vb*{r})) \right)
\end{equation}

\section*{ファラデー配置におけるモード設定}
本実験系(ファラデー配置)に合わせ, 以下の幾何学的条件を設定する. 
\begin{itemize}
    \item 光の伝搬方向(波数ベクトル): \(\vb*{k} \parallel z\)
    \item 静磁場方向: \(\vb*{B}_{\text{DC}} \parallel z\)
    \item キャビティモード: \(z\) 軸方向に定在波を形成する直線偏光モード
\end{itemize}
ここで, 磁場が \(x\) 軸方向に偏光している(\(\vb*{B} \parallel x\))ようなモードを考える. 電磁波の横波性により電場は \(y\) 方向成分を持つため, ベクトルポテンシャルのモード関数を以下のように仮定できる. 
\begin{equation}
    \vb*{u}(\vb*{r}) = u(z) \vb*{e}_y
\end{equation}
これの回転をとると, 
\begin{equation}
    \nabla \times \vb*{u}(\vb*{r}) = \mqty|\vb*{e}_x & \vb*{e}_y & \vb*{e}_z \\ \partial_x & \partial_y & \partial_z \\ 0 & u(z) & 0| = -\frac{d u(z)}{dz} \vb*{e}_x
\end{equation}
となる. スピン集団が配置されている位置(例えば \(z=0\))において磁場が腹(最大振幅)になると仮定し, その局所的な空間微分値を定数 \(C\) (無次元量のオーダー)として扱うと, 
\begin{equation}
    \hat{\vb*{B}} = \sqrt{\frac{\hbar}{2\epsilon_0 \omega_c V}} \frac{k c}{\omega_c} C (\hat{a} + \hat{a}^\dagger) \vb*{e}_x
\end{equation}
のように整理できる. ここで \(\omega_c = c k, c=\frac{1}{\sqrt{\epsilon_0 \mu_0}}\) の関係を用い, すべての定数を真空磁場揺らぎ振幅 \(B_{\text{vac}}\) に押し込めると, 最終的に以下の簡潔な形を得る. 
\begin{equation}
    \hat{\vb*{B}} = B_{\text{vac}} (\hat{a} + \hat{a}^\dagger) \vb*{e}_x, \quad B_{\text{vac}} \equiv C \sqrt{\frac{\hbar \omega_c \mu_0}{2V}}
    \label{eq:B_field}
\end{equation}
これにより, 直線偏光モードの量子化磁場が定義された. 

\section{磁気双極子相互作用}
\section*{ハミルトニアンの定義}
\(N\) 個の局在スピン \(\hat{\vb*{S}}_i\) と量子化磁場 \(\hat{\vb*{B}}\) との相互作用ハミルトニアン \(\hat{\mathcal{H}}_{\text{int}}\) は, ゼーマン・エネルギーの形式(\(-\hat{\boldsymbol{\mu}} \cdot \hat{\vb*{B}}\))で与えられる. 
\begin{equation}
    \hat{\mathcal{H}}_{\text{int}} = - \sum_{i=1}^N \hat{\boldsymbol{\mu}}_i \cdot \hat{\vb*{B}}
\end{equation}
磁気モーメント演算子は \(\hat{\boldsymbol{\mu}}_i = -g_L \mu_B \hat{\vb*{S}}_i\) である(\(g_L\): ランデのg因子, \(\mu_B\): ボーア磁子). 式(\ref{eq:B_field})を代入すると, 磁場が \(x\) 成分のみを持つため, スピンの \(x\) 成分のみが結合する. 
\begin{equation}
    \hat{\mathcal{H}}_{\text{int}} = g_L \mu_B B_{\text{vac}} \sum_{i=1}^N \hat{S}_x^i (\hat{a} + \hat{a}^\dagger)
\end{equation}

\section*{昇降演算子による展開と反回転項}
スピン演算子の \(x\) 成分を昇降演算子 \(\hat{S}_\pm^i = \hat{S}_x^i \pm i \hat{S}_y^i\) を用いて \(\hat{S}_x^i = \frac{1}{2}(\hat{S}_+^i + \hat{S}_-^i)\) と書き換える. 
さらに, 集団スピン演算子 \(\hat{J}_\pm = \sum_i \hat{S}_\pm^i\) を導入する. 
\begin{align}
    \hat{\mathcal{H}}_{\text{int}} &= \frac{g_L \mu_B B_{\text{vac}}}{2} (\hat{J}_+ + \hat{J}_-) (\hat{a} + \hat{a}^\dagger) \\
    &= \frac{g_L \mu_B B_{\text{vac}}}{2} \left[ \underbrace{(\hat{a} \hat{J}_+ + \hat{a}^\dagger \hat{J}_-)}_{\text{回転項}} + \underbrace{(\hat{a} \hat{J}_- + \hat{a}^\dagger \hat{J}_+)}_{\text{反回転項}} \right]
\end{align}
この式変形により, エネルギー保存則を一見破るように見える項(光子生成かつスピン励起 \(\hat{a}^\dagger \hat{J}_+\), およびその逆過程)が自然に出現することがわかる. これらがUSC領域で重要となる反回転項である. 

\section*{Dickeモデルハミルトニアン}
単一スピン当たりの結合定数 \(\hbar g_0 = \frac{1}{2}g_L \mu_B B_{\text{vac}}\) を定義し, 集団増強された結合定数を \(g_{\text{eff}} = g_0 \sqrt{N}\) とおくと, 全ハミルトニアンは以下のように求まる. 
\begin{equation}
    \hat{\mathcal{H}}_{\text{Dicke}} = \hbar \omega_c \hat{a}^\dagger \hat{a} + \hbar \omega_s \hat{J}_z + \frac{\hbar g_{\text{eff}}}{\sqrt{N}} (\hat{a}^\dagger + \hat{a})(\hat{J}_+ + \hat{J}_-) \label{Dicke_Hamiltonian}
\end{equation}
ここで \(\omega_s\) は静磁場によるLarmor周波数である. 
以上より, 直線偏光モードを持つキャビティ場とスピン系の磁気双極子相互作用から, 反回転項を含むDickeハミルトニアンが第一原理的に導出された. 

\section{Holstein-Primakoff変換によるボゾン化}
熱力学的極限(\(N \gg 1\))かつ低励起極限(\(\langle \hat{J}_z \rangle \approx -N/2\))において, 集団スピン演算子に対してHolstein-Primakoff変換を適用し, スピン自由度をボゾン演算子\(\hat{b}, \hat{b}^\dagger\)(マグノン)に写像する. 
\begin{align}
    \hat{J}_+ &\simeq \sqrt{N} \hat{b}^\dagger, \quad \hat{J}_- \simeq \sqrt{N} \hat{b} \\
    \hat{J}_z &= -N/2 + \hat{b}^\dagger \hat{b}
\end{align}
これらをDickeハミルトニアン(式\ref{Dicke_Hamiltonian})に代入し, 定数項を除くと, 以下の2次形式ボゾンハミルトニアン(Hopfieldハミルトニアン)が得られる. 
\begin{equation}
    \hat{\mathcal{H}}_{\text{Hop}} = \hbar \omega_c \hat{a}^\dagger \hat{a} + \hbar \omega_s \hat{b}^\dagger \hat{b} + \hbar g_{\text{eff}} (\hat{a}^\dagger + \hat{a})(\hat{b}^\dagger + \hat{b}) 
    \label{eq:Hopfield}
\end{equation}
式(\ref{eq:Hopfield})は, \(\hat{a}^\dagger \hat{b}^\dagger\)(光子・マグノンの同時生成)および\(\hat{a}\hat{b}\)(同時消滅)を含んでおり, RWAでは記述できない物理現象を示唆している. 

\section{Bogoliubov変換と固有エネルギー}
このハミルトニアン(式\ref{eq:Hopfield})を対角化するために, 一般化されたBogoliubov変換(Hopfield変換)を導入し, 新たな固有演算子(ポラリトン演算子)\(\hat{p}_k\) (\(k=\text{LP, UP}\)) を定義する. 
\begin{equation}
    \hat{p}_k = w_k \hat{a} + x_k \hat{b} + y_k \hat{a}^\dagger + z_k \hat{b}^\dagger
\end{equation}
Heisenbergの運動方程式に基づく行列形式の固有値問題を解くことで, Zeemanポラリトンの分散関係(固有エネルギー \(E_{\pm} = \hbar \Omega_{\pm}\))は以下のように導出される. 
\begin{equation}
    \Omega_{\pm}^2 = \frac{1}{2} \left[ \omega_c^2 + \omega_s^2 \pm \sqrt{(\omega_c^2 - \omega_s^2)^2 + 16 g_{\text{eff}}^2 \omega_c \omega_s} \right]
\end{equation}
この結果は, USC領域において固有振動数が結合強度\(g_{\text{eff}}\)に対して非線形に依存することを示している. 特に\(\Omega_-\)(Lower Polariton)は, Bloch-Siegertシフトを含んだエネルギー準位となる. 
\section{基底状態の量子性}
Dickeモデルにおける真の基底状態(USC真空)\(|G_{\text{USC}}\rangle\) は, ポラリトン消滅演算子に対し \(\hat{p}_k |G_{\text{USC}}\rangle = 0\) を満たす状態として定義される. しかし, 元の粒子演算子(\(\hat{a}, \hat{b}\))に対しては真空ではない. 
すなわち, 
\begin{equation}
    \langle G_{\text{USC}} | \hat{a}^\dagger \hat{a} | G_{\text{USC}} \rangle \neq 0, \quad \langle G_{\text{USC}} | \hat{b}^\dagger \hat{b} | G_{\text{USC}} \rangle \neq 0
\end{equation}
となり, 基底状態においても有限の光子数とマグノン数が存在する(仮想光子の衣をまとう). これは, 系がスクイーズド真空状態にあることを意味し, キャビティ量子電磁力学(Cavity QED)における顕著な量子効果の一つである. 