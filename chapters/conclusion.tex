\chapter{結論}
\label{chap:conclusion}

本手法は, 物理モデルに基づくパラメータ推定において, 重要な特徴量に焦点を当てた解析を可能にし, 従来の最小二乗法に比べて精度と信頼性の向上を実現した. また, ベイズ推定により得られた事後分布は, パラメータ間のトレードオフやモデルの適合度に関する深い洞察を提供し, 物理現象の理解に寄与する. 今後の課題として, 提案手法のさらなる一般化と他の物理系への応用が挙げられる. 例えば, 異なるスペクトル解析や多変量データ解析への適用が考えられる. さらに, 計算効率の向上や大規模データセットへの対応も重要な研究方向である. 本研究の成果は, 量子光学実験におけるデータ解析の新たな基盤を提供し, 今後の研究発展に貢献することが期待される.

本研究の主たる成果は, GGGのTHz磁気光学応答に対するベイズ統計的モデル比較手法を確立し, HモデルとBモデルの予測性能を定量的に評価したことにある. 従来の解析が定性的な議論に留まっていたのに対し, 本研究ではWAICおよびPSIS-LOOCVという厳密な評価指標を用いることで, 「両モデル間に統計的有意差がない」という明確な結論を得た.

この結論自体は, SRPTの発現可能性について決定的な答えを与えるものではないが, 以下の点で重要な貢献をなす:

\begin{enumerate}
    \item 現在の実験精度ではモデル選択が困難であることを定量的に示した
    \item モデル識別に必要な実験条件の方向性を明らかにした
    \item ベイズ推定に基づくパラメータの不確実性評価により, 結晶場パラメータ$B_4, B_6$の決定精度の限界を明らかにした
    \item 将来のより精密な実験設計や解析のための基盤を提供した
\end{enumerate}

SRPTの実現は, 量子技術における革新的な要素となる可能性を秘めている. 本研究で確立した解析手法と得られた知見は, この目標に向けた重要な一歩である.