\chapter{結論}
\label{chap:conclusion}
本研究では, THz帯量子光学実験における透過スペクトル解析のための新たな解析手法を提案し, その有効性を実証した. 重み付き非線形最小二乗法により, 物理的に重要な共鳴ピーク形状に焦点を当てたパラメータ推定を行い, さらにベイズ推定を通じてパラメータの不確実性と相関を定量化した. 最後に, PSIS-LOOCVを用いたモデル比較により, 提案手法の汎化性能を評価した.
本手法は, 物理モデルに基づくパラメータ推定において, 重要な特徴量に焦点を当てた解析を可能にし, 従来の最小二乗法に比べて精度と信頼性の向上を実現した. また, ベイズ推定により得られた事後分布は, パラメータ間のトレードオフやモデルの適合度に関する深い洞察を提供し, 物理現象の理解に寄与する. 今後の課題として, 提案手法のさらなる一般化と他の物理系への応用が挙げられる. 例えば, 異なるスペクトル解析や多変量データ解析への適用が考えられる. さらに, 計算効率の向上や大規模データセットへの対応も重要な研究方向である. 本研究の成果は, 量子光学実験におけるデータ解析の新たな基盤を提供し, 今後の研究発展に貢献することが期待される.