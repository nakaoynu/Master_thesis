\chapter{プログラム実装}
\label{chap:implementation}

\section{重み付き非線形最小二乗法の実装}

\section{ベイズ推定の実装}
\subsection{PyMCにおけるWAICの計算}
\label{subsec:waic_implementation}

PyMC\cite{Abril-Pla2023}およびArviZ\cite{arviz_2019}ライブラリでは,
\texttt{az.waic()}関数によりWAICを計算する.計算手順は以下の通りである:

\begin{enumerate}
    \item SMC/MCMCサンプリングを実行し,サンプル\texttt{trace}(InferenceData形式)を取得
    \item \texttt{pm.compute\_log\_likelihood(trace)}により各データ点の対数尤度を計算
    \item 計算結果を\texttt{trace.log\_likelihood}に格納(形状:chains $\times$ draws $\times$ n\_obs)
    \item \texttt{az.waic(trace)}を実行し,以下を計算:
    \begin{itemize}
        \item 各データ点$i$について,対数尤度$L_i^{(s)}$のサンプル列を取得
        \item $\mathrm{lppd}_i = \log\left(\frac{1}{S}\sum_{s=1}^{S} \exp(L_i^{(s)})\right)$を計算(log-sum-exp安定化)
        \item $p_{\mathrm{WAIC},i} = \mathrm{Var}(L_i^{(s)})$を計算
        \item $\widehat{\mathrm{lppd}} = \sum_{i=1}^{N} \mathrm{lppd}_i$
        \item $\hat{p}_{\mathrm{WAIC}} = \sum_{i=1}^{N} p_{\mathrm{WAIC},i}$
        \item $\widehat{\mathrm{elpd}}_{\mathrm{WAIC}} = \widehat{\mathrm{lppd}} - \hat{p}_{\mathrm{WAIC}}$
        \item $\mathrm{SE} = \sqrt{N \cdot \mathrm{Var}(\mathrm{lppd}_i - p_{\mathrm{WAIC},i})}$
    \end{itemize}
    \item $\widehat{\mathrm{elpd}}_{\mathrm{WAIC}}$, $\hat{p}_{\mathrm{WAIC}}$, $\mathrm{SE}$を取得
\end{enumerate}

本研究での実装コード(Python/PyMC):
\begin{verbatim}
import pymc as pm
import arviz as az

# SMCサンプリング実行
with model:
    trace = pm.sample_smc(draws=10000, chains=16)
    
    # log_likelihoodを計算してtraceに追加
    pm.compute_log_likelihood(trace, model=model)

# WAIC計算
waic_result = az.waic(trace)
print(f"elpd_waic: {waic_result.elpd_waic:.2f}")
print(f"p_waic: {waic_result.p_waic:.2f}")
print(f"SE: {waic_result.se:.2f}")
\end{verbatim}