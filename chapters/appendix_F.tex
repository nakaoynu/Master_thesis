\chapter{相互作用描像・線形応答理論に基づく磁気感受率の導出}
\label{chap:Kubo_formula_derivation}

GGGの磁気光学的応答(透過率や反射率)を計算するためには, 物質の動的磁気感受率\(\chi(\omega)\)を知る必要がある. これは, 外部からの微弱な振動磁場に対する磁化の応答として定義され, 量子統計力学における線形応答理論(Linear Response Theory)を用いて厳密に導出される. 本節では, 久保公式\(^{\cite{Kubo1957}}\)の導出過程を相互作用描像を用いて詳細に示す. 

\section{Liouville方程式と相互作用描像}
\subsection{Liouville-von Neumann方程式の導出}

密度演算子(密度行列)\(\hat{\rho}(t)\)の時間発展方程式であるLiouville-von Neumann方程式を, Schrödinger方程式より導出する. 
ある量子系が, 確率\(p_n\)で純粋状態\(|\psi_n(t)\rangle\)にある混合状態を考える. このとき, 密度演算子は以下のように定義される. 

\begin{equation}
\hat{\rho}(t) = \sum_n p_n |\psi_n(t)\rangle \langle \psi_n(t)|
\end{equation}

ここで, 確率\(p_n\)は時間的に変化しない(\(\dot{p}_n = 0\))と仮定する. これは, 外部環境とのエネルギーや粒子のやり取りによる緩和過程を含まない, 閉じた系におけるユニタリ発展を記述するためである. 
\(\hat{\rho}(t)\)を時間\(t\)で微分すると, Leibniz則より以下のようになる. 

\begin{equation}
\frac{\partial \hat{\rho}(t)}{\partial t} = \sum_n p_n \left[ \left( \frac{\partial}{\partial t} |\psi_n(t)\rangle \right) \langle \psi_n(t)| + |\psi_n(t)\rangle \left( \frac{\partial}{\partial t} \langle \psi_n(t)| \right) \right]
\label{eq:rho_dot}
\end{equation}

一方, 状態ベクトル\(|\psi_n(t)\rangle\)は時間依存Schrödinger方程式に従う. 

\begin{equation}
i\hbar \frac{\partial}{\partial t} |\psi_n(t)\rangle = \hat{\mathcal{H}}(t) |\psi_n(t)\rangle
\end{equation}

この式の両辺を\(i\hbar\)で割り, ケットベクトルの時間微分を得る. 

\begin{equation}
\frac{\partial}{\partial t} |\psi_n(t)\rangle = \frac{1}{i\hbar} \hat{\mathcal{H}}(t) |\psi_n(t)\rangle = -\frac{i}{\hbar} \hat{\mathcal{H}}(t) |\psi_n(t)\rangle
\end{equation}

また, エルミート共役をとることで, ブラベクトルの時間微分が得られる(ハミルトニアンのエルミート性 \(\hat{\mathcal{H}}^\dagger = \hat{\mathcal{H}}\) を用いる). 

\begin{equation}
-i\hbar \frac{\partial}{\partial t} \langle \psi_n(t)| = \langle \psi_n(t)| \hat{\mathcal{H}}(t) \quad \longrightarrow \quad \frac{\partial}{\partial t} \langle \psi_n(t)| = \frac{i}{\hbar} \langle \psi_n(t)| \hat{\mathcal{H}}(t)
\end{equation}

これらを式(\ref{eq:rho_dot})に代入すると, 

\begin{align}
\frac{\partial \hat{\rho}(t)}{\partial t} &= \sum_n p_n \left[ \left( -\frac{i}{\hbar} \hat{\mathcal{H}}(t) |\psi_n(t)\rangle \right) \langle \psi_n(t)| + |\psi_n(t)\rangle \left( \frac{i}{\hbar} \langle \psi_n(t)| \hat{\mathcal{H}}(t) \right) \right] \nonumber \\
&= -\frac{i}{\hbar} \hat{\mathcal{H}}(t) \left( \sum_n p_n |\psi_n(t)\rangle \langle \psi_n(t)| \right) + \frac{i}{\hbar} \left( \sum_n p_n |\psi_n(t)\rangle \langle \psi_n(t)| \right) \hat{\mathcal{H}}(t) \nonumber \\
&= -\frac{i}{\hbar} \left( \hat{\mathcal{H}}(t)\hat{\rho}(t) - \hat{\rho}(t)\hat{\mathcal{H}}(t) \right) \nonumber \\
&= -\frac{i}{\hbar} [\hat{\mathcal{H}}(t), \hat{\rho}(t)]
\end{align}

整理して, 以下のLiouville-von Neumann方程式を得る. 

\begin{equation}
i\hbar \frac{\partial \hat{\rho}(t)}{\partial t} = [\hat{\mathcal{H}}(t), \hat{\rho}(t)]
\end{equation}

この方程式は, 古典力学におけるLiouville方程式の量子力学的対応物であり, 量子状態の確率分布の時間発展を記述する基礎方程式である. \\
\(\hat{\mathcal{H}}(t)\)は, 時間的に変化しない非摂動系のハミルトニアン\(\hat{\mathcal{H}}_0\)と, 時刻\(t_0\)以降に印加される外部摂動\(\hat{\mathcal{H}}'(t)\)の和で表される. 
\begin{equation}
\hat{\mathcal{H}}(t) = \hat{\mathcal{H}}_0 + \hat{\mathcal{H}}'(t)
\end{equation}
外部振動場\(\vb*{h}(t)\)に対するZeemanエネルギーを摂動とすると, \(\hat{\mathcal{H}}'(t) = - \hat{\vb*{M}} \cdot \vb*{h}(t)\)である(\(\hat{\vb*{M}}\)は全磁化演算子). 

ここで, 計算の見通しを良くするために, Schrödinger描像から相互作用描像への変換を行う. 相互作用描像における密度演算子\(\rho_I(t)\)および任意の演算子\(\hat{A}_I(t)\)は以下のように定義される:

\begin{align}
\rho_I(t) &= e^{i\hat{\mathcal{H}}_0 t / \hbar} \rho(t) e^{-i\hat{\mathcal{H}}_0 t / \hbar} \\
\hat{A}_I(t) &= e^{i\hat{\mathcal{H}}_0 t / \hbar} \hat{A} e^{-i\hat{\mathcal{H}}_0 t / \hbar}
\end{align}

Liouville方程式に\(\rho(t) = e^{-i\hat{\mathcal{H}}_0 t / \hbar} \rho_I(t) e^{i\hat{\mathcal{H}}_0 t / \hbar}\)を代入して整理すると, 相互作用描像における時間発展方程式は, 摂動項\(\hat{\mathcal{H}}'_I(t)\)のみを含む形となる:

\begin{equation}
i\hbar \frac{\partial \rho_I(t)}{\partial t} = [\hat{\mathcal{H}}'_I(t), \rho_I(t)]
\end{equation}

\section{摂動展開と線形応答}

系は\(t=-\infty\)において熱平衡状態\(\rho_{\text{eq}} = e^{-\beta \hat{\mathcal{H}}_0} / Z\)にあったと仮定する(\(\beta = 1/k_B T\)). 摂動が微弱であるとして, \(\rho_I(t)\)を\(\hat{\mathcal{H}}'\)について級数展開し, 一次の項までをとる(線形近似). 

\begin{gather}
\rho_I(t) = \rho_0 + \Delta \rho_I(t) \\
i\hbar \frac{\partial \Delta \rho_I(t)}{\partial t} \approx [\hat{\mathcal{H}}'_I(t), \rho_0]
\end{gather}

これを積分形に直すと, 

\begin{equation}
\Delta \rho_I(t) = -\frac{i}{\hbar} \int_{-\infty}^t dt' [\hat{\mathcal{H}}'_I(t'), \rho_0]
\end{equation}

シュレーディンガー描像における密度行列の変化\(\Delta \rho(t)\)は, 逆変換により以下のように得られる:

\begin{equation}
\Delta \rho(t) = e^{-i\hat{\mathcal{H}}_0 t / \hbar} \Delta \rho_I(t) e^{i\hat{\mathcal{H}}_0 t / \hbar} = -\frac{i}{\hbar} \int_{-\infty}^t dt' e^{-i\hat{\mathcal{H}}_0 t / \hbar} [\hat{\mathcal{H}}'_I(t'), \rho_0] e^{i\hat{\mathcal{H}}_0 t / \hbar}
\end{equation}

\section{久保公式(Kubo Formula)の導出}

観測量である磁化\(\hat{M}_\alpha\)(\(\alpha = x, y, z\))の期待値の変化\(\Delta \langle M_\alpha(t) \rangle\)は, トレースを用いて計算される. 

\begin{equation}
\Delta \langle M_\alpha(t) \rangle = \text{Tr}(\Delta \rho(t) \hat{M}_\alpha)
\end{equation}

トレースの巡回不変性(\(\text{Tr}(ABC) = \text{Tr}(BCA)\))を利用して式を変形する. 

\begin{align}
\Delta \langle M_\alpha(t) \rangle &= -\frac{i}{\hbar} \int_{-\infty}^t dt' \text{Tr}\left( [\hat{\mathcal{H}}'_I(t'), \rho_0] e^{i\hat{\mathcal{H}}_0 t / \hbar} \hat{M}_\alpha e^{-i\hat{\mathcal{H}}_0 t / \hbar} \right) \nonumber \\
&= -\frac{i}{\hbar} \int_{-\infty}^t dt' \text{Tr}\left( [\hat{\mathcal{H}}'_I(t'), \rho_0] \hat{M}_{\alpha, I}(t) \right) \nonumber \\
&= -\frac{i}{\hbar} \int_{-\infty}^t dt' \text{Tr}\left( \rho_0 [\hat{M}_{\alpha, I}(t), \hat{\mathcal{H}}'_I(t')] \right) \quad (\because \text{Tr}([A, \rho]B) = \text{Tr}(\rho [B, A]))
\end{align}

外部磁場が\(\vb*{h}(t')\)として\(\hat{M}_\beta\)成分に結合している場合, \(\hat{\mathcal{H}}'_I(t') = - \hat{M}_{\beta, I}(t') h_\beta(t')\)であるから, 

\begin{equation}
\Delta \langle M_\alpha(t) \rangle = \int_{-\infty}^t dt' \frac{i}{\hbar} \langle [\hat{M}_{\alpha, I}(t), \hat{M}_{\beta, I}(t')] \rangle_0 h_\beta(t')
\end{equation}

ここで, \(\langle \dots \rangle_0 = \text{Tr}(\rho_0 \dots)\)は平衡状態でのアンサンブル平均を表す. 応答関数\(\phi_{\alpha\beta}(t-t')\)を以下のように定義すると, これが久保公式である. 

\begin{equation}
\phi_{\alpha\beta}(t-t') = \frac{i}{\hbar} \theta(t-t') \langle [\hat{M}_\alpha(t), \hat{M}_\beta(t')] \rangle_0
\end{equation}

\(\theta(t)\)は因果律を表す階段関数である. 周波数領域での感受率\(\chi_{\alpha\beta}(\omega)\)は, この応答関数のフーリエ変換として与えられる. 

\begin{equation}
\chi_{\alpha\beta}(\omega) = \int_0^\infty dt e^{i\omega t} \phi_{\alpha\beta}(t)
\end{equation}

この式は, \textbf{「外部磁場に対する巨視的な磁化の応答(非平衡過程)は, 熱平衡状態における磁化の量子力学的な揺らぎ(交換関係の相関関数)によって完全に記述される」}という揺動散逸定理の物理的基礎を与えている. 