% chapters/method.tex の中身

\chapter{解析方法}
\label{chap:method} % 後で参照するためにラベルを付ける

本章では, 研究の解析方法について述べる.

\section{低周波領域のデータ解析}
\label{sec:method_low_freq}
低周波領域のデータ解析には, MCMCサンプリングによるベイズ推定を用いて以下のパラメータセットの推定を行う. 具体的には, 各データポイントを一つずつ除外し, 残りのデータでモデルを学習し, 除外したデータポイントに対する予測精度を評価する. 

\section{高周波領域のデータ解析}
\label{sec:method_high_freq}
高周波領域のデータ解析には, ベイズ推定とMCMCを組み合わせて用いる. 具体的には, 観測データに対して事前分布を設定し, MCMCを用いて事後分布からサンプリングを行う. 

\section{全体としてのデータ解析}
\label{sec:method_all_freq}
全体のデータ解析には, 低周波領域と高周波領域のデータを統合して解析を行う. 具体的には, 各領域で得られた知見をもとに, 全体のモデルを構築し, 解析を行う. 

\section{モデル選択}
\label{sec:method_model_selection}
モデル選択には, WAICを用いて複数のモデルの適合度を比較する. WAICが最も低いモデルを選択し, そのモデルを用いて最終的な解析を行う.   