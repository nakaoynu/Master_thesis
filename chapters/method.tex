% chapters/method.tex の中身

\chapter{解析方法}
\label{chap:method} % 後で参照するためにラベルを付ける

本研究では, 取得されたTHz帯透過スペクトルの解析にあたり, 物理モデルに基づくパラメータ推定を行う. 解析は三段階で構成される. 第一に, 重み付き非線形最小二乗法を用いて大域的な最適解近傍を探索し, 第二に, その結果を事前情報として組み込んだベイズ推定を行うことで, パラメータの不確実性と相関を評価する. 第三に, WAIC及びPSISを用いたLOOCVにより, モデルの汎化性能を2つの情報規準に基づき評価する. 以下, 各手法の詳細について述べる.

\section{各準位の緩和係数}
磁気感受率は式\ref{eq:chi_linear}によって計算される. GGGは8準位あるので, 全\(8 \times 7 = 56\)個の遷移ペア( \(n \leftrightarrow n'\) )の寄与を考える必要がある. 計算コストの観点から, 本研究では, 式\ref{eq:gamma_approximation}を仮定し, 表\ref{tab:gamma_transition}で示すように, 56個の遷移の緩和係数を7個の緩和係数で記述する. 以下, この仮定に基づく緩和係数を共有\(\gamma\)モデルと呼称する. 
\begin{equation}
    \gamma (n, n') = \gamma_{\text{min}} (n, n')
    \label{eq:gamma_approximation}
\end{equation}
この物理的意味は, 緩和過程が主に初期状態(低エネルギー側)の特性で決まるという仮定である. 
\begin{table}[htbp]
\centering
\caption{共有\(\gamma\)モデル}
\begin{tabular}{lcccc}
\toprule
\textbf{\(\gamma_{i}\)} & \textbf{遷移パターン} & \textbf{遷移数}\\
\midrule
\(\gamma_0\) & \(|0\rangle \leftrightarrow |1,2,\dots,7\rangle\) & \(7 \times 2 = 14\)\\
\(\gamma_1\) & \(|0\rangle \leftrightarrow |2, \dots,7\rangle\) & \(6 \times 2 = 12\)\\
\(\gamma_2\) & \(|0\rangle \leftrightarrow |3, \dots,7\rangle\) & \(5 \times 2 = 10\)\\
\(\gamma_3\) & \(|0\rangle \leftrightarrow |4,\dots,7\rangle\) & \(4 \times 2 =8\)\\
\(\gamma_4\) & \(|0\rangle \leftrightarrow |5,\dots,7\rangle\) & \(3 \times 2 = 6\)\\
\(\gamma_5\) & \(|0\rangle \leftrightarrow |6,7\rangle\) & \(2 \times 2 = 4\)\\
\(\gamma_6\) & \(|0\rangle \leftrightarrow |7 \rangle\) & \(1 \times 2 = 2\)\\
\bottomrule
\end{tabular}
\label{tab:gamma_transition}
\end{table}

\section{推定するパラメータ群}
\label{sec:parameters_to_estimate}
本解析において推定対象となる主要なパラメータ群とその物理的役割を Table \ref{tab:parameters_1} にまとめる. 

\begin{table}[htbp]
    \centering
    \caption{本解析における推定パラメータ群とその物理的役割}
    \label{tab:parameters_1}
    \renewcommand{\arraystretch}{1.3} % 行の高さを調整
    \begin{tabular}{@{}llp{8.5cm}@{}}
        \toprule
        \textbf{記号} & \textbf{パラメータ名称} & \textbf{物理的役割とモデルへの寄与} \\
        \midrule
        \(g_{J}\) & g因子 & ゼーマンエネルギー \(\muB g_{J} B\) を決定し, スペクトルの共鳴中心周波数を支配する.  \\
        \(B_{4}, B_{6}\) & 結晶場パラメータ & ゼロ磁場分裂や準位の混合を引き起こし, ピークの微細構造(分裂幅)やポラリトンモードに寄与する. \\
        \(\gamma_{i}\) & 緩和係数 & 遷移の寿命 \(\tau = 1/\gamma\) に対応し, スペクトルの線幅(Lorentzian幅)を決定する. コヒーレンスの減衰や散乱過程を反映する.  \\
        \(a_{\vb*{scale}}\) & スケーリング係数 & 実効的なスピン密度や試料充填率に依存する吸収強度の絶対値を補正する係数. \(\chi \propto a_{\vb*{scale}} \cdot g^2\) の関係を持つ.  \\
        \(\varepsilon_{\vb*{bg}}\) & 背景誘電率 & 磁気共鳴以外のバックグラウンド透過率および高次の共振器モードを決定する.  \\
        \bottomrule
    \end{tabular}
\end{table}

\section{重み付き非線形最小二乗法}
\label{sec:wnlls}
本節では, スペクトルフィッティングに用いる重み付き非線形最小二乗法の詳細について述べる.
本研究におけるスペクトルフィッティングでは, モデルパラメータ \(\bm{\theta}\) に対する物理的な制約条件(正値性や上下限)の遵守に加え, スペクトル形状の重要な特徴(共鳴ピークやディップ構造等)を精度よく再現することが求められる. 
均等な重み付けでは, 信号強度の小さい領域やノイズの影響により, 主要な特徴のフィッティング精度が損なわれる可能性がある. そのため, 各データ点の重要度を任意に調整可能な\textbf{重み付き信頼領域反射法}(Weighted Trust Region Reflective algorithm: Weighted-TRF)\(^{\cite{Branch1999}}\)を採用した. 

\subsection{目的関数と制約条件}
観測データベクトルを \(\bm{y}^{\mathrm{obs}}\), モデル関数を \(\bm{f}(\bm{\theta})\) とする. 各データ点 \(i\) に対する重要度を示す重み係数を \(w_i\) とし, これを対角成分に持つ重み行列 \(\vb*{W} = \mathrm{diag}(w_1, \dots, w_N)\) を導入する. 
目的関数 \(S(\bm{\theta})\) は, この重み行列を用いた重み付き残差二乗和として定義される. 
\begin{equation}
    \min_{\bm{\theta}} S(\bm{\theta}) = \frac{1}{2} \left( \bm{y}^{\mathrm{obs}} - \bm{f}(\bm{\theta}) \right)^T \vb*{W} \left( \bm{y}^{\mathrm{obs}} - \bm{f}(\bm{\theta}) \right)
    \label{eq:objective_function}
\end{equation}
ここで, 重み \(w_i\) はスペクトルの特徴的な領域(例えば透過率の変化が急峻な領域)に対して大きな値を設定することで, その領域のフィッティング優先度を高める役割を果たす. 
また, パラメータベクトル \(\bm{\theta}\) は以下の不等式制約を満たすものとする. 
\begin{equation}
    \bm{l} \le \bm{\theta} \le \bm{u}
    \label{eq:boundary_constraints}
\end{equation}
\(\bm{l}, \bm{u}\) はそれぞれパラメータの下限および上限ベクトルである. 

\subsection{信頼領域サブ問題と重み付きスケーリング}
Weighted-TRF法は反復解法であり, 第 \(k\) ステップにおけるパラメータ推定値 \(\bm{\theta}_k\) の近傍(信頼領域)において, 目的関数 \(S(\bm{\theta})\) を二次近似したモデル \(m_k(\bm{p})\) を最小化する探索ベクトル \(\bm{p}\) を決定する. 
\begin{equation}
    m_k(\bm{p}) = S(\bm{\theta}_k) + \vb*{g}_k^T \bm{p} + \frac{1}{2} \bm{p}^T \vb*{H}_k \bm{p}
    \label{eq:quadratic_model}
\end{equation}
ここで, 重み付き勾配ベクトル \(\vb*{g}_k\) およびガウス・ニュートン近似を用いた重み付きヘッセ行列 \(\vb*{H}_k\) は, ヤコビ行列 \(\vb*{J}_k\) を用いて以下のように記述される. 
\begin{equation}
    \vb*{g}_k = -\vb*{J}_k^T \vb*{W} (\bm{y}^{\mathrm{obs}} - \bm{f}(\bm{\theta}_k)), \quad \vb*{H}_k \approx \vb*{J}_k^T \vb*{W} \vb*{J}_k
\end{equation}
重み行列 \(\vb*{W}\) の導入により, 重要度の高いデータ点での残差を優先的に減少させる探索方向が選択される. 

\subsection{Levenberg-Marquardt法との比較}
一般に非線形最小二乗法の標準解法として知られるLevenberg-Marquardt法\(^{\cite{More1978}}\)は, ダンピング係数 \(\lambda\) を用いて最急降下法とガウス・ニュートン法を補間する強力な手法であるが, 本質的には制約なし問題を対象としている. LM法で境界条件を扱う場合, 変数のクリッピングやペナルティ関数などのヒューリスティックな処理が必要となり, 数値的な不安定性を招く恐れがある. 

\subsection{フィッティング手法とプログラムの全体フロー}    
\label{subsec:wnlls_method}
まず, Kritzellらの実験データを読み込み, 磁場・温度ごとにデータを分割する. 次に, 各モデル(H形式, B形式)について, 手作業でフィッティングした際の知見を活かした初期パラメータを設定し, 重み付き非線形最小二乗法を適用してフィッティングを行う. フィッティングには\texttt{Scipy}ライブラリの\texttt{optimize}モジュールに含まれる\texttt{least\_squares}関数を使用し, 各パラメータの最適値と共分散行列を取得する. 最後に, フィッティング結果を用いて各準位のエネルギー固有値と占有確率, 磁気感受率, 透過スペクトルを算出し, 結果を保存する. 重み付け方法としては, ピーク検出アルゴリズムにより特定された共鳴中心周波数および半値全幅(FWHM)内の領域に対して大きな値を与え, それ以外の領域には小さな値を与える. ただし, ポラリトン形成領域のフィッティングが困難だった為, ポラリトン形成領域の方が高次共振器モードよりも高く設定した.
図\ref{fig:flowchart_wnlls}に, 重み付き非線形最小二乗法の全体フローチャートを示す. 初期条件は表\ref{tab:parameters_initial_wnlls}, フィッティング条件は表\ref{tab:stepwise_fitting_parameters}の通りである.サンプリング効率を向上させるため,各パラメータをスケーリングして最適化空間で同程度の変動幅を持つようにした:
\begin{equation}
    \theta_{\mathrm{scaled}} = \theta_{\mathrm{physical}} \times s_\theta
\end{equation}
スケーリング係数\(s_\theta\)を表\ref{tab:scaling}に示す.

\begin{table}[htbp]
\centering
\caption{重み付き非線形最小二乗法の全パラメータの初期条件}
\begin{tabular}{lcccc}
\toprule
\textbf{カテゴリ} & \textbf{パラメータ} & \textbf{初期値} & \textbf{範囲}\\
\midrule
\multirow{5}{*}{Global (5個)} 
 & \(g_{J}\) & 1.95 & [1.5, 2.8]\\
 & \(a\) & 1.0 & [0.1, 5.0]\\
 & \(B_4\) & 2.02 mK & [0.1, 30] mK\\
 & \(B_6\) & \(-0.012\) mK & [-1.0, 1.0] mK\\
 & \(\varepsilon_{\text{bg}}\) & 14.4 & [13.0, 16.0]\\
\midrule
\multirow{7}{*}{Shared \(\gamma\) (7個)} 
 & \(\gamma_0\) & 0.10 THz & [0.01, 0.5] THz\\
 & \(\gamma_1\) & 0.15 THz & [0.01, 0.5] THz\\
 & \(\gamma_2\) & 0.12 THz & [0.01, 0.5] THz\\
 & \(\gamma_3\) & 0.11 THz & [0.01, 0.5] THz\\
 & \(\gamma_4\) & 0.14 THz & [0.01, 0.5] THz\\
 & \(\gamma_5\) & 0.13 THz & [0.01, 0.5] THz\\
 & \(\gamma_6\) & 0.16 THz & [0.01, 0.5] THz\\
\bottomrule
\end{tabular}
\label{tab:parameters_initial_wnlls}
\end{table}

\begin{table}[htbp]
    \centering
    \caption{\textbf{重み付き非線形最小二乗法の段階的フィッティング条件}}
    \label{tab:stepwise_fitting_parameters}
    \renewcommand{\arraystretch}{1.2} % 行間調整
    \begin{tabular}{@{}lccc@{}}
        \toprule
        \multicolumn{4}{c}{\textbf{段階的最適化設定}} \\
        \cmidrule(lr){1-4}
        \textbf{パラメータ項目} & \textbf{Stage 1} & \textbf{Stage 2} & \textbf{Stage 3} \\
        & (粗探索) & (中間精緻化) & (微調整) \\
        \midrule
        最大反復回数 (\(N_{\text{max}}\)) & 5,000 & 15,000 & 30,000 \\
        関数値収束許容誤差 (\texttt{ftol}) & \(1.0 \times 10^{-5}\) & \(1.0 \times 10^{-7}\) & \(1.0 \times 10^{-9}\) \\
        パラメータ収束許容誤差 (\texttt{xtol}) & \(1.0 \times 10^{-5}\) & \(1.0 \times 10^{-7}\) & \(1.0 \times 10^{-9}\) \\
        \midrule
        \multicolumn{4}{c}{\textbf{共通設定}} \\
        \cmidrule(lr){1-4}
        \multicolumn{2}{l}{最適化アルゴリズム} & \multicolumn{2}{l}{TRF法} \\
        \multicolumn{2}{l}{ポラリトンモードの重み (\(w_{\text{pol}}^{\text{WNLLS}}\))} & \multicolumn{2}{l}{1.5} \\
        \multicolumn{2}{l}{共振器モードの重み (\(w_{\text{cav}}^{\text{WNLLS}}\))} & \multicolumn{2}{l}{1.0} \\
        \multicolumn{2}{l}{背景・ノイズ領域の重み (\(w_{\text{bg}}^{\text{WNLLS}}\))} & \multicolumn{2}{l}{0.01} \\
        \bottomrule
    \end{tabular}
\end{table}

\begin{table}[htbp]
\centering
\caption{パラメータスケーリング係数}
\label{tab:scaling}
\begin{tabular}{lcc}
\toprule
パラメータ & スケーリング係数 \(s_\theta\) & スケール後の範囲 \\
\midrule
\(g_{J}\) & 38.0 & \([57, 106]\) \\
\(a\) & 10.2 & \([1.0, 102]\) \\
\(B_4\) & 1672.0 & \([0.017, 83.6]\) \\
\(B_6\) & 25000.0 & \([-50, 50]\) \\
\(\varepsilon_{\mathrm{bg}}\) & 17.0 & \([221, 272]\) \\
\(\gamma\) & 100.0 & \([0.5, 50]\) \\
\bottomrule
\end{tabular}
\end{table}

\begin{figure}[htbp]
\centering
\begin{tikzpicture}[
    node distance=1.2cm,
    startstop/.style={rectangle, rounded corners, minimum width=3cm, minimum height=0.8cm, text centered, draw=black, fill=red!30},
    process/.style={rectangle, minimum width=3cm, minimum height=0.8cm, text centered, draw=black, fill=orange!30},
    io/.style={trapezium, trapezium left angle=70, trapezium right angle=110, minimum width=3cm, minimum height=0.8cm, text centered, draw=black, fill=blue!30},
    decision/.style={diamond, minimum width=2cm, minimum height=0.8cm, text centered, draw=black, fill=green!30},
    arrow/.style={thick,->,>=stealth}
]

% ノード
\node (start) [startstop] {開始};
\node (load) [io, below of=start] {データロード (10セット)};
\node (detect) [process, below of=load] {ポラリトン検出・重み付け};
\node (init) [process, below of=detect] {初期値・境界値設定};
\node (opt1) [process, below of=init] {Stage 1: 粗探索};
\node (opt2) [process, below of=opt1] {Stage 2: 中間精緻化};
\node (opt3) [process, below of=opt2] {Stage 3: 微調整};
\node (analyze) [process, below of=opt3] {結果解析・診断};
\node (plot) [io, below of=analyze] {プロット生成・保存};
\node (end) [startstop, below of=plot] {終了};

% 矢印
\draw [arrow] (start) -- (load);
\draw [arrow] (load) -- (detect);
\draw [arrow] (detect) -- (init);
\draw [arrow] (init) -- (opt1);
\draw [arrow] (opt1) -- (opt2);
\draw [arrow] (opt2) -- (opt3);
\draw [arrow] (opt3) -- (analyze);
\draw [arrow] (analyze) -- (plot);
\draw [arrow] (plot) -- (end);

% モデル形式のループ
\node[draw, dashed, fit=(opt1)(opt2)(opt3)(analyze), inner sep=0.3cm, label=right:{\small H形式・B形式で繰り返し}] {};

\end{tikzpicture}
\caption{プログラムの処理フロー}
\label{fig:flowchart_wnlls}
\end{figure}
\clearpage

\section{重み付き尤度に基づくベイズ推定}
\label{sec:weighted_bayesian}
点推定である最小二乗法に対し, パラメータの事後確率分布 \(p(\vb*{\theta} | D)\) を求めることで, 推定値の不確実性とパラメータ間の相関を定量化する. 

\subsection{尤度関数の設計}
\label{subsec:likelihood}

観測透過率\(T_{\mathrm{obs},i}\)とモデル予測透過率\(T_{\mathrm{model},i}(\bm{\theta})\)の残差分布として,
外れ値に頑健なStudent-t分布を採用した\(^{\cite{Lange1989}}\):
\begin{equation}
    T_{\mathrm{obs},i} \sim \mathrm{StudentT}\left(\nu, T_{\mathrm{model},i}(\bm{\theta}), \sigma_{\mathrm{eff},i}\right)
    \label{eq:studentt_likelihood}
\end{equation}
ここで,\(\nu = 4\)は自由度であり,正規分布よりも裾の重い分布を実現する.

\subsubsection{Student-t分布の定義と性質}

Student-t分布は1908年にW. S. Gosset(ペンネーム「Student」)によって導入された確率分布である\(^{\cite{Student1908}}\).
自由度\(\nu\),位置パラメータ\(\mu\),スケールパラメータ\(\sigma\)を持つStudent-t分布の確率密度関数は次式で定義される\(^{\cite{Gelman2013}}\):
\begin{equation}
    p(x|\nu, \mu, \sigma) = \frac{\Gamma\left(\frac{\nu+1}{2}\right)}{\Gamma\left(\frac{\nu}{2}\right)\sqrt{\pi\nu}\sigma}
    \left(1 + \frac{1}{\nu}\left(\frac{x-\mu}{\sigma}\right)^2\right)^{-\frac{\nu+1}{2}}
    \label{eq:studentt_pdf}
\end{equation}
ここで\(\Gamma(\cdot)\)はガンマ関数である.

Student-t分布は自由度\(\nu\)によって裾の重さが制御される:
\begin{itemize}
    \item \(\nu = 1\):\textbf{コーシー分布}に一致.平均・分散が定義されない極端に裾の重い分布.
    \item \(\nu = 4\):本研究で採用.平均が存在し(\(\nu > 1\)),分散も有限(\(\nu > 2\))だが,正規分布より裾が重い.
    \item \(\nu = 10\):正規分布に近づくが,まだ裾が重い.
    \item \(\nu \to \infty\):\textbf{正規分布}に収束.
\end{itemize}

自由度\(\nu\)が小さいほど裾が重くなり,外れ値の影響を受けにくくなる.
一方,\(\nu\)が小さすぎると推定効率が低下する.
Langeら\(^{\cite{Lange1989}}\)の研究に基づき,頑健性と効率性のトレードオフを考慮して\(\nu = 4\)を採用した.

Student-t分布の主要な統計量は以下の通りである:
\begin{align}
    \text{期待値} &: \quad \mathbb{E}[X] = \mu \quad (\nu > 1) \\
    \text{分散} &: \quad \mathrm{Var}[X] = \sigma^2 \frac{\nu}{\nu - 2} \quad (\nu > 2) \\
    \text{尖度} &: \quad \kappa = \frac{6}{\nu - 4} + 3 \quad (\nu > 4)
\end{align}

\(\nu = 4\)の場合,分散は\(2\sigma^2\)となり,正規分布の2倍の広がりを持つ.
また尖度は定義されない(\(\nu > 4\)が必要)が,これは裾の重さを反映している.

有効標準偏差\(\sigma_{\mathrm{eff},i}\)は重み付き誤差として次式で定義される:
\begin{equation}
    \sigma_{\mathrm{eff},i} = \frac{\sigma_0}{\sqrt{w_i}}
    \label{eq:sigma_eff}
\end{equation}
ここで\(\sigma_0 = 0.01\)は基準標準偏差,\(w_i\)は各データ点の重みである.

重みは物理的重要性に基づいて表\ref{tab:bayes_weights}のように設定した:
\begin{table}[htbp]
    \centering
    \caption{ベイズ推定における周波数領域ごとの重み付け定義}
    \label{tab:bayes_weights}
    \begin{tabular}{llcl}
        \toprule
        領域区分 & 周波数帯域 (\(f\)) & 重み (\(w^{\text{bayes}}\)) & 物理的解釈・備考 \\
        \midrule
        ポラリトン領域 & \(< 0.3615~\text{THz}\) & \(2.0\) & \begin{tabular}[t]{@{}l@{}}磁気ポラリトン形成の核心領域 \\ \footnotesize{(注: WNLLS値 \(1.5\) とは異なる値を採用)}\end{tabular} \\
        \addlinespace
        共振器モード領域 & \(> 0.45~\text{THz}\) & \(1.0\) & 高次Fabry-Pérot干渉モード \\
        \addlinespace
        その他の領域 & その他 & \(0.01\) & 背景領域(軽視) \\
        \bottomrule
    \end{tabular}
\end{table}

\subsection{サンプリング手法}
事後分布からのサンプリングには, 効率的なサンプリングが可能なSMC法を採用する. 

\subsection{ベイズ階層モデルの構築}
\label{sec:hierarchical_model}
\subsubsection{階層モデルの必要性}

7つの緩和係数パラメータ\(\gamma_1, \ldots, \gamma_7\)は個別に推定すると識別不能性の問題が生じる.
すなわち,異なる\(\gamma_i\)の組み合わせが同程度の尤度を与え,事後分布が多峰性を示す.
この問題を解決するため,\(\gamma_i\)を共通のハイパーパラメータから生成される階層モデルを採用した.

\subsubsection{Non-centered Parameterizationの導入}

階層モデルにおける「漏斗問題」を回避するため,Non-centered Parameterizationを採用した.

\textbf{Centered版}(従来法,収束性不良):
\begin{equation}
    \gamma_i \sim \mathcal{N}(\mu_\gamma, \sigma_\gamma)
\end{equation}

\textbf{Non-centered版}(本研究):
\begin{align}
    z_i &\sim \mathcal{N}(0, 1) \\
    \log\gamma_i &= \log\mu_\gamma + \log\sigma_\gamma \cdot z_i
    \label{eq:non_centered}
\end{align}

Non-centered版では標準正規変数\(z_i\)がハイパーパラメータと独立であるため,SMCサンプラーが効率的に事後分布を探索できる.

\subsubsection{事前分布の設定}
\label{subsec:priors}

各パラメータの事前分布は物理的制約に基づいて設定した.表\ref{tab:priors}に一覧を示す.ただし,表\ref{tab:scaling}と同様のスケーリング係数を適用し, 最適化空間で同程度の変動幅を持つようにした.また, HモデルとBモデルの事前分布図をそれぞれ図\ref{fig:prior_H},図\ref{fig:prior_B}にプロットする. また, HモデルとBモデルの事前分布図をそれぞれ図\ref{fig:prior_H},図\ref{fig:prior_B}にプロットする. 

\begin{table}[htbp]
    \centering
    \caption{ベイズモデルの事前分布設定}
    \label{tab:priors}
    \begin{tabular}{llcl}
        \toprule
        パラメータ & 分布型 & パラメータ & 物理的根拠 \\
        \midrule
        \(g_{J}\) & TruncatedNormal & \(\mu=2.0\), \(\sigma=0.05\), \([1.5, 2.8]\) & Gd\(^{3+}\)理論値 \\
        \addlinespace
        \(a\) & HalfNormal & \(\sigma=2.0\), \([0.1, 10.0]\) & 低値優先,正値制約 \\
        \addlinespace
        \(B_4\) & LogNormal & \(\mu_{\log}=\log(2\mathrm{mK})\), \(\sigma_{\log}=1.2\) & 正値制約,mKオーダー \\
        \addlinespace
        \(B_6\) & Normal & \(\mu=0\), \(\sigma=0.5\mathrm{mK}\), \([-2, +2]\mathrm{mK}\) & ゼロ中心対称 \\
        \addlinespace
        \(\varepsilon_{\mathrm{bg}}\) & TruncatedNormal & \(\mu=14.0\), \(\sigma=0.3\), \([13, 16]\) & 実験値\(^{\cite{Kritzell2024}}\)参照 \\
        \addlinespace
        \(\log\mu_\gamma\) & Normal & \(\mu=\log(0.074)\), \(\sigma=0.3\) & 最適化結果\footnotesize{(詳細は\ref{sec:wnlls_results}節.)} \\
        \addlinespace
        \(\log\sigma_\gamma\) & HalfNormal & \(\sigma=0.3\) & 正値制約 \\
        \addlinespace
        \(z_i\) (\(i=1,\ldots,7\)) & Normal & \(\mu=0\), \(\sigma=1\) & Non-centered基底 \\
        \bottomrule
    \end{tabular}
\end{table}

\begin{figure}[htbp]
    \centering
    \includegraphics[width=1.0\textwidth]{prior_distributions_H.png}
    \caption{H形式モデルの事前分布. 赤点線は重み付き非線形最小二乗法でのフィッティング値を示す.}
    \label{fig:prior_H}
\end{figure}

\begin{figure}[htbp]
    \centering
    \includegraphics[width=1.0\textwidth]{prior_distributions_B.png}
    \caption{B形式モデルの事前分布. 赤点線は重み付き非線形最小二乗法でのフィッティング値を示す.}
    \label{fig:prior_B}
\end{figure}

\subsection{プログラムの全体フロー}
図\ref{fig:flowchart_bayesian}にて, ベイズ推定を行うプログラムコードの全体フローチャートを示す. 
\begin{figure}[htbp]
\centering
\begin{tikzpicture}[
    node distance=1cm,
    startstop/.style={rectangle, rounded corners, minimum width=3cm, minimum height=0.7cm, 
                      text centered, draw=black, fill=red!30},
    process/.style={rectangle, minimum width=3.5cm, minimum height=0.7cm, 
                    text centered, draw=black, fill=orange!30},
    io/.style={trapezium, trapezium left angle=70, trapezium right angle=110, 
               minimum width=3cm, minimum height=0.7cm, text centered, draw=black, fill=blue!30},
    decision/.style={diamond, minimum width=2cm, minimum height=0.8cm, 
                     text centered, draw=black, fill=green!30},
    arrow/.style={thick,->,>=stealth}
]

\node (start) [startstop] {開始};
\node (loadv6) [io, below=of start] {v6最適化結果読み込み (H/B)};
\node (loaddata) [io, below=of loadv6] {実験データ読み込み (10セット)};
\node (weight) [process, below=of loaddata] {重み配列生成};

\node (modelH) [process, below left=1cm and 0.5cm of weight] {H形式モデル構築};
\node (modelB) [process, below right=1cm and 0.5cm of weight] {B形式モデル構築};

\node (smcH) [process, below=of modelH] {SMCサンプリング (H)};
\node (smcB) [process, below=of modelB] {SMCサンプリング (B)};

\node (save) [io, below=1.5cm of $(smcH)!0.5!(smcB)$] {結果保存 (trace, summary)};
\node (plot) [process, below=of save] {比較プロット生成};
\node (end) [startstop, below=of plot] {終了};

\draw [arrow] (start) -- (loadv6);
\draw [arrow] (loadv6) -- (loaddata);
\draw [arrow] (loaddata) -- (weight);
\draw [arrow] (weight) -| (modelH);
\draw [arrow] (weight) -| (modelB);
\draw [arrow] (modelH) -- (smcH);
\draw [arrow] (modelB) -- (smcB);
\draw [arrow] (smcH) |- (save);
\draw [arrow] (smcB) |- (save);
\draw [arrow] (save) -- (plot);
\draw [arrow] (plot) -- (end);

\end{tikzpicture}
\caption{プログラム全体のフローチャート}
\label{fig:flowchart_bayesian}
\end{figure}

\section{モデル比較と評価}
ベイズ推定により得られた事後分布を用いて, H形式モデルとB形式モデルの適合度を比較・評価する.
本研究では, WAICとPSIS-LOO-CVの両方を採用した. それぞれ表\ref{tab:model_comparison}に示す特徴があり,補完的に利用することでモデル選択の信頼性を高めることができる.
WAICのみ悪い場合は事後分布の形状に問題がある可能性があり, PSIS-LOO-CVのみ悪い場合は外れ値の影響が大きい可能性があることが示唆される.
\begin{table}[htbp]
\centering
\caption{モデル比較指標の特徴}
    \begin{tabular}{lcc}
        \toprule
        指標 & WAIC & PSIS-LOO-CV \\
        \midrule
        計算方法 & 事後分布の対数尤度の情報量基準 & LOOCVの厳密近似, Pareto \(k\)診断で評価 \\
        \addlinespace
        長所 & 計算が高速で容易 & 外れ値に頑健,過学習を抑制 \\
        \addlinespace
        短所 & 外れ値に敏感,過学習のリスク & 計算コストが高い \\
        \bottomrule
    \end{tabular}
    \label{tab:model_comparison}
\end{table}