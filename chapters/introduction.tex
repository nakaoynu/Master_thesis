% chapters/introduction.tex の中身

\chapter{序論}
\label{chap:introduction} % 後で参照するためにラベルを付ける

本章では、研究の背景と目的について述べる。
近年の〇〇分野では、△△が重要な課題となっている

\section{研究背景:物質と光の強結合}
\label{sec:intro_background}
物質と光の相互作用は、現代物理学における最も根源的かつ重要なテーマの一つである。特に、原子や電子スピンなどの量子系を光共振器(キャビティ)内に配置し、両者の相互作用を極限まで増強させた「キャビティ量子電磁力学(Cavity QED)」の分野は、量子情報技術や新奇物性発現の舞台として精力的に研究が行われている。

相互作用の強さは結合エネルギー${g}$で特徴づけられ、これが量子系と光子の損失率$(\gamma,\kappa)$を上回る領域は強結合領域と呼ばれる。近年では、結合エネルギーgが量子系の遷移エネルギー${\omega_{0}}$に匹敵する超強結合領域 (Ultra-Strong Coupling: USC) が実現され、非摂動的な相互作用に起因する非古典的な基底状態の形成など、新たな物理現象が観測されている。


\section{超放射相転移}
\label{sec:intro_SRPT}
物質と光の相互作用をさらに強め、多数の量子系を集団化させた際に現れる究極的な現象が超放射相転移 (Superradiant Phase Transition: SRPT) である [引用: Dicke]。これは、物質-光結合の強さが臨界値を超えると、系が自発的に巨視的な数の光子を放出して秩序化し、基底状態の性質が質的に変化する量子相転移である。

SRPTは、多体物理学と量子光学が交差する魅力的な現象であるが、その実現には極めて強い相互作用が求められるため、実験的な検証は長年の課題であった。


\section{研究対象物質:ガドリニウム・ガリウム・ガーネット (GGG)}
\label{sec:intro_GGG}
本研究で着目するのは、希土類磁性体の一種であるガドリニウム・ガリウム・ガーネット (Gd$_3Ga5O{12}$; GGG) である。GGG中の磁性イオンであるGd$^{3+}$は、S=7/2という大きなスピンを持ち、スピン間の相互作用が弱い理想的な常磁性体として知られる。

特に、その電子スピン共鳴(EPR)はテラヘルツ(THz)帯に存在し、高密度のスピン集団とTHz光を結合させることで、超強結合領域、さらにはSRPTの実現可能性を秘めた有望な物質系として注目されている。
\section{先行研究と本研究の課題}
\label{sec:intro_previous_work}
GGGのTHz帯における磁気光学応答スペクトルは、結晶場とゼーマン効果により分裂した複数の吸収ピークを持つ複雑な形状を示す。このスペクトルを理論的に記述するため、先行研究では主に二つのモデルが提案されてきた。

H形式: 磁化が外部から印加された磁場に直接応答すると考えるモデル。

B形式: 磁化が、物質自身の磁化による内部磁場(局所場)を含む有効的な磁場に応答すると考えるモデル。

これら二つのモデルは、特にスピン密度が高い系において、予測される物理現象(特にSRPTの臨界条件)に質的な違いをもたらす。しかし、どちらのモデルがGGGの磁気応答をより適切に記述するのか、定量的な評価は十分になされていなかった。

\section{本研究の目的と構成}
本研究の目的は、GGGのTHz透過スペクトルの実験データを基に、H形式およびB形式の妥当性を、現代的な統計科学の手法であるベイズ推定を用いて定量的に比較・評価することである。

ベイズ推定を用いることで、各モデルパラメータの最適値だけでなく、その不確かさやモデル全体の確からしさ(モデルエビデンス)を評価することが可能となる。特に、モデルの予測性能を客観的に評価する**LOO-CV(Leave-One-Out Cross-Validation)**と情報量規準を導入し、どちらの物理モデルが観測事実をより良く説明するかを明らかにする。

本論文の構成は以下の通りである。第2章では、GGGの磁気感受率を導出するための量子力学的な枠組みと、透過スペクトルの計算手法、そしてモデル評価に用いるベイズ推定の理論を述べる。第3章では、実際に行った解析手法と結果について詳述する。第4章では、得られた結果を物理的に考察し、結論と今後の展望を述べる。