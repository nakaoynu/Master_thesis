% chapters/introduction.tex の中身

\chapter{序論}
\label{chap:introduction} % 後で参照するためにラベルを付ける

本章では, 研究の背景と目的について述べる. 
光と物質の相互作用は量子光学の根幹をなすテーマであり, レーザー, 原子時計, そして現代の量子情報技術分野の基礎を形成している. 

\section{物質と光の強結合}
\label{sec:intro_background}
原子や電子スピンなどの量子系を光共振器内に配置し, 両者の相互作用を増強させた「共振器量子電磁力学(Cavity QED)」の分野は, 量子情報技術や新奇物性発現の舞台として精力的に研究が行われている. 
光と物質の相互作用の大きさは, 結合定数 \(g\) によって特徴付けられる. 本論文においては, Kritzellらと同様の定義を用いており, 結合定数\(g\)の数式(\ref{eq:Coupling})を以下に示す.\(^{\cite{Kritzell2024}}\)
\begin{equation}
    g = \frac{g_{J} \mu_{B}}{2 \hbar} \sqrt{\mu_{0} N S \hbar \omega_{\text{cav}}}
    \label{eq:Coupling}
\end{equation}
ただし, \(g_{J}\)はランデのg因子, \(\mu_{B}\)はボーア磁子, \(\hbar\)は換算プランク定数, \(\omega_{\text{cav}}\)は共振器の共鳴周波数, \(\mu_{0}\)は真空の透磁率, \(N\)はスピン数密度, \(S\)はスピン量子数を表す.\\
また, 正規化結合定数\(\eta\)は以下のように定義される.
\begin{equation}
    \eta = \frac{g}{\omega_{\text{cav}}}
    \label{eq:normalized_coupling}
\end{equation}
近年のナノテクノロジーや超伝導回路技術の進展により, 正規化結合定数\(\eta \approx 0.10\)に達する「超強結合(Ultra-Strong Coupling: USC)領域」, さらには \(\eta \geq 1\) となる「深強結合(Deep Strong Coupling: DSC)領域」へのアクセスが可能となった.これらの領域では, 従来の摂動論的アプローチや回転波近似(Rotating Wave Approximation: RWA)が破綻し, 光と物質の境界が曖昧な混成状態である準粒子(ポラリトン)が形成される\(^{\cite{Ciuti2005, FornDiaz2019, Kockum2019}}\). 数学的には, ポラリトン状態は光子と物質励起の生成・消滅演算子の線形結合として表される. 光子の消滅演算子を\(\hat{a}\), 物質励起の消滅演算子を\(\hat{b}\)とすると, 両者の相互作用を含むハミルトニアン\(\hat{\mathcal{H}}\)は, USC領域において以下のモデルで記述される.\(^{\cite{Hopfield1958}}\)
\begin{equation}
    \hat{\mathcal{H}} = \hbar \omega_{\text{cav}} \hat{a}^\dagger \hat{a} + \hbar \omega_{\text{ex}} \hat{b}^\dagger \hat{b} + \hbar g (\hat{a} + \hat{a}^\dagger)(\hat{b} + \hat{b}^\dagger)
    \label{eq:Hopfield_Hamiltonian}
\end{equation}
ここで, \(g\)は光子と物質励起の結合定数を表す. このハミルトニアンを対角化するために, 新たな準粒子演算子(ポラリトン演算子)\(\hat{p}_{k}\)を導入する.
\begin{equation}
    \hat{p}_{k} = w_{k} \hat{a} + x_{k} \hat{b} + y_{k} \hat{a}^\dagger + z_{k} \hat{b}^\dagger, k=\text{LP, UP}
\end{equation}
この演算子\(\hat{p}_{k}\)はポラリトンの消滅演算子であり, 係数\(w_{k}, x_{k}, y_{k}, z_{k}\)は対角化により決定される複素数である. これにより, ハミルトニアン\(\hat{\mathcal{H}}\)はポラリトン演算子を用いて以下のように対角化される.
\begin{equation}
    \hat{\mathcal{H}} = \sum_{k=\text{LP, UP}} \hbar \Omega_{k} \hat{p}_{k}^\dagger \hat{p}_{k}
\end{equation}
ここで得られる固有エネルギー\(\hbar \Omega_{LP}, \hbar \Omega_{UP}\)はそれぞれ下枝ポラリトン(Lower Polariton: LP)と上枝ポラリトン(Upper Polariton: UP)のエネルギーを表し, 実験で観測される共振器モードの分裂ピークに対応する. すなわち, 観測されるスペクトルはもはや純粋な光子や物質励起のものではなく, これらが強く混成したポラリトン状態のものであることを示している.

\section{超放射相転移}
\label{sec:intro_SRPT}
USC領域において特筆すべき物理現象として, Heppら\(^{\cite{Hepp1973, Wang1973}}\)によって1973年に提唱された「超放射相転移(Superradiant Phase Transition: SRPT)」が挙げられる. SRPTとは多数の原子と単一の光モードが相互作用するDickeモデル\(^{\cite{Dicke1954}}\)において, 結合強度が臨界点を超えた瞬間に, 光子場と物質場がそれぞれ静的横方向電磁場と静的分極として熱平衡状態で自発的に現れる.これに伴い,基底状態が巨視的な光子数を伴う状態へと相転移する現象である. SRPTの特異性は, これが光と物質の結合系の熱平衡状態における静的な相転移であることだ. 
従来の量子技術では外部駆動で生成した光子のスクイーズド状態を利用するが,その状態は伝搬・検出過程の光子損失で急速に劣化する.これに対しSRPTの臨界点では,系の基底状態として強いスクイージングが自発的に現れるため\(^{\cite{Hayashida2023}}\),外部駆動不要で時間経過に頑健である.この「内在的スクイージング」は環境ノイズに強い量子技術の基盤となりうる. 

\section{No-go定理と磁性体の可能性}
\label{sec:intro_no_go_theorem}
SRPTの実現に向けた障壁として「no-go定理」が存在する.(詳細は\ref{chap:appendix_A}にて記載.) 最小結合ハミルトニアンで記述される系, すなわち電磁場と相互作用する荷電粒子系においては, ベクトルポテンシャルの2乗項(\(A^2\)項)に由来する反磁性効果により, 熱平衡状態でのSRPTは原理的に実現できないことが示唆されている.\(^{\cite{Rzazewski1975}}\) 実際に, 熱平衡状態におけるSRPTは実験的には未だ観測されていない. これに対し,磁気相互作用が支配的な系はこの定理を回避しうると指摘されていた.\(^{\cite{Knight1978}}\)
これまで磁気相互作用による結合定数は電気相互作用に比べて小さく, USCの達成が困難であった. 

\section{研究対象物質:\(\text{Gd}_3 \text{Ga}_5 \text{O}_{12}\)(Gadolinium Gallium Garnet; GGG)}
\label{sec:intro_GGG}
本研究で着目するのは, 希土類磁性体の一種である\(\text{Gd}_3 \text{Ga}_5 \text{O}_{12}\); GGG である. GGG中の磁性イオンである\(\text{Gd}^{3+}\)は, 電子配置\(4f^7 5s^2 p^6\), \(S=7/2\)という大きなスピンを持ち, スピン間の相互作用が弱い理想的な常磁性体として知られる. 
特に, その電子常磁性共鳴(Electron Paramagnetic Resonance; EPR)はTHz帯に存在し, 高密度のスピン集団とTHz光を結合させることで, USC領域, さらにはSRPTの実現可能性を秘めた有望な物質系として注目されている. 

\section{Zeemanポラリトン形成の仕組み}
\label{sec:intro_zeeman}
GGGのような磁性体においては, 電磁場の振動磁場成分と電子スピンとがZeeman相互作用を介して結合するため, これらの混成準粒子は「Zeemanポラリトン」と称されることがある.\(^{\cite{Kritzell2024,Sakata2025}}\) 
Zeemanポラリトンの形成は, 光(共振器モード)と, 静磁場によるZeeman分裂で生じたスピン系のエネルギー準位が共鳴する条件下で顕著に現れ, 透過スペクトルにおける分裂や反交差現象として観測される. これらの特徴的なスペクトル構造は, 光-物質間のZeeman相互作用の強さや性質を反映しており, SRPTの発現可能性を評価する上で重要な手がかりとなる.
ここでは, GGGにおけるZeemanポラリトン形成の基本的な仕組みを以下に示す. 定量的な説明は付録\ref{chap:zeeman_polariton}で詳述する.
\begin{enumerate}
    \item Zeeman分裂(静的効果):\\
    GGGは\(\text{Gd}^{3+}\)イオンを含む常磁性体であり, \(\text{Gd}^{3+}\)イオンは7つの不対電子を持つため, 大きな全スピン量子数\(S=7/2\)を有する. ゼロ磁場下ではエネルギー準位が縮退しているが, 外部から静磁場\(\vb*{B}_{\text{ext}}\)を印加すると, 縮退していたエネルギー準位がスピン磁気量子数\(m_s\) (\(m_s = -7/2, \dots, +7/2\)) に応じて8個の準位に分裂する. この分裂幅は印加した静磁場\(\vb*{B}_{\text{ext}}\)の強度に比例する. 
    
    \item 電子常磁性共鳴 (EPR):\\
    Zeeman分裂によって生じたエネルギー準位間に対し, そのエネルギー差に相当する周波数を持つTHz光(振動磁場)が入射すると, 電子スピンが光子のエネルギーを共鳴的に吸収して励起される. これが電子常磁性共鳴(EPR)であり, 本研究におけるGGGではTHz帯域で観測される. 
    
    \item Zeeman相互作用によるZeemanポラリトンの形成(動的効果):\\
    スピン集団の磁気モーメントと共振器光子場による磁場の間のZeeman相互作用(結合強度)が, 共振器の損失や脱位相を上回り, かつ超強結合(USC)領域(\(\eta > 0.1\))に達すると, 系は強い量子相関を持つようになる. この領域では, 光子とスピン励起(マグノン)の間でコヒーレントなエネルギー交換が高速に行われ, 両者が量子力学的に混成した新たな固有状態である「Zeemanポラリトン」が形成される. この詳細な数理的定義については付録\ref{chap:zeeman_polariton}で述べる. 
    % スピンと光子の相互作用がUSC領域に達すると, スピンの向きと電磁場の変化が時間的に同期し, エネルギーのやり取りが繰り返し行われるようになる. この結果, 光子とスピン励起が強く混成し, ハイブリッドな準粒状態であるZeemanポラリトンが形成される.より詳細な議論は付録\ref{chap:zeeman_polariton}で述べる.
\end{enumerate}

\section{GGGの先行研究}
この節では, GGGの光-スピン強結合系に関する2つの先行研究について概説する.

\subsection{実験的探索: Kritzellらの先行研究}
\label{sec:intro_previous_work_Kritzell}
近年, Kritzellらは常磁性体であるGGGとTHz光の結合系においてZeemanポラリトン形成を観測したことでUSCを実験的に達成し, この系がSRPTの有力な候補であることを示した.\(^{\cite{Kritzell2024}}\) 実験系の概要を図\ref{fig:intro_kritzell_setup}に示す. \\
彼らは, GGG試料の厚みを研磨して調整し, 試料と真空の界面を利用してFabrry-Pérot(FP)共振器を形成するように設計した. 外部静磁場\(\vb*{B}_{\text{ext}}\)を掃引することでEPR周波数\(\omega_{\text{EPR}}\)を変化させ, FP共振器モードの共鳴周波数\(\omega_{\text{cav}}\)を共鳴させた. 外部静磁場と温度を制御しながら透過スペクトルを測定した. その結果, スピン-光子結合による明確な分裂構造が観測され, 低温(\(T=1.5~K\))において, 結合強度が共鳴周波数の約60\%(正規化結合強度\( \eta = 0.32\))に達する結合強度がUSC領域に達していることが確認された. さらに, 温度依存性の測定から, GGGのスピン系が熱平衡状態にあることも言及された. これらの結果は, GGGのTHz磁気光学応答がSRPTの実現に向けた有望なプラットフォームであることを強く示唆している. 

\begin{figure}[h]
    \centering
    \includegraphics[width=0.7\textwidth]{figures/exp_systme_sample.png}
    \caption{Kritzellらの実験系の概要図.\(^{\cite{Kritzell2024}}\) GGG試料の厚みを調整することで, 試料自体がFabrry-Pérot(FP)共振器を形成するようにに設計した. EPRの共鳴周波数\(\omega_{\text{EPR}}\)とFP共振器モードの共鳴周波数\(\omega_{\text{cav}}\)を共鳴させることでZeemanポラリトン形成を観測した. 外部静磁場と温度を制御しながら透過スペクトルを測定した.}
    \label{fig:intro_kritzell_setup}
\end{figure}
\clearpage

\subsection{解析: 山田・坂田の先行研究}
\label{sec:intro_previous_work_sakata}
GGGのTHz帯における磁気光学応答の透過スペクトルは分裂した複数のピークが現れるが, これは結晶場とZeeman分裂, Zeeman相互作用に起因すると考えられている.\(^{\cite{Yamada2024}}\) このスペクトルを理論的に記述するため, 先行研究ではZeeman相互作用に注目して二つのモデルが提案されてきた. \(^{\cite{Sakata2025}}\)
GGGのZeeman相互作用ハミルトニアン\(\hat{\mathcal{H}}_{H,B}^{'}\)は\(\text{Gd}^{3+}\)イオンの電子スピンに由来する磁気モーメント\(\hat{\vb*{d}}\)を用いてそれぞれ以下のように定義されている. 

\begin{enumerate}
    \item Bモデル (Zeeman相互作用: 磁気モーメント\(\hat{\vb*{d}}\) - 磁束密度\(\hat{\vb*{B}}\))\\
    定義式: \begin{equation}
            \hat{H}_{B}^{'}=-\hat{\vb*{d}}\cdot \hat{\vb*{B}}
            \label{def:H_B}
            \end{equation}
    物理的解釈: Maxwell方程式(\(\vb*{B}=\vb*{\nabla} \times \vb*{A}\))を量子化されたベクトルポテンシャル\(\hat{\vb*{A}}\)を用いて解くことで, 磁束密度演算子\(\hat{\vb*{B}}\)が光子演算子\(\hat{a}\)を含む形で表される. (詳細な計算は\ref{chap:zeeman_polariton}にて示す. )このとき\(\hat{\vb*{B}}\)は共振器由来の磁場のみを表し, 試料の磁化\(\vb*{M}\)による寄与は含まれない. 従って, Bモデルでは磁気モーメント\(\hat{\vb*{d}}\)は光子(共振器モード)\(\hat{a}, \hat{a}^{\dagger}\)にのみ結合する. \\
    \item Hモデル (Zeeman相互作用: 磁気モーメント\(\hat{\vb*{d}}\) - 磁場 \(\vb*{H}\) )\\
    定義式: \begin{equation}
            \hat{H}_{H}^{'}=-\hat{\vb*{d}}\cdot\mu_{0} \hat{\vb*{H}}
            \label{def:H_H}
            \end{equation}
    物理的解釈: 磁場\(\vb*{H}\)は\(\vb*{H} = \vb*{B}/\mu_{0} - \vb*{M}\)で定義されるため, 試料内の他の磁気モーメントによる巨視的磁化\(\hat{\vb*{M}}=\frac{N}{V}\hat{\vb*{d}}\)が考慮される. 結果として, Hモデルでは磁気モーメント\(\hat{\vb*{d}}\)は光子(共振器モード)\(\hat{a}, \hat{a}^{\dagger}\)と磁化演算子\(\hat{\vb*{M}}\)の両方に結合する. \\
\end{enumerate}
ここで\(\mu_{0}\)は真空の透磁率, \(N/V\)は単位体積あたりの\(\text{Gd}^{3+}\)イオン数を表す. 磁気モーメント\(\hat{\vb*{d}}\)はボーア磁子\(\mu_{B}\), g因子\(g_{J}\), スピン演算子\(\hat{\vb*{S}}\)を用いて次のように表される. 
\begin{equation}
    \hat{\vb*{d}} = -g_{J} \mu_{B} \hat{\vb*{S}}
\end{equation}

Zeeman相互作用ハミルトニアンがHモデルまたはBモデルのどちらで記述されるべきか現在も議論の対象である. Bモデルはno-go定理を回避できる一方で, Hモデルでは\({A^2}\)項に相当する相互作用が現れ相転移が抑制されるため\(^{\cite{Sakata2025}}\), 物理モデル選択はSRPT実現可能性の結論に直結するため重要な項目である. SRPTの発現可能性については, 付録\ref{chap:SRPT_realization}で詳細に議論する. 
坂田らは線形応答理論によりGGGの磁気感受率\(\chi (\omega)\)と両モデルの透過スペクトルを導出したが, 以下の理由から実験データを説明するモデルの特定には至らなかった. 
第一に, Kritzellらの実験が磁場・温度依存性の双方を測定したのに対し, 坂田らの解析は一方の解析に留まっていた点である. 第二に, 理論と実験の比較が定性的な議論に終始していた点である.  \(^{\cite{Sakata2025}}\)


\section{本研究の目的と構成}
本研究の目的は, GGGを用いた光・スピン強結合系におけるSRPT発現の真の可能性を明らかにすることとした.具体的には, Kritzellらの実験データを磁場依存性・温度依存性の両観点から統合的に解析することに加えて, ベイズ推定に基づく定量的評価を行い, Hモデル及びBモデルの妥当性を判定することである. \\
従来の最小二乗法が単一の最適解(点推定)を求めるのに対し, ベイズ推定はパラメータを確率変数と見なし, その存在確率の分布(事後分布)全体を導出する. これにより, パラメータ間の相関や不定性, モデル全体の確からしさを定量化し, 不確実性を考慮した堅牢なモデル比較が可能となる. 特に, モデルの予測性能を評価するを評価指標を2種類導入し, どちらの物理モデルがKritzellらの実験データ\(^{\cite{Kritzell2024}}\)をより良く説明するかを明らかにする. 本研究では, 確率的プログラミングライブラリPyMC\(^{\cite{Abril-Pla2023}}\)を用いてこれを実装した.\\
本論文の構成は以下の通りである. 第\ref{chap:theory}章では, GGGの磁気感受率を導出するための半古典論的な枠組みと, 転送行列法による透過スペクトルの計算手法, そしてモデル評価に用いるベイズ推定の理論を述べる. 第\ref{chap:method}章では, 実際に行った解析手法について詳述する. 第\ref{chap:results}章では, 得られた結果を物理的に考察する. 第\ref{chap:conclusion}章で結論と今後の展望を述べる. 