\subsection{NUTS(No-U-Return Sampler)}
NUTS (No-U-Turn Sampler) は, ハミルトニアンモンテカルロ法 (HMC) を基盤とした現代のベイズ統計ソフトウェア(Stan, PyMCなど)で標準エンジンとなっている最先端アルゴリズムである. \(^{\cite{Hoffman2014}}\)HMCは, 物理学のハミルトニアン力学系の概念を利用して, パラメータ空間を効率的に探索する手法であり, 以下に概要を示す. 
\subsubsection*{HMCの基礎}
HMCは, パラメータ \(\theta\) に対して仮想的な運動量 \(p\) を導入し, ハミルトニアン \(H(\theta, p) = U(\theta) + K(p)\) を定義する. ここで, \(U(\theta) = -\log P(\theta|D)\) はポテンシャルエネルギー, \(K(p) = \frac{1}{2} p^T M^{-1} p\) は運動エネルギーであり, \(M\) は質量行列である. HMCは, レヴィ・カンター方程式に基づいて \((\theta, p)\) の時間発展をシミュレートし, 新しい候補状態を生成する. この方法により, 高次元空間でも効率的にサンプリングが可能となる.
\subsubsection*{NUTSの革新性}
HMCの最大の課題は, 積分ステップ数\(L\)の調整である. \(L\)が短すぎるとランダムウォークになり, 長すぎると軌道が一周して元の位置に戻ってしまい(Uターン), 計算資源を浪費する. 
NUTSは, 以下の手順でこれを自動化する:
\begin{enumerate}
    \item \textbf{再帰的な二分木の構築}: ハミルトニアン力学系のシミュレーションを前進と後退の両方向に行い, ツリー構造を再帰的に構築する.
    \item \textbf{No-U-Turn停止条件の判定}: 構築された軌道の始点\(\theta^-\)と終点\(\theta^+\)を結ぶベクトルと, 終点の運動量ベクトル\(r^+\)の内積を確認する. これが負になった(\(( \theta^+ - \theta^- ) \cdot r^+ < 0\)), すなわち粒子が来た道を戻り始めた時点で, 木の拡張を停止する.
    \item \textbf{詳細釣り合いの維持}: 新しい候補状態を選択する際に, 停止位置までの軌道上の点から, スライスサンプリングの考え方を応用して次の点を抽出する.これにより, 詳細釣り合い条件を満たすように確率的に選択を行う.
\end{enumerate}
これにより, 高次元かつ相関の強いGGGの結晶場パラメータ空間においても, 極めて高い効率(有効サンプルサイズの最大化)で事後分布を探索することが可能となる. 