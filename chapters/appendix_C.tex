\chapter{H形式とB形式のSRPT発現可能性の違い}
\label{chap:SRPT_realization}

電磁気学において, 磁場を表す物理量には磁束密度\(\vb*{B}\)と磁場の強さ\(\vb*{H}\)の二種類が存在する. これらは真空中では\(\vb*{B} = \mu_0 \vb*{H}\)という単純な比例関係にあるが, GGGのような磁性体内では, その定義と物理的役割に本質的な差異が生じる. \ref{sec:theory_permeability}節でも述べたように, 転送行列法による光学特性の計算においては, どちらの形式を採用するかによって透磁率や境界条件の定義が異なるため, 明確な区別が必要である.
本付録では, 一般の物質中における電磁波の分散関係を導出し, その過程で\(\vb*{B}\)形式と\(\vb*{H}\)形式の違いがどのように分散関係に影響を与えるかを明らかにする. これにより, Zeemanポラリトンの分散関係が両形式でどのように異なるかを理解し, SRPT発現可能性の違いを理論的に裏付ける基礎を提供する.

\section{一般の物質中における電磁波の分散関係の導出}
\subsection{マクスウェル方程式と構成方程式}
電荷密度 \(\rho = 0\) および電流密度 \(\vb*{J} = \vb*{0}\) の一般の物質中におけるMaxwell方程式は以下の通りである. 
\begin{align}
    \nabla \cdot \vb*{D} &= 0 \\
    \nabla \cdot \vb*{B} &= 0 \\
    \nabla \times \vb*{E} &= -\frac{\partial \vb*{B}}{\partial t} \label{eq:Faraday} \\
    \nabla \times \vb*{H} &= \frac{\partial \vb*{D}}{\partial t} \label{eq:Ampere}
\end{align}
ここで, 線形応答を示す等方的な物質を仮定し, 周波数領域における構成方程式を導入する. 
電束密度 \(\vb*{D}\) および磁束密度 \(\vb*{B}\) は, それぞれ電場 \(\vb*{E}\) および磁場 \(\vb*{H}\) と以下の関係にある. 
\begin{align}
    \vb*{D} &= \epsilon_0 \epsilon_r(\omega) \vb*{E} \\
    \vb*{B} &= \mu_0 \mu_r(\omega) \vb*{H}
\end{align}
\(\epsilon_0, \mu_0\) は真空の誘電率と透磁率であり, \(\epsilon_r(\omega), \mu_r(\omega)\) はそれぞれ物質の比誘電率と比透磁率である. 

\subsection{波動方程式の導出}
式(\ref{eq:Faraday})の両辺の回転をとる. 
\begin{equation}
    \nabla \times (\nabla \times \vb*{E}) = -\frac{\partial}{\partial t} (\nabla \times \vb*{B})
\end{equation}
左辺に対し, ベクトル解析の恒等式 \(\nabla \times (\nabla \times \vb*{A}) = \nabla(\nabla \cdot \vb*{A}) - \nabla^2 \vb*{A}\) を適用する. 均質な媒質中では \(\nabla \cdot \vb*{E} = 0\) となるため, 左辺は \(-\nabla^2 \vb*{E}\) となる. 
右辺に対し, \(\vb*{B} = \mu_0 \mu_r \vb*{H}\) および式(\ref{eq:Ampere})を代入する. 
\begin{align}
    -\nabla^2 \vb*{E} &= -\mu_0 \mu_r \frac{\partial}{\partial t} (\nabla \times \vb*{H}) \nonumber \\
    &= -\mu_0 \mu_r \frac{\partial}{\partial t} \left( \epsilon_0 \epsilon_r \frac{\partial \vb*{E}}{\partial t} \right) \nonumber \\
    &= -\epsilon_0 \mu_0 \epsilon_r \mu_r \frac{\partial^2 \vb*{E}}{\partial t^2}
\end{align}
真空中の光速 \(c = 1/\sqrt{\epsilon_0 \mu_0}\) を用いて整理すると, 以下の波動方程式が得られる. 
\begin{equation}
    \left( \nabla^2 - \frac{\epsilon_r \mu_r}{c^2} \frac{\partial^2}{\partial t^2} \right) \vb*{E}(\vb*{r}, t) = 0
\end{equation}

\subsection{平面波解と分散関係}
単色平面波解 \(\vb*{E}(\vb*{r}, t) = \vb*{E}_0 e^{i(\vb*{k} \cdot \vb*{r} - \omega t)}\) を仮定し, 波動方程式に代入する. 
空間微分 \(\nabla \to i\vb*{k}\), 時間微分 \(\partial/\partial t \to -i\omega\) の置き換えにより, 
\begin{equation}
    -k^2 \vb*{E}_0 + \frac{\epsilon_r \mu_r}{c^2} \omega^2 \vb*{E}_0 = 0
\end{equation}
非自明な解(\(\vb*{E}_0 \neq \vb*{0}\))が存在するための条件として, 以下の分散関係式が得られる. 
\begin{equation}
    k^2 = \epsilon_r(\omega) \mu_r(\omega) \frac{\omega^2}{c^2}
\end{equation}
あるいは, 屈折率 \(n(\omega) = \sqrt{\epsilon_r(\omega) \mu_r(\omega)}\) を用いて, 
\begin{equation}
    \frac{ck}{\omega} = \sqrt{\epsilon_r(\omega) \mu_r(\omega)}
    \label{eq:dispersion_general}
\end{equation}
と表される. この式(\ref{eq:dispersion_general})が, 一般の物質中における光子(ポラリトン)の分散関係の基礎式である. 
本研究で扱うZeemanポラリトンの場合, \(\mu_r(\omega)\) の定義(B形式またはH形式)によって右辺の周波数依存性が変化し, 分散曲線の形状に図\ref{fig:dispersion_relation}のように決定的な影響を与える. 

\begin{figure}
    \centering
    \includegraphics[width=1.0\textwidth]{figures/dispersion_relation.png}
    \caption{一般の物質中における光子(ポラリトン)の分散関係. 比誘電率 \(\epsilon_r(\omega)\) および比透磁率 \(\mu_r(\omega)\) の周波数依存性により, 分散曲線の形状が決定される. Zeemanポラリトンの場合, \(\mu_r(\omega)\) の定義(B形式またはH形式)によって分散関係が変化する. ただし, \(\omega_0\)は\(\omega_{\text{cav}}=\omega_{\text{EPR}}\)を満たす共鳴周波数}
    \label{fig:dispersion_relation}
\end{figure}