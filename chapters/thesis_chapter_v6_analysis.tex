% ============================================================================
% 修士論文:共有ガンマモデルによるグローバルフィッティング解析
% pre_test_v6_shared_gamma.py による解析結果
% ============================================================================

\chapter{WNLLSの結果と考察}



\section{考察}

\subsection{共有ガンマモデルの有効性}

本研究で提案した共有ガンマモデルは,以下の点で従来モデルより優れている:

\begin{enumerate}
    \item \textbf{パラメータ削減}:75個$\to$12個(84\%削減)
    \item \textbf{条件数改善}:$10^{16} \to 10^5$--$10^6$(約10桁改善)
    \item \textbf{物理的解釈}:$\gamma_k$を材料固有の緩和率として明確に解釈可能
    \item \textbf{汎用性}:異なる温度・磁場条件のデータを統一的に解析可能
\end{enumerate}

\subsection{モデル形式の選択}

H形式とB形式の比較から,以下の知見が得られた:

\begin{itemize}
    \item B形式の方がわずかに低いコスト(5.4\%改善)を示すが,
          数値不安定性($|\chi| > 1$)の問題がある
    \item H形式は数値的に安定であり,全データセットで不安定点がゼロ
    \item 決定係数$R^2$は両形式でほぼ同等(0.65--0.77)
\end{itemize}

実用的には,数値安定性の観点からH形式が推奨されるが,
強結合領域ではB形式の非線形補正が物理的により正確である可能性がある. 

\subsection{フィット品質の改善余地}

低磁場(4.2T--5.0T)データでは$R^2 \approx 0.5$と低く,
以下の改善策が考えられる:

\begin{enumerate}
    \item 磁場依存の緩和率モデル:$\gamma_k(B)$
    \item 非線形結合効果の導入
    \item ポラリトンモード領域の重み付け最適化
    \item 追加の結晶場項(低対称性補正)
\end{enumerate}

\subsection{結論}

共有ガンマモデルによるグローバルフィッティングにより,
GGG単結晶の磁気ポラリトン透過スペクトルを統一的に解析することができた. 
主要な物理パラメータ($g$因子,結晶場パラメータ,緩和率)を
12個のフィッティングパラメータで記述し,
条件数の大幅な改善($10^{16} \to 10^5$)を達成した. 

今後の課題として,ベイズ推定によるパラメータ不確かさの定量化,
および低磁場領域でのモデル改善が挙げられる. 