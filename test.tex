\documentclass[a4paper,11pt]{jsarticle}
\usepackage[dvipdfmx]{graphicx}
\usepackage{amsmath, amssymb}
\usepackage{bm} % ベクトル太字用
\usepackage{url}
\usepackage{cite}

\title{ガドリニウム・ガリウム・ガーネットにおけるゼーマン・ポラリトン形成の微視的理論:\\ハミルトニアン動力学, 結晶場相互作用, およびDicke模型への写像に関する包括的研究報告書}
\author{}
\date{\today}

\begin{document}

\maketitle



\section{ガドリニウム・ガリウム・ガーネット(GGG)の結晶学的・磁気的基礎}

微視的ハミルトニアンを構築する上で, ホスト結晶であるGGGの構造的特徴と, 磁性中心である$\mathrm{Gd}^{3+}$イオンの局所環境を理解することは不可欠である. 

\subsection{結晶構造と空間群}
GGGは化学式 $\mathrm{Gd_3Ga_5O_{12}}$ で表されるガーネット構造を有する酸化物結晶である. その結晶構造は立方晶系に属し, 空間群は $Ia\bar{3}d$ ($O_h^{10}$) である. 単位胞(ユニットセル)は大きく, 8式量(計160原子)を含んでいる. ガーネット構造における陽イオンの占有サイトは以下の3種類に分類される:

\begin{itemize}
    \item \textbf{24c サイト(十二面体サイト)}: 大きなイオン半径を持つ希土類イオンが占有する. GGGにおいては $\mathrm{Gd}^{3+}$ イオンがこの位置にある. 酸素イオンによって形成される歪んだ十二面体(Dodecahedron)の中心に位置する. 
    \item \textbf{16a サイト(八面体サイト)}: $\mathrm{Ga}^{3+}$ イオンが占有する. 酸素八面体(Octahedron)の中心に位置し, $C_{3i}$ ($S_6$) 対称性を持つ. 
    \item \textbf{24d サイト(四面体サイト)}: $\mathrm{Ga}^{3+}$ イオンが占有する. 酸素四面体(Tetrahedron)の中心に位置し, $S_4$ 対称性を持つ. 
\end{itemize}

ゼーマン・ポラリトンの形成に寄与するのは, 24cサイトに位置する $\mathrm{Gd}^{3+}$ イオンのスピンである. 結晶全体としては立方対称性を持つものの, 個々の $\mathrm{Gd}^{3+}$ イオンが感じる局所的な結晶場(Local Crystal Field)は, より低い対称性である\textbf{斜方対称性(Orthorhombic symmetry, 点群 $D_2$ または 222)}を有している\cite{ref189}. この局所対称性の低下こそが, 結晶場ハミルトニアンにおける異方性項の出現を許し, スピン準位の微細構造(Fine Structure)を決定づける要因となる. 

\subsection{磁気的に非等価なサイト}
単位胞内には24個の $\mathrm{Gd}^{3+}$ イオンが存在するが, 外部磁場を特定の結晶軸方向に印加した場合, これらのイオンは磁気的に等価な複数のサブグループ(副格子)に分類される. 一般の磁場方向に対しては, 最大で6種類の磁気的に非等価なサイトが存在する. しかし, 実験的な解析を容易にするため, 通常は高い対称性を持つ軸(例えば $[100], [110], [111]$ 方向)に沿って磁場を印加する. 

\begin{itemize}
    \item \textbf{[100] 方向}: 24個のサイトは2つの等価なグループに分かれる(各12個, あるいは8個と16個等の構成となるが, 角度により縮退する). 
    \item \textbf{[110] 方向}: 一般に2つの非等価なサイト(個数比 1:3 程度)として観測されることが多い. 
\end{itemize}

このサイトの非等価性は, EPR(電子スピン共鳴)スペクトルにおいて複数の共鳴線が観測される原因となる. ゼーマン・ポラリトンの理論モデルにおいては, これらのサイトが独立したアンサンブルとして振る舞うか, あるいは平均化された有効的な単一アンサンブルとして扱えるかが重要な議論点となるが, 多くの場合, 主要な遷移に注目することで単一の有効スピン系として記述が可能である. 

\subsection{$\mathrm{Gd}^{3+}$ イオンの電子配置と基底状態}
ガドリニウム(Gd)は原子番号64の希土類元素(ランタノイド)であり, 3価のイオン $\mathrm{Gd}^{3+}$ として結晶中に存在する. その電子配置は $[\mathrm{Xe}] 4f^7$ である. 4f殻がちょうど半閉殻(Half-filled shell)となっている点が極めて重要である. 

Hundの規則(フントの規則)に従い, 基底状態の項記号(Term Symbol)は以下のように決定される:
\begin{itemize}
    \item \textbf{スピン角運動量 $S$}: 7つの電子がすべて平行スピンを持つ配置をとるため, $S = \sum m_s = 7 \times (1/2) = 7/2$ となる. 
    \item \textbf{軌道角運動量 $L$}: 磁気量子数 $m_l$ は $+3, +2, +1, 0, -1, -2, -3$ を取るため, その総和は $L = \sum m_l = 0$ となる. 
    \item \textbf{全角運動量 $J$}: $L=0$ であるため, スピン軌道相互作用による分裂を考慮しても $J = S = 7/2$ である. 
\end{itemize}
したがって, 基底状態は $^8S_{7/2}$ である. 

$L=0$ であることから, $\mathrm{Gd}^{3+}$ は「S状態イオン(S-state ion)」と呼ばれる. 第一近似において, 球対称な電荷分布を持つため, 結晶場(配位子場)の静電ポテンシャルによる直接的な影響を受けにくい. しかしながら, 「結晶場分裂がゼロである」というのはあくまで第一近似に過ぎない. 実際には, スピン軌道相互作用やスピン-スピン相互作用を介した高次の摂動効果により, 励起状態($L \neq 0$ を持つ項, 例えば $^6P$ 項など)が基底状態にわずかに混和する\cite{ref178}. この微小な混和を通じて, 結晶場の対称性を反映した異方性が基底状態の $2S+1 = 8$ 重の縮退を解き, ゼロ磁場分裂(Zero-Field Splitting: ZFS)を引き起こす. この微細構造こそが, GGGにおけるゼーマン・ポラリトン系の非調和性を生み出す源泉である. 

\section{微視的ハミルトニアンの定式化}

GGG中の単一 $\mathrm{Gd}^{3+}$ イオンを記述する微視的ハミルトニアン $H_{micro}$ は, 以下の主要な項の和として表される:

\begin{equation}
H_{micro} = H_{Zeeman} + H_{CF} + H_{dip} + H_{ex} + H_{HF}
\end{equation}

ここで, $H_{Zeeman}$は外部磁場とのゼーマン相互作用, $H_{CF}$は結晶場相互作用, $H_{dip}$は磁気双極子相互作用, $H_{ex}$は交換相互作用, $H_{HF}$は超微細構造相互作用を表す. 
本研究の対象である「ゼーマン・ポラリトン」は, 通常 1 テスラ(T)以上の強磁場下, かつ極低温から室温までの領域で議論される. この条件下では, ゼーマン項が支配的であり, 結晶場項がそれに次ぐ摂動として準位構造を決定する. 双極子相互作用と交換相互作用は線幅や微小なシフトに寄与し, 超微細相互作用は通常無視できるほど小さい\cite{ref179}. 

\subsection{ゼーマン相互作用項}
外部静磁場 $\mathbf{B}_{0}$ が印加された際のゼーマン・ハミルトニアンは次式で与えられる:

\begin{equation}
H_{Zeeman} = \mu_B \mathbf{B}_{0} \cdot \mathbf{g} \cdot \mathbf{S}
\end{equation}

ここで, $\mu_B$ はボーア磁子, $\mathbf{S}$ は有効スピン演算子($S=7/2$), $\mathbf{g}$ はgテンソルである. $\mathrm{Gd}^{3+}$ はS状態イオンであるため, g値は自由電子のg値($g_e \approx 2.0023$)に極めて近い等方的な値($g \approx 1.99 \sim 2.00$)をとる\cite{ref185}. したがって, 実用的なハミルトニアンとしては, スカラーのg因子を用いて単純化できる:

\begin{equation}
H_{Zeeman} \approx g \mu_B B_z S_z
\end{equation}

この項は, エネルギー準位を等間隔($\Delta E = g \mu_B B_0$)に分裂させる. 磁場 $B_0 = 1$ T において, この分裂幅は約 28 GHz に相当する. 

\subsection{結晶場(CF)ハミルトニアンとスティーブンス演算子}
結晶場ハミルトニアン $H_{CF}$ は, 局所対称性が $D_2$ であることを考慮し, スティーブンス演算子(Stevens Operators)$O_k^q$ を用いて展開するのが標準的な手法である\cite{ref182}. 一般形は以下の通りである:

\begin{equation}
H_{CF} = \sum_{k} \sum_{q=-k}^{k} B_k^q O_k^q
\end{equation}

ここで, $B_k^q$ は結晶場パラメータ(CFPs)である. $\mathrm{Gd}^{3+}$ ($S=7/2$) の場合, パリティ($k$は偶数), スピンの大きさ($k \le 2S=7$), 対称性($q$は偶数)の制約により, 以下の9項が許容される:

\begin{equation}
H_{CF} = B_{2}^{0}O_{2}^{0} + B_{2}^{2}O_{2}^{2} + B_{4}^{0}O_{4}^{0} + B_{4}^{2}O_{4}^{2} + B_{4}^{4}O_{4}^{4} + B_{6}^{0}O_{6}^{0} + B_{6}^{2}O_{6}^{2} + B_{6}^{4}O_{6}^{4} + B_{6}^{6}O_{6}^{6}
\end{equation}

このうち, 最も支配的な影響を与えるのは2次の項($B_2^0, B_2^2$)であり, これらがゼロ磁場分裂(ZFS)の主成分をなす. 

\subsubsection{スティーブンス演算子の具体的な行列表現}
$S=7/2$ 系における主要な演算子の定義を以下に示す\cite{ref183}. これらはすべて $8 \times 8$ の行列として表現される. 

\textbf{ランク2(四重極項):}
\begin{align}
O_2^0 &= 3S_z^2 - S(S+1) \\
O_2^2 &= \frac{1}{2} (S_+^2 + S_-^2)
\end{align}
$O_2^0$は「軸性(Axial)」異方性を表し, 準位の等間隔性を破る主要因である. $O_2^2$は「菱面体(Rhombic)」異方性を表し, $\Delta m_S = \pm 2$ の状態混合を引き起こす. 

\textbf{ランク4:}
\begin{align}
O_4^0 &= 35S_z^4 - 30S(S+1)S_z^2 + 25S_z^2 - 6S(S+1) + 3S^2(S+1)^2 \\
O_4^4 &= \frac{1}{2} (S_+^4 + S_-^4)
\end{align}
これらは立方対称性の場において重要な項である. 

これらの演算子は, ハミルトニアンの固有状態 $|\psi_n\rangle$ を純粋な $|m_S\rangle$ 状態ではなく, それらの線形結合(混成状態)とする. 

\subsection{結晶場パラメータ(CFP)の標準化と実験値}
Rudowiczらは, パラメータセットが一意に定まらない問題に対し, 「標準化(Standardization)」の手法を提唱している\cite{ref176, ref187}. GGGにおける結晶場パラメータの代表的な値を表\ref{tab:cfp}に示す. 

\begin{table}[h]
\centering
\caption{GGGおよびYAG中のGd$^{3+}$に対する結晶場パラメータ(スティーブンス記法)}
\label{tab:cfp}
\begin{tabular}{|l|c|c|c|l|}
\hline
パラメータ & 記号 & GGGでの概算値 & 対応周波数 & 物理的意味 \\
 & & ($10^{-4} \text{ cm}^{-1}$) & (MHz) & \\
\hline
2次軸性 & $B_2^0$ & $-41 \sim -100$ & $-120 \sim -300$ & 主軸方向の異方性(ZFS主成分) \\
2次菱面体 & $B_2^2$ & $135 \sim 200$ & $400 \sim 600$ & XY面内の異方性と状態混合 \\
4次軸性 & $B_4^0$ & $-1 \sim -2$ & $-3 \sim -6$ & 4次の多重極相互作用 \\
4次立方 & $B_4^4$ & $\sim -10$ & $\sim -30$ & 立方対称性を反映した項 \\
6次項 & $B_6^q$ & $< 1$ & $< 3$ & 非常に小さい \\
\hline
\end{tabular}
\end{table}

この表から, \textbf{$B_2^0$ と $B_2^2$ が他の項に比べて圧倒的に大きい}ことがわかる. これは, ハミルトニアンの非調和性が主に2次の四重極相互作用によって支配されていることを意味する. 

\section{エネルギー準位構造と非調和性}

\subsection{ゼロ磁場分裂(ZFS)}
外部磁場がない場合 ($B=0$), $S=7/2$(半整数スピン)であるため, クラマースの定理(Kramers' Theorem)により, 8つの基底状態は4つのクラマース二重項(Kramers Doublets)に分裂する. 

\subsection{強磁場下での準位構造と非調和ラダー}
強磁場(例えば $B \parallel z$)を印加すると, ゼーマン項が支配的となり, 準位は再配列される. しかし, 結晶場項 $B_2^0 O_2^0$ の存在により, エネルギー固有値の一次摂動近似は以下のようになる:

\begin{equation}
E_m \approx g \mu_B B_0 m + B_2^0
\end{equation}

隣接する準位間の遷移エネルギー $\Delta E_{m \to m+1}$ は:

\begin{equation}
\Delta E_{m \to m+1} = E_{m+1} - E_m \approx g \mu_B B_0 + 3 B_2^0 (2m+1)
\end{equation}

この式は, 遷移エネルギーが量子数 $m$ に依存する($3 B_2^0 (2m+1)$ の項)という\textbf{非調和性(Anharmonicity)}を示している. この「非等間隔なエネルギー梯子(Anharmonic Ladder)」構造こそが, この系をボゾン系と区別し, Dicke模型における「スピン」としての性質を際立たせる要因である. 

\section{ゼーマン・ポラリトンの形成理論}

\subsection{スピン-光子相互作用ハミルトニアン}
キャビティ内の量子化された磁場 $\hat{\mathbf{h}}_{cav}$ とスピン集団との相互作用は, 以下の形になる:

\begin{equation}
H_{int} = \sum_{j=1}^N \hbar g_0 (\mathbf{r}_j) (S_{j,+} + S_{j,-}) (\hat{a} + \hat{a}^\dagger)
\end{equation}

ここで, $g_0$ は単一スピンと単一光子の結合定数である. 

\subsection{集団結合増強と超強結合}
$N$ 個のエミッターがコヒーレントに相互作用する場合, 実効的な結合定数 $g_{coll}$ は $\sqrt{N}$ 倍に増強される:

\begin{equation}
g_{coll} = g_0 \sqrt{N}
\end{equation}

GGG結晶中のスピン密度は非常に高く, $N$ は $10^{18} \sim 10^{21}$ のオーダーに達する. これにより結合強度は数GHzに達し, 比率 $\eta = g_{coll} / \omega_c$ が $0.1$ を超える\textbf{超強結合領域(Ultrastrong Coupling Regime: USC)}に容易に到達する\cite{ref194}. 

\subsection{ポラリトン固有状態と真空ラビ分裂}
ハミルトニアンの対角化により得られる「ゼーマン・ポラリトン」は, 共鳴条件において大きな分裂(真空ラビ分裂)を示す:

\begin{equation}
2\Omega_{Rabi} \approx 2 g_{coll} = 2 g_0 \sqrt{N}
\end{equation}

\section{Dicke模型への写像と温度依存性}

\subsection{一般化Dickeハミルトニアン}
$N$ 個の $S=7/2$ スピンと単一光子モードの結合系に対するハミルトニアンは以下のように書ける:

\begin{equation}
H_{Dicke} = \hbar \omega_c \hat{a}^\dagger \hat{a} + \hbar \omega_Z \hat{S}_z^{tot} + \frac{\hbar \lambda}{\sqrt{2S N}} (\hat{S}_+^{tot} + \hat{S}_-^{tot}) (\hat{a} + \hat{a}^\dagger)
\end{equation}

\subsection{スピン性の発現と熱脱分極}
常磁性体であるGGGでは, 温度 $T$ において各スピンがボルツマン分布に従って準位を占有するため, 実効的な結合強度は準位間の\textbf{占有数差(Population Difference)}に依存する\cite{ref190}. 

\begin{equation}
g_{eff}(T) \propto \sqrt{N \left( P_m(T) - P_{m+1}(T) \right)}
\end{equation}

温度上昇に伴って真空ラビ分裂幅が減少する「熱脱分極(Thermal Depolarization)」の観測事実は, 系がボゾン(無限準位)ではなく, スピン(有限準位)であることを証明する決定的な証拠である. これは, GGG系がDicke物理を探求するプラットフォームたり得ることを示している. 

また, 磁気双極子結合系であるGGGでは, ゲージ不変性に由来する $A^2$ 項(反磁性項)が存在しないため, 超放射相転移(SRPT)を阻害するNo-go定理を回避できる純粋なDicke模型を実現できる点も重要である. 

\section{結論}

本報告書では, GGGにおけるゼーマン・ポラリトン形成の微視的機構について包括的な解析を行った. 
\begin{itemize}
    \item \textbf{微視的起源}: ポラリトンは$\mathrm{Gd}^{3+}$イオンスピンとキャビティ光子の磁気双極子結合に由来し, その微細構造は$D_2$対称性の結晶場(特に$O_2^0, O_2^2$項)による非調和性に支配されている. 
    \item \textbf{非調和性の役割}: 結晶場による準位の不等間隔性は, この系を単純なボゾン系から区別する要素である. 
    \item \textbf{Dicke模型の妥当性}: 温度依存性を持つ真空ラビ分裂は, この系がDicke模型によって正確に記述されることを実証している. 
    \item \textbf{新規性}: 高密度スピンによる超強結合の実現と, $A^2$項の欠如により, 純粋なDicke物理を探求する理想的なテストベッドを提供する. 
\end{itemize}

\begin{thebibliography}{99}
\bibitem{ref176} On standardization and algebraic symmetry of the ligand field Hamiltonian for rare earth ions at monoclinic symmetry sites - AIP Publishing.
\bibitem{ref177} Spectroscopic data on 2-mm laser systems - SPIE Digital Library.
\bibitem{ref178} Energy levels, wave functions, dipole and quadrupole transitions of trivalent gadolinium ions in sapphire - NIST Technical Series Publications.
\bibitem{ref179} Magnetic interactions and short range order in gadolinium gallium garnet - AIP Publishing.
\bibitem{ref182} Quantum mechanical operator equivalents and magnetic anisotropy of the heavy rare earth metals - CORE.
\bibitem{ref183} Quantum mechanical operator equivalents used in the theory of magnetism - SciSpace.
\bibitem{ref185} Electron paramagnetic resonance study of gadolinium(3+) in single crystals of yttrium vanadate.
\bibitem{ref187} On the standardization of crystal-field parameters and the multiple correlated fitting technique: Applications to rare-earth compounds.
\bibitem{ref189} Crystal field parametrizations for low symmetry systems.
\bibitem{ref190} Zeeman polaritons as a platform for probing Dicke physics in condensed matter - arXiv (2024).
\bibitem{ref193} Strong to ultrastrong coherent coupling measurements in a YIG/cavity system at room temperature.
\bibitem{ref194} arXiv:2005.10605v4 [cond-mat.str-el].
\end{thebibliography}

\end{document}